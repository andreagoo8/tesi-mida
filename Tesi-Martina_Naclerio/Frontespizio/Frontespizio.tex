%% !TEX root = .tex
% !TEX TS-program = lualatex
% !TEX spellcheck = it-IT
% !TEX encoding = UTF-8
% !BIB TS-program = biber

\documentclass[%
%draft,%
12pt]{article}

% Pacchetti base per pdflatex
\usepackage[T1]{fontenc} % Dichiara la codifica dei font: così le lettere accentate non vengono rappresentate come sovrapposizioni di lettere e accenti ma con glifi appositi
\usepackage[utf8]{inputenc} % Codifica dei caratteri; deve essere uguale alla codifica dell'editor
\usepackage[%
%greek.ancient,%
italian]{babel} % Dichiara le lingue del documento, l'ultima è quella principale
%\newcommand{\greco}{\foreignlanguage{greek}}

%\renewcommand{\familydefault}{\sfdefault}
\usepackage{times}
%\usepackage{helvet}

%% Pacchetti base per lualatex
%\usepackage{fontspec} % Gestisce i font del sistema
%\setmainfont{Minion3} % Imposta il font principale del documento
%\usepackage{polyglossia} % Corrispettivo di babel
%\setmainlanguage{italian} % Imposta la lingua principale del documento
%\setotherlanguage[variant=ancient]{greek} % Imposta il greco antico come lingua secondaria

% Formattazione generale
\usepackage{microtype}
%\setlength\parindent{0pt}
\usepackage{geometry}
\geometry{a4paper,%
	top=2.5cm,bottom=2cm,left=2cm,right=2cm,%
heightrounded} % modifica di poco le dimensioni della gabbia del testo per farle contenere un numero intero di righe


% Colori
\usepackage[svgnames]{xcolor}


% Figure e tabelle
\usepackage{graphicx}
\usepackage{float} % Per inserire un oggetto mobile nel punto esatto in cui compare nel testo con il comando [H]


\begin{document}

\thispagestyle{empty} % Per non non far comparire il numero di pagina
\begin{figure}
	\centering
	\includegraphics[width=0.2\textwidth]{UniTS_100-blu}
\end{figure}
	
\vspace*{0.75em}

\begin{center}

{\selectfont{\sffamily

	{\Large Dipartimento di Scienze della Vita}

}}

\vspace*{3.5em}

{\selectfont{\sffamily
		{\Large Classe delle Lauree Magistrali in Biologia LM-06\\
		
\vspace*{1ex}
		
		Corso di Laurea magistrale in Ecologia dei cambiamenti globali}
}}



\vspace*{5.5em}

{\linespread{1.75}\selectfont{\sffamily
		{\LARGE \textbf{Effect of algae fortification}}
		
		{\LARGE \textbf{on physicochemical and sensory properties}}
		
		{\LARGE \textbf{of plant-based burgers}}
		
		\textcolor{white}{.}
}}

\vspace*{5.5em}

{\linespread{1}\selectfont
{\sffamily
	\begin{flushleft}
		{\large
		Laureanda:			\hfill Relatore:
		
		Martina Naclerio	\hfill Prof.~Stanislao Bevilacqua \\ % NON è meglio laureando??
\vspace*{1em}
		\hfill Correlatrice:\\
		\vspace*{0.75mm}
		\hfill Dott.ssa~Kricelle Deamici
		}
	\end{flushleft}
}}

\vfill

{\linespread{2}\selectfont{\sffamily
		{\Large Anno accademico 2023/2024}
}}

\end{center}
\end{document}
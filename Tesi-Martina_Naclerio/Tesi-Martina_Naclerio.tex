%% !TEX root = .tex
% !TEX TS-program = lualatex
% !TEX spellcheck = 
% !TEX encoding = UTF-8
% !BIB TS-program = biber

\documentclass[%
%draft,%
%twoside,% #Stampa
12pt]{article}

%% Pacchetti base per pdflatex
%\usepackage[T1]{fontenc} % Dichiara la codifica dei font: così le lettere accentate non vengono rappresentate come sovrapposizioni di lettere e accenti ma con glifi appositi
%\usepackage[utf8]{inputenc} % Codifica dei caratteri; deve essere uguale alla codifica dell'editor
%\usepackage[%
%%greek.ancient,%
%italian]{babel} % Dichiara le lingue del documento, l'ultima è quella principale
%%\newcommand{\greco}{\foreignlanguage{greek}}
%
%\renewcommand{\familydefault}{\sfdefault}
%\usepackage{times}
%%\usepackage{helvet}

% Pacchetti base per lualatex
\usepackage{fontspec} % Gestisce i font del sistema
\setmainfont{Minion3} % Imposta il font principale del documento
\usepackage{polyglossia} % Corrispettivo di babel
\setmainlanguage{english} % Imposta la lingua principale del documento
\setotherlanguage%
%[variant=ancient]%
{greek} % Imposta il greco antico come lingua secondaria

% Formattazione generale
\usepackage{microtype}
\usepackage{indentfirst}
%\setlength\parindent{0pt}
\usepackage{geometry}
\geometry{a4paper,heightrounded} % modifica di poco le dimensioni della gabbia del testo per farle contenere un numero intero di righe
\usepackage{layaureo}
%\usepackage{multicol}
\usepackage{enumitem}
\setlist{itemsep=0.2em}
\setlist[description]{labelsep=0.75em}

% Colori
\usepackage[svgnames]{xcolor}
%\color{white} % #DarkMode
%\definecolor{SfondoGrigio}{RGB}{55,55,55} % #DarkMode
%\pagecolor{SfondoGrigio} % #DarkMode

% Figure e tabelle
\usepackage{graphicx}
\usepackage{tabularx}
\usepackage{float} % Per forzare l'inserimento di un oggetto mobile nel punto esatto in cui compare nel testo con il comando [H]
\usepackage{booktabs} % Produce filetti delle tabelle migliori di quelli standard
\usepackage{subcaption} % Permette di inserire più figure all'interno dello stesso oggetto mobile
\DeclareCaptionSubType[Alph]{figure} % Label delle subfigure in maiuscolo, senza parentesi, con un due punti
\subcaptionsetup{labelformat=simple,labelsep=colon} % Label delle subfigure in maiuscolo, senza parentesi, con un due punti
%\subcaptionsetup{
	%	width=0.9\linewidth, % Utile per quando si hanno sotto-didascalie lunghe e le sotto-figure si toccano: infatti se si lasciasse il default (ovvero la \linewidth) la didascalia di figure ravvicinate potrebbe toccarsi pur senza presenare errori
	%	list=true, % Aggiunge le sotto-figure alla lista delle figure creata con '\listoffigures'
	%	hypcap=true, % Regola il comportamento dei comandi per riferirsi alle figure come '\ref{}':
	%%				- se 'hypcap' è impostato come 'true' il collegamento ipertestuale porta sempre alla sotto-figura
	%%				- se 'hypcap' è impostato come 'false', dipende anche da come è impostato 'hypcap' del pacchetto caption:
	%%					- se 'hypcap' di 'caption' è impostato come 'true', il collegamento ipertestuale porta alla 'figure' generale
	%%					- se 'hypcap' di 'caption' è impostato come 'false', il collegamento ipertestuale porta alla sotto-didascalia
	%}
\usepackage{caption}
\captionsetup{%
	%	indention=\parindent,
	%	width=(\textwidth-(\parindent*2)),%
	hypcap=true, % % Regola il comportamento dei comandi per riferirsi alle figure come '\ref{}': se 'hypcap' è impostato come 'true' il collegamento ipertestuale porta alla figura; se 'hypcap' è impostato come 'false' il collegamento ipertestuale porta alla didascalia
	font=small,labelfont={sf,bf},tableposition=top,figureposition=bottom, % La posizione della didascalia va comunque specificata nel sorgente mettendo \caption{} prima o dopo \includegraphics{}
}

% Definizione dimensioni per sottofigure
\newlength{\hbtwsfig} % Spazio orizzontale fra sottofigure
\setlength{\hbtwsfig}{1em}
\newlength{\vbtwsfig} % Spazio verticale fra sottofigure
\setlength{\vbtwsfig}{2ex}
\newlength{\subfigwth} % Larghezza delle sottofigure: è il \textwidth meno lo spazio orizzontale fra sottofigure diviso due. Studiato per l'inserimento di sole due sottofigure per riga
\setlength{\subfigwth}{(\textwidth - \hbtwsfig) / 2}

\newlength{\trehbtwsfig} % Spazio orizzontale fra sottofigure
\setlength{\trehbtwsfig}{1em}
\newlength{\tresubfigwth}
\setlength{\tresubfigwth}{(\textwidth - (\trehbtwsfig * 2)) / 3}

% Pacchetti scientifici
\usepackage{siunitx}
\sisetup{output-decimal-marker={,},	list-units=single,range-units=single,
	%	detect-all,
	mode=match,propagate-math-font=true,reset-math-version=false,reset-text-family=false,reset-text-series=false,text-family-to-math=true,text-series-to-math=true,
	range-open-phrase={\text{from} },
	%	range-phrase={--},
	%	retain-explicit-plus,
}
\usepackage{chemformula}
\usepackage{amsmath}
\usepackage{amssymb}
\DeclareSIUnit\litre{l}

% Bibliografia automatica
% !TeX encoding = UTF-8

% Bibliografia automatica

% Pacchetto csquotes
\usepackage[autostyle,italian=quotes]{csquotes} % Migliora la formattazione delle citazioni: 'autostyle' adatta lo stile delle citazioni alla lingua corrente del documento, '*lingua*=guillemets/quotes' racchiude automaticamente tra virgolette caporali o virgolette ad apice i campi che prevedono le virgolette

% Pacchetto biblatex
\usepackage[%
style=authoryear, % Specifica lo stile bibliografico e lo stile di citazione,
uniquename=mininit,uniquelist=minyear,% Con 'style=authoryear' se ci sono due reference con lo stesso primo autore, per evitare ambiguità viene citato anche il secondo, anche se l'anno è diverso. Con queste opzioni, se l'anno è diverso allora quest'ultimo viene usato per la disambiguazione, mentre se anche l'anno è uguale viene citato il secondo autore
maxcitenames=1,maxbibnames=2,% Il numero massimo di nomi che vengono elencati nella citazione (cite) e nella bibliografia (bib); superato il valore, viene inserito un nome seguito da "et al."
sorting=none,% L'ordine con cui inserire le opere nella bibliografia; con 'none' vengono inserite nell'ordine in cui compaiono nel testo
dashed=false, % L'opzione true (default), in caso di voci consecutive degli stessi autori nella bibliografia, inserisce una linea tratteggiata al posto dei nomi degli autori (utile quando sono citate molte opere dello stesso autore)
autolang=hyphen,
hyperref=true,backend=biber]{biblatex} % Compone la bibliografia: 'hyperref' rende le citazioni dei collegamenti cliccabili, 'backend' imposta biber come motore bibliografico

% Altre impostazioni
\DeclareNameAlias{sortname}{given-family} % Formatta i nomi degli autori secondo lo schema "Nome Cognome"; il contrario si ottiene con 'family-given'
\AtEveryBibitem{\clearfield{month}} % Omette la stampa del mese del field 'date'
\AtEveryBibitem{\clearfield{day}} % Omette la stampa del giorno del field 'date'
\AtEveryBibitem{\clearfield{note}\clearfield{addendum}} % Omette la stampa del field 'note'
\AtEveryCitekey{\clearfield{note}\clearfield{addendum}} % Fa qualcos'altro riguardo il field 'note'
\AtEveryBibitem{\clearfield{urlyear}} % Omette la stampa del campo 'urldate'

% Stampa i fields 'url', 'eprint' e 'issn' solo se il field 'doi' non è presente
\DeclareSourcemap{
	\maps[datatype=bibtex, overwrite]{
		\map{
			\step[fieldsource=doi, final]
			\step[fieldset=url, null]
			\step[fieldset=eprint, null]
			\step[fieldset=issn, null]
		}
	}
}


% Rende tutta la citazione nel corpo del testo cliccabile
\DeclareCiteCommand{\cite}
{\usebibmacro{prenote}}
{\usebibmacro{citeindex}%
	\printtext[bibhyperref]{\usebibmacro{cite}}}
{\multicitedelim}
{\usebibmacro{postnote}}

\DeclareCiteCommand*{\cite}
{\usebibmacro{prenote}}
{\usebibmacro{citeindex}%
	\printtext[bibhyperref]{\usebibmacro{citeyear}}}
{\multicitedelim}
{\usebibmacro{postnote}}

\DeclareCiteCommand{\parencite}[\mkbibparens]
{\usebibmacro{prenote}}
{\usebibmacro{citeindex}%
	\printtext[bibhyperref]{\usebibmacro{cite}}}
{\multicitedelim}
{\usebibmacro{postnote}}

\DeclareCiteCommand*{\parencite}[\mkbibparens]
{\usebibmacro{prenote}}
{\usebibmacro{citeindex}%
	\printtext[bibhyperref]{\usebibmacro{citeyear}}}
{\multicitedelim}
{\usebibmacro{postnote}}

\DeclareCiteCommand{\footcite}[\mkbibfootnote]
{\usebibmacro{prenote}}
{\usebibmacro{citeindex}%
	\printtext[bibhyperref]{ \usebibmacro{cite}}}
{\multicitedelim}
{\usebibmacro{postnote}}

\DeclareCiteCommand{\footcitetext}[\mkbibfootnotetext]
{\usebibmacro{prenote}}
{\usebibmacro{citeindex}%
	\printtext[bibhyperref]{\usebibmacro{cite}}}
{\multicitedelim}
{\usebibmacro{postnote}}

\DeclareCiteCommand{\textcite}
{\boolfalse{cbx:parens}}
{\usebibmacro{citeindex}%
	\printtext[bibhyperref]{\usebibmacro{textcite}}}
{\ifbool{cbx:parens}
	{\bibcloseparen\global\boolfalse{cbx:parens}}
	{}%
	\multicitedelim}
{\usebibmacro{textcite:postnote}}

%% Specifica da quale file prendere le referenze bibliografiche
%\addbibresource{Bibliografia-.bib}

\addbibresource{bibliography/bibliografia-problematica.bib} % Specifica da quale file prendere le referenze bibliografiche
\addbibresource{bibliography/bibliografia-zotero.bib} % Specifica da quale file prendere le referenze bibliografiche

% Altri pacchetti
%\usepackage[useregional]{datetime2}
\usepackage[numberlinked=true,symbol=$\!^\wedge\,$]{footnotebackref} % Da commentare per la #Stampa
\hyphenation{%
	pyro-se-quencing %
	li-ga-to %
	o-li-go-nu-cleo-ti-de o-li-go-nu-cleo-ti-di %
	a-dap-tor a-dap-tors %
	for-ward %
	re-verse %
	pri-mer %
} % Regola la sillabazione: si inseriscono le parole con la propria sillabazione separando le sillabe con uno trattino (più parole vanno separate con uno spazio) % Regola la sillabazione: si inseriscono le parole con la propria sillabazione separando le sillabe con uno trattino (più parole vanno separate con uno spazio)
% Per regolare la sillabazione solo localmente, si possono inserire direttamente nel testo:
	% '\-' specifica direttamente i punti della parola in cui è permessa la sillabazione
	% \mbox{} mantiene unita la parola al suo interno e ne impedisce la sillabazione

% Nuovi comandi
\newcommand{\ap}[1]{\textsuperscript{#1}}
\newcommand{\ped}[1]{\textsubscript{#1}}
\newcommand{\wariv}{\colorbox{Yellow}{.}} % Soft-D@RiV
\newcommand{\zariv}{\colorbox{Yellow}{\textcolor{Crimson}{D@RiV}}} % Hard-D@RiV
\usepackage{soul}
\newcommand{\zxriv}[1]{\textcolor{Crimson}{\hl{#1}}} % Hard-D@RiV
\newcommand{\species}[1]{\textit{#1}}
%\newcommand{\pianta}[1]{\textit{#1}}

% Pacchetti/comandi temporanei
\usepackage{pdfpages}

% Altro
\overfullrule=1cm
\usepackage{hyperref} % Va caricato per ultimo (ma prima di bookmarks)
\hypersetup{%
	pdftitle={},
	pdfauthor={Martina Naclerio},
	pdfsubject={},
%	pdfkeywords={}, % le keywords vanno separate con una virgola
%	pdfcreator={},
	colorlinks=true,
	linkcolor=MidnightBlue, % colore dei normali interlink
	citecolor=red, % colore delle citazioni bibliografiche nel testo
	urlcolor=DarkRed, % colore dei collegamenti \url{} al web
%	hidelinks % rende tutti i collegamenti del colore principale e senza riquadri (utile per la #Stampa)
}
\usepackage{bookmark} % Produce l'indice interno al pdf e permette di personalizzarlo meglio rispetto al solo hyperref

% Glossario
\usepackage[abbreviations]{glossaries-extra}
\setabbreviationstyle[acronym]{long-short}
\loadglsentries{glossary}

%	Caratteri per copia-incolla rapidi:
%		’ ' apostrofo curvo e dritto
%		“” " virgolette aperte, chiuse e indifferenziate

\begin{document}
\pagenumbering{Roman}
\pagestyle{empty}


%\pdfbookmark[1]{Frontespizio}{frontespizio}
%\includepdf{Frontespizio/Frontespizio-nofirma.pdf}


%\cleardoublepage
%\input{Dedica.tex}


\cleardoublepage
\thispagestyle{empty}
\pdfbookmark[1]{Indice}{indice}
{\hypersetup{linkcolor=black}%
	\tableofcontents%
}


\cleardoublepage
\pagestyle{plain}
%\input{Ringraziamenti.tex}


\cleardoublepage
\pagenumbering{arabic}
\section*{Abstract}
\addcontentsline{toc}{section}{Abstract}
Algae are promising organisms for the production of sustainable products to be used as raw materials for food, animal feed, chemicals, biofuels, and much more. In this period of change and demographic growth, it will be necessary to focus on the potential of these organisms as a sustainable food resource, also considering the European environmental regulations and commitment towards an Ecological Transition and a new concept of Blue Economy.
Initiatives such as the “EU Green Deal”, “Farm to Fork” and the “EU Algae Strategy” are just some examples promoting the adoption of more sustainable and environmentally friendly dietary practices. %#Andrea usare corsivo?
From an overview of the characteristics of algae, their cultivation methods, and their multiple uses, the benefits they offer for both human health and the ecosystem will be evident. Through an experimental project, the integration of algae into plant-based burgers has been examined, and by analyzing the nutritional and sensory aspects of this innovative food proposal, we seek to outline a path towards a more aware and sustainable future.
(When I have the results of the statistical analysis, I will add a more detailed section on what has been done, the main findings, and a conclusion section).


%\cleardoublepage
% !TeX encoding = UTF-8

\section{Introduction}
The following project provides an overview of the benefits and applications of algae in the Blue Economy. By examining this concept, along with the Ecological Transition and strategies implemented at the European level, I explored some of the diverse applications of algae, particularly focusing on their use in human nutrition. These organisms represent one of the most abundant and sustainable resources on Earth and play a crucial role in promoting the Blue Economy and ecological transition by offering a wide range of commercial opportunities and applications. Algae can be utilized to produce food, biofuels, cosmetics, agricultural fertilizers, and much more, thereby helping to reduce dependence on fossil resources and minimize the negative environmental impact of human activities.


\subsection{European Regulations}
The \gls{unfccc} is the primary international agreement on climate action adopted in Rio in 1992. When it was realized that collective action was necessary to protect people and the environment and to limit greenhouse gas emissions, the Kyoto Protocol was concluded to reduce these emissions. On November 4, 2016 the Paris Agreement entered into force, representing the renewed commitment of nations to combat and limit global warming, with the aim of creating a zero-climate-impact society by 2050~\parencite{paris_agreement}.

In addition, in 2015, the Agenda 2030 was born, a program of action comprising 17 \glspl{sdg} to be achieved in the environmental, economic, social, and institutional spheres, a goal that calls for global efforts aimed at ensuring a better present and future for our planet, committing to eliminating or finding solutions to the issues that concern us globally. This Agenda is based on five key concepts: people, prosperity, peace, partnership, and planet, and its objectives are all closely interconnected. The \glspl{sdg} are universal and as such must be achieved by all countries, which will need to define their own strategies in order to implement them~\parencite{agenda_2030}.

In 2019, three years later, the European Parliament declared a climate emergency and established a key strategy of the European Union, the Green Deal. It consists of a series of action plans aimed at making Europe the first continent to achieve climate neutrality by 2050 (in line with the Paris Agreement) and leading it towards a green transition by implementing economic, energy, industrial, and agricultural initiatives. In addition to climate neutrality, the Green Deal has two other main purposes: circular economy and environmental protection. This project includes legislations and initiatives to achieve the established objectives. The first climate-related legislation is the “Fit for 55\%” package, which aims to adapt the current European regulations to align with the goals of the Green Deal. Following this, achieving climate neutrality becomes a legal obligation for the EU, with a commitment to reduce net greenhouse gas emissions by 55\% by 2030 compared to 1990 levels. Furthermore, adaptation strategies to climate change are implemented to enhance understanding and effects, and to create appropriate solutions to address them, such as adapting civil protection through prevention, preparedness, response, and recovery.

There is also an industrial strategy, the objective of which is to support the industry in its role as an accelerator and driver of change, innovation, and growth, guiding us towards a green transformation. We must not overlook the action plan for the circular economy, a crucial step for economic growth, an increase in sustainable products, and better resource use, both for production and consumption. Additionally, it is essential to consider that achieving climate neutrality will not be equitable for all; for some member states, it will be more challenging, especially those dependent on fossil fuels or with high-carbon-intensive industries. This is why the EU has introduced a mechanism for a just transition, providing financial and technical support to these regions. Regarding greenhouse gas emissions, we know that 75\% of them are attributable to the production and use of energy. In this regard, the Green Deal proposes objectives for affordable and secure energy through the use of cleaner sources such as renewables, also revising and amending current legislation accordingly.

Further lines of action are: biodiversity, where the Green Deal aims to restore biodiversity in Europe by 2030 through ecosystem restoration actions, reducing pesticide use, expanding land and marine areas, and increasing funding for the implementation of these goals; chemicals, which are also approached from a sustainability perspective to preserve the environment and human health; forests and deforestation, which is linked to the biodiversity plan and the reduction of greenhouse gas emissions. This involves offering financial incentives and proposing the planting of new trees to address deforestation issues~\parencite{green_deal}.

In May 2020, the European Commission presented the “Farm to Fork” strategy as one of the key actions of the Green Deal, also aligning with the Sustainable Development Goals mentioned in the Agenda 2030, aiming to make the current food system more sustainable. The objectives of this initiative consider food security from various aspects, including the adoption of organic farming, a plan for food safety, the promotion of environmentally friendly agricultural practices that allow carbon sequestration in soils, sustainable aquaculture facilities, and proper labeling and marketing standards for products. All of this is aimed at guiding Europe towards a fair, healthy, and environmentally respectful food system~\parencite{farm_to_fork}.


\subsection{Ecological Transition and Blue Economy}
When discussing the Ecological Transition, we cannot overlook the \gls{ngeu} recovery plan, of which it is a key focus, linking with the perspectives of the Green Deal. This project aims to create a healthier and greener future for Europe through a wide range of investments, facilitated by the \gls{rrf} and the \gls{nrrp}.

\gls{ngeu} focuses on six key sectors, one of which is the Ecological Transition, providing momentum to the green transition by promoting renewable energies, sustainable mobility, enhancing waste recycling, and much more.

In more detail, Europe commits to:\footnote{\zxriv{ho messo in minuscolo le iniziali dei punti}}
\begin{itemize}
\item improve the quality of water in rivers and seas, reduce polluting waste and plastics, plant billions of trees, and repopulate the world with bees;
\item make agriculture more environmentally friendly so that our food is healthier;
\item create green spaces in our cities and utilize renewable energies more effectively~\parencite{next_gen_eu}.
\end{itemize}

The ecological transition is a process aimed at transforming production and consumption patterns to make the economy more sustainable and environmentally friendly. This involves adopting practices and technologies that reduce environmental impact, promoting energy efficiency, the use of renewable energies, and the adoption of sustainable natural resource management strategies. It is based on principles of equity, social justice, and environmental responsibility, aiming to reconcile economic development with environmental preservation and therefore plays a fundamental role in addressing current challenges related to it. Through the promotion of sustainable practices and policies, the ecological transition can contribute to mitigating climate change, reducing pollution, preserving biodiversity, and ensuring a better future for future generations. Additionally, it fosters the creation of new jobs in renewable energy, energy efficiency, and sustainable technology sectors, stimulating innovation and the green economy.

An example of an approach to the responsible use of natural sources is introduced by the strategy "Innovating for Sustainable Growth: A Bioeconomy for Europe" (a project connected to the Green Deal and Farm to Fork), initiated in 2012 and subsequently updated in 2018.

Our economy, based on fossil fuels, has reached its limits; therefore, we need to revolutionize the way we produce and consume, respecting our planet. The purpose of the bioeconomy is precisely this: it covers all sectors focused on biological resources and connects terrestrial and marine ecosystems with a view to sustainability. Furthermore, to build a zero-emission future, a bioeconomy aiming at the use of renewable resources for energy production is necessary~\parencite{eucommission_sustainable_2018}.

To achieve the goals of the Green Deal, increasing attention is being paid to the role that marine ecosystems and the blue economy can play and the opportunities they can offer as a source of resources. The Agenda 2030 recognizes that without healthy oceans, life on our planet is at risk, and without these resources, societies lose the ability to sustain themselves. This is why European thinking is shifting from the simple concept of the bioeconomy to the idea of a sustainable blue bioeconomy, which promotes the responsible management of oceans and marine resources to ensure economic benefits without compromising ecosystem health. This can contribute to carbon neutrality through the development of renewable energies produced at sea (offshore) and by making maritime transport and ports more environmentally friendly. Through better use of marine resources and the selection of alternative sources of food and feed, the blue economy can help alleviate the pressure on the climate and natural resources from food production~\parencite{sustainable_blue_economy_2021}.

In line with the "Farm to Fork" and projects for a blue bioeconomy, in November 2022, the European Commission announced the EU Algae Strategy, "Towards a strong and sustainable algae sector," highlighting the role of algae as an alternative protein source in the perspective of a sustainable food system~\parencite{blue_bioeconomy}.

To fully exploit the opportunities offered by the algae sector in the EU, it is necessary to promote the integration and development of markets related to their applications and consequently promote increased cultivation and production of them throughout the territory.

To this end, this strategy identifies 23 actions aimed at:
\begin{enumerate}
\item improving governance framework and regulations;
\item enhancing the business environment;
\item addressing knowledge gaps, research, technology, and innovation;
\item increasing social awareness and market acceptance of algae and algae-based products in the EU~\parencite{kuech_future_2023}.
\end{enumerate}

Algae indeed have significant exploitable potential useful for addressing the issues affecting our world. The aim is to create a high-level industry that plays a significant global role, both in terms of climate change, food security, and supporting the marine ecosystem and economic growth~\parencite{seaweed_manifesto}.

\subsection{Algae}
Algae are photoautotrophic organisms that have played and continue to play a crucial role in shaping the planet's biosphere. It is believed that four billion years ago, the Earth's atmosphere was completely different from what we know today; it was a hostile environment devoid of oxygen where life was almost impossible. This changed when the first cyanobacteria appeared, utilizing light and pigments such as chlorophyll a and phycobiliproteins to transform \ch{CO2}, producing the oxygen that enabled life to thrive and creating a world rich in diverse living beings~\parencite{mayfield_algae_2021}.

For centuries, algae have been utilized by humans for various purposes, serving economic and social roles. For example, algae were an important resource in prehistoric times, likely for food or commercial purposes. References dating back around \num{5000} years demonstrate their use in traditional Chinese and Aboriginal medicine. Their use in human diet dates back to the \zxriv{4th} century in Japan~\parencite{jacquin_Selected_2014} and ancient Greeks utilized these organisms as fertilizers, while their use as fertilizer was a common practice for ancient Romans~\parencite{jacquin_Selected_2014}.

Algae are a group of autotrophic, unicellular organisms that, thanks to their ability to adapt to various environmental conditions, can be found in diverse habitats ranging from extreme conditions to more favorable environments. They can be found in saline oceans, freshwater lakes, rivers, and ponds, as well as in various types of soil and rocks, or in areas with freezing temperatures (such as the Himalayas) to those with the hottest temperatures (deserts). They are also often found in symbiotic association with plants and animals~\parencite{rindi_Diversity_2007}. Algae are often found in symbiotic association with plants and animals. They possess a flagellum and have a well-developed nucleus, a cell wall, and a chloroplast containing chlorophyll and other pigments. Their shape varies from small unicellular organisms (\qty{1}{\micro\metre}) to large multicellular forms such as kelp (approximately \qty{60}{\metre})~\parencite{sahoo_Algae_2015} and based on this, they are classified into two different groups: Microalgae, visible only under a microscope, and Macroalgae, visible to the naked eye. Microalgae are traditionally classified based on their cytological and morphological aspects, type of reserve metabolites, components of the cell wall, and pigments. For example, cyanobacteria (blue-green algae) contain chlorophyll a and blue phycocyanins, while the brownish colorations given by xanthophyll pigments are typical of diatoms. On the other hand, macroalgae are classified based on their chemical and morphological characteristics, particularly the presence of specific pigments. In this case, we have red algae (Rhodophyceae), characterized by phycoerythrin and phycocyanin; brown algae (Phaeophyceae) for the presence of fucoxanthin, and green algae (Chlorophyceae) containing chlorophyll a and b~\parencite{scieszka_Algae_2019}.

These organisms possess a vegetative body called a thallus, which lacks the differentiation of typical terrestrial plants into roots, stems, and leaves, and the peculiar vascular system. Furthermore, this is a distinctive organ among the various species. Algae, therefore, based on the morphology of the thallus, exhibit different shapes and consistencies: they can be filamentous, cartilaginous, laminar, spongy, calcareous, or they can have a cylindrical axis or even be flattened.

Moreover, these organisms exhibit two types of reproduction: sexual and asexual. The former requires a fusion between sex cells through meiosis and is divided into isogamy if both gametes are mobile and anisogamy if the male gamete encounters the female one. As for asexual reproduction, which does not require fertilization between two gametes, it can occur through thallus fragmentation, propagules, or by spores~\parencite{pereira_Macroalgae_2021}.


\subsection{Algae as a renewable resource: cultivation methods, uses, and benefits for humans and the environment}
Green cities play a crucial role in the battle against the climate emergency and in ensuring a sustainable future for the next generations. Therefore, in facing future challenges, cities must be at the forefront, focusing their efforts on the development of ecological communities. One of the paramount issues is the integration of renewable energy sources and the implementation of an intelligent energy distribution system, ensuring a fair distribution of energy resources~\parencite{chew_Algae_2021}.

Algae are a crucial group both for the ecosystem and for the organisms inhabiting it. In fact, besides being photosynthetic, they are also an important direct source of nutrition for many species, including humans. Apart from their significant nutritional qualities, algae have considerable ecological importance. They play a fundamental role in transitioning towards a more sustainable and environmentally responsible society, representing a renewable resource of particular interest due to their ability to grow rapidly and regenerate continuously. This capability offers the potential to reduce dependence on non-renewable energies and limit the environmental impact associated with their extraction and combustion. Furthermore, algae offer multiple possibilities for use in critical sectors such as food, energy, and industry, making them a valuable asset for sustainable development.


\subsubsection{Cultivation system}
\zxriv{Rivedere posizionamento figure}

Referring to the energy use of algae, a \gls{lca} study has highlighted how the impacts from the cultivation phase are the main contributors to environmental exacerbations~\parencite{clarens_Environmental_2010}. In this regard, it is appropriate to have an overview of the different methods of algal cultivation. These are mainly divided into open (\zxriv{Fig. 1A}) and closed systems. The former are preferred for large-scale commercial production, as they are less expensive to install, manage, and maintain~\parencite{roselet_Comparison_2013}. Nevertheless, due to their open nature, they are subject to risks of contamination and evaporation, limiting control over environmental parameters such as temperature, salinity, and irradiation, and require extensive space. On the other hand, closed systems are characterized by more precise control over environmental conditions, which improves control over species composition and growth conditions. This leads to an overall higher biomass and/or lipid yield and a higher energy density of harvested algae, reducing space requirements, increasing light availability, and decreasing contamination issues. However, closed systems are more complex and expensive to build and operate, making them less suitable for large-scale commercial production as they are difficult to scale up to meet production demands~\parencite{resurreccion_Comparison_2012}. Furthermore, they are subject to bio-fouling, overheating, growth of benthic algae, cleaning issues, and high accumulation of dissolved oxygen, leading to growth limitations~\parencite{narala_Comparison_2016}.

\begin{figure}[H]
\centering
% Subfigure 1
	\subcaptionbox%
		[]%
		{Open Pond culture system\label{subfig:}}%
		{\includegraphics[width=\tresubfigwth]{example-image}}%
\hspace*{\trehbtwsfig}%
% Subfigure 2
	\subcaptionbox%
		[]%
		{Photobioreactor cultivation system\label{subfig:}}%
		{\includegraphics[width=\tresubfigwth]{images/photobioreactor}}%
\hspace*{\trehbtwsfig}%
% Subfigure 3
	\subcaptionbox%
		[]%
		{Two-stage microalgae cultivation system (hybrid)\label{subfig:}}%
		{\includegraphics[width=\tresubfigwth]{images/hybrid_system}}%
\caption%
[]%
{Algae culture system}
\label{fig:}
\end{figure}


In terms of algal cultivation methods, there are various techniques utilized, including cultivation in open ponds, tubular or flat-panel photobioreactors, and hybrid systems that combine elements of both. Each method has its own characteristics in terms of efficiency, cost, and environmental control, and the choice depends on the specific project requirements and available resources.

Open ponds are large shallow tanks where algae are cultivated using sunlight and nutrients present in the water. They fall under open systems because they are not completely enclosed and are exposed to the surrounding environment. They are easy to manage and construct but are, of course, susceptible to contamination and environmental fluctuations.

Photobioreactors (\zxriv{Fig. 1B}) are closed systems consisting of transparent tubes or panels containing algae, allowing for greater control of growth conditions such as temperature, light, and nutrients. They offer more precise environmental control and greater protection from external contamination, but they are more expensive to build and require higher maintenance.

Hybrid systems (\zxriv{Fig. 1C}) combine elements of both open and closed systems to harness the advantages of each, thus separating biomass growth from lipid accumulation. Algae are initially cultivated in an open environment, such as ponds or lakes, to utilize sunlight and reduce initial costs. Subsequently, the algae can be transferred to closed photobioreactors for a more controlled growth phase. In this configuration, the open phase provides ample area for initial cultivation and easy access to sunlight, while the closed phase allows for greater control of growth conditions such as temperature, \ch{CO2} concentration, and nutrition. A recent LCA found that hybrid cultivation has a reduced environmental impact compared to open and closed systems, making it preferable to these alternatives~\parencite{narala_Comparison_2016}.


\subsection{Algae, ecological importance: uses and benefits}
Algae play a fundamental role in maintaining environmental balance and promoting the health of our planet, offering a range of essential benefits for the ecosystem.

One example is their role as renewable resources in biofuel production, following the rapid increase in oil prices, depletion of reserves, and growing awareness of the environmental damage caused by the use of fossil fuels. Attention is shifting towards the development of alternative technologies that are more robust and secure. Furthermore, as the population grows, so does the energy demand, making it increasingly challenging to meet this demand with current energy sources~\parencite{faruk_role_2023}.

In the biofuel market, the potential of algae is increasingly growing, thanks to their characteristics as energy producers. They have a simple cellular structure and are rich in lipids (\zxriv{from 40\% to 80\%} of dry weight), producing a large quantity compared to traditional crops. Moreover, they have a very fast reproductive rate, and algal biofuels are biodegradable, non-toxic, and sulfur-free. Additionally, algae can convert almost all the energy contained in biomass residues and waste into methane and hydrogen~\parencite{suganya_Macroalgae_2016}. We know that the combustion of fossil fuels releases \ch{CO2} and other greenhouse gases, whereas algae are capable of absorbing \ch{CO2} from gases emitted by factories and power plants, producing significant amounts of biomass. To convert it, this is dehydrated and subjected to acid pretreatment, after which its carbohydrates are broken down into monomers. The sugars released are then converted into other products, including ethanol or combustible hydrocarbons or chemicals~\parencite{salami_AlgaeBased_2021}.

Estimates from the Pacific Northwest National Laboratory model have suggested that algal biofuels, particularly biodiesel, have the potential to meet up to 17\% of the demand for transportation fuel~\parencite{dalrymple_Wastewater_2013}. By cultivating algae as a source of fuel, it is possible to slow down the increase in atmospheric and oceanic \ch{CO2}, thereby limiting the rise in global temperatures~\parencite{raven_possible_2017}. Additionally, these organisms can enhance the sequestration of organic carbon for extended periods, depositing it on the seabed and in aquatic sediments. Thus, along with the reduction of \ch{CO2} and temperature increases, ocean acidification can also be mitigated~\parencite{prasad_Role_2021}.

In addition to providing lipids for fuel production, algae farming and cultivation also play a role in mitigating these changes. In fact, it is possible for a 1~hectare algae pond to sequester one ton of \ch{CO2} per day~\parencite{proksch_growing_2013}.

Algal cultivation offers numerous agricultural advantages compared to conventional plants, including:
\begin{enumerate}
\item high yield per unit of land, allowing for optimized land use;
\item efficient water usage for biomass production;
\item full utilization of the plant for various applications;
\item significant production of proteins, lipids, and vitamins per unit of land;
\item utilization of carbon as the primary resource for growth~\parencite{sivakumar_role_2013s}.
\end{enumerate}

Furthermore, algae exhibit greater energy efficiency compared to food crops, yet they do not compete with them. Moreover, their photosynthesis occurs at a rate three to five times faster than that of plants~\parencite{salami_AlgaeBased_2021}.

Algal cultivation, not competing with others, indeed does not require traditional agricultural land; they can be grown in marine or lagoon environments, with wastewater or saline water, thereby reducing the impact on freshwater resources~\parencite{narala_Comparison_2016}. Additionally, algae play an important role as fertilizers for agriculture. They are rich in minerals and nutrients useful for this purpose and have the ability to absorb and retain water in the soil, particularly beneficial in dry and arid lands. Moreover, algae can fix atmospheric nitrogen, a crucial component for plant growth, making it available for them. Another favorable aspect is that these organisms confer greater resistance to diseases and insects~\parencite{garima_diverse_2015}.

Furthermore, it has been observed that crops of fruits, vegetables, and flowers treated with algal fertilizers exhibited increased vigor, improved nutrient absorption, higher yields, successful seed germination, and better preservation~\parencite{rupawalla_Algae_2021}.

Another advantageous use of algae is in the treatment of wastewater, which is also responsible for greenhouse gas emissions. In recent years, to reduce environmental impact, there has been a need to research processes that reduce energy consumption and seek eco-friendly and sustainable alternatives to replace those already in use.

The strategic use of algae in wastewater treatment is based on their ability to utilize both organic and inorganic carbon sources, as well as the nitrogen and phosphorus elements present in inorganic form in wastewater, to fuel their growth process. This mechanism leads to a decrease in the concentration of such compounds within the water body. In this synergy, the main benefit of integrating algae into the wastewater treatment process lies in the production of oxygen through photosynthesis. This \ch{O2} is crucial for the heterotrophic bacteria involved in the biodegradation process of organic materials present in wastewater, thereby accelerating the degradation rate of carbon compounds~\parencite{mohsenpour_Integrating_2021}.

Algae utilize pollutants as energy and nutrient resources for their growth, producing biomass. The process of bioremediation of pollutants by microorganisms, included in treatment operations, occurs through two main mechanisms: bioaccumulation and bioabsorption, followed by subsequent biodegradation. In bioaccumulation, pollutants are absorbed inside the cell to be metabolized, while in bioabsorption, pollutants bind to the cell membrane forming complexes, which are then removed from the environment. The latter mechanism, in particular, represents an effective strategy for the removal of heavy metals from wastewater, as algal cells can incorporate such metals into their growth and development. Algal varieties demonstrate the ability to metabolize a wide range of pollutants, both organic and inorganic, present in wastewater, producing intermediate metabolites through the action of algal enzymes. The algal biomass generated during this process can be considered a source of renewable energy, indicating algae-mediated wastewater treatment as a highly recommended practice~\parencite{bhatt_Algae_2022}.

Recently, there has been a growing interest in using algae for the production of bioplastics, offering a biomass rich in hydrocarbons useful for extracting high-purity cellulose, a key material in their production. These organisms are a promising renewable source, capable of doubling their biomass in a single day and growing in a variety of environments without the need for cultivable land. Scientific research is focusing on potential algal species and the compounds derived from them, such as \gls{pha}, which can be used to develop bioplastics with various industrial applications, including cosmetics, pharmaceuticals, food packaging, and medicine. Furthermore, studies are exploring the possibilities of using these organisms to create a sustainable circular economy, thus contributing to reducing the use of petroleum-derived plastics~\parencite{dang_Current_2022}. Therefore, we can say that algae play a fundamental role in human life and in the mitigation of pollution and critical effects caused by global changes.


\subsubsection{Algae, properties and uses for human health and in future food}
Scientific studies have demonstrated how algae play a crucial role in promoting human health, thanks to their biochemical composition rich in nutrients and bioactive compounds. Among the many areas where these organisms prove useful, cosmetics stand out, where they serve as antioxidants, thickeners, and binding agents. By stimulating skin elasticity and renewal, algae have anti-aging and anti-cellulite properties, can cleanse, tone, and detoxify the skin, as well as increase its brightness and hydration. They have anti-inflammatory and regenerative properties, thus contributing to reducing wrinkles by acting as moisturizers, and additionally have softening and shining effects on hair~\parencite{garima_diverse_2015}.

It is well known that algae are a rich source of metabolites with different activities beneficial to human health, including anti-HIV, antifungal, antitumor, antimalarial, and antimicrobial properties. They are rich in antioxidants, which prevent oxidative damage by eliminating free radicals and reactive oxygen species, thus limiting the onset and formation of tumor cells. Moreover, they are extremely useful compounds in the fight against chemical agents and diseases such as atherosclerosis, cardiovascular disorders, aging processes, and cancer. 

Research on the pharmacological properties of algae has become increasingly important, especially considering the growing resistance to antibiotics and the need for new sources of antimicrobial drugs. Many of these organisms produce antibiotic substances capable of inhibiting bacteria, viruses, fungi, and other organisms. It seems that the antibiotic characteristic depends on various factors, including the specific algae species, microorganisms, season, and growth conditions~\parencite{raja_biological_2013}.

It has been demonstrated that some pigments, such as phycocyanin found in cyanobacteria, are beneficial in the treatment of diseases such as Alzheimer's and Parkinson's. They can also treat ulcers and reduce the onset of heart diseases~\parencite{subhashini_Molecular_2004}. Algae have many other medical properties in addition to those listed so far, and they also have crucial nutritional functions beneficial to human health. Considering omega-3 fatty acids, it has long been known that these polyunsaturated lipids play a fundamental role in human physiology, influencing various aspects of health, including the development of the nervous system and cardiovascular function. It has always been thought that these oils were produced by fish, but this is not the case; algae play a crucial role in the production of these fatty acids, which are then transferred along the food chain to reach humans through the consumption of fish products~\parencite{mayfield_algae_2021}.

Thanks to their nutritional value, algae are commonly used as dietary supplements, and their inclusion in the diet provides a healthy protein intake. Additionally, these organisms promote the body's detoxification process, preserve the integrity of the gastric mucosa, and facilitate digestion~\parencite{scieszka_Algae_2019}. To consider algae as a new renewable food source, a crucial factor to consider is obviously their nutritional content, which varies based on various growth factors such as environment, light, and temperature. Moreover, it also varies from species to species.

One of the most abundant compounds in algae is protein; in fact, approximately 50\% of the \zxriv{dry weight (dw)} of various species can consist of them. The highest protein contents are found in \zxriv{Spirulina} (50–70\% dw) and \species{\zxriv{Chlorella}} (51–58\% dw). These macromolecules are essential for the human diet and provide most of the nitrogen and amino acids needed by humans. Essential amino acids are referred to, which are fundamental organic molecules that cannot be synthesized endogenously and must therefore be incorporated through diet. Algal proteins are rich in essential amino acids, making them a complete protein source~\parencite{torres-tiji_Microalgae_2020}.

The lipids found in algae are relatively modest in quantity, constituting only about 1-5\% of their dry weight. Despite their modest presence, they provide a significant contribution to human health as a low-energy food source. Approximately half of the lipids present consist of polyunsaturated fatty acids, including \gls{epa} and \gls{aa}. These have been shown to have beneficial effects on health, helping to regulate blood pressure, blood clotting, and reducing the risk of chronic diseases such as cardiovascular diseases, osteoporosis, and diabetes. Red and brown algae are particularly rich in these fatty acids, while green algae tend to contain predominantly hexadecatetraenoic, oleic, and palmitic fatty acids. We know that lipids and fatty acids are essential components of cells and are precursors to many essential molecules, making their adequate intake fundamental to the human diet.

Algae are also rich in essential minerals and trace elements, which are crucial for health as they contribute to tissue formation and regulate many vital reactions as cofactors of metalloenzymes. Minerals are absorbed by these organisms through their cell surface polysaccharides, allowing them to accumulate significant amounts of minerals from the surrounding environment. The mineral composition of algae varies depending on the type, season, geographic location, and cultivation method, but they can contain calcium, phosphorus, magnesium, potassium, sulfur, and sodium. Therefore, these organisms represent a valuable source of minerals and trace elements and are used as dietary supplements to ensure an adequate intake of essential nutrients.

Vitamins, considered fundamental catalysts for human metabolism, must be obtained through the diet as the human body has limited capacity to synthesize them independently. Algae have been identified as a rich source of various vitamins, including vitamin A, C, B-group vitamins (B1 thiamine, B2 riboflavin, B3 niacin, B6 pyridoxine, B12 cobalamin), and vitamins E and K, along with a variety of carotenoids. Red and brown algae are particularly rich in B-group vitamins, while brown algae tend to have a higher content of vitamin E compared to others. Generally, the levels of vitamins in these organisms vary depending on the species, harvesting period, and environmental conditions. Algae, therefore, can constitute a significant source of vitamins in the human diet, contributing to maintaining an optimal nutritional status~\parencite{tiwari_Seaweed_2015}.

Malnutrition is a widespread issue affecting over two~billion people worldwide, with one in nine suffering from chronic hunger, lacking adequate protein and calorie intake. The global population is steadily increasing and is projected to reach \num{9,7}~billion by 2050, thus increasing the demand for food by 60\%. This puts pressure on the already limited resources of the planet, threatening global food security. Additionally, current food systems contribute to environmental problems such as pollution and climate change. To address these challenges, it is necessary to transform food systems to make them more sustainable and capable of meeting global food demand~\parencite{hosseinkhani_Key_2022}.

Furthermore, excessive meat consumption is associated with a series of negative impacts on the environment, animal welfare, and human health. This is primarily due to greenhouse gas emissions, intensive land use, and disruption of biogeochemical cycles such as phosphorus and nitrogen, which contribute to global warming and biodiversity loss. Additionally, consumption of red and processed meat has been linked to various health risks, including an increased risk of developing diseases~\parencite{michel_multinational_2021}.

For these reasons, reducing current meat consumption could lead to significant benefits for the environment, animal welfare, and human health. Therefore, we can say that algae represent a promising solution, as they are rich in nutrients and bioactive compounds essential for a healthy diet and are increasingly being considered as potential foods or food ingredients.

%
%\cleardoublepage
%\section{Thesis objective}
Hamburgers are one of the most popular and beloved dishes worldwide, but they are often associated with high saturated fat content and low nutritional value. Therefore, there is an interest in developing healthier and more sustainable food alternatives. Algae, in particular, represent a nutrient-rich source and are considered a promising food to improve human health and environmental sustainability. However, incorporating these organisms into hamburgers requires a careful evaluation of preparation methodologies and the effects on the nutritional and sensory profile of the final product. The aim of this research is thus to investigate the effect of adding algae on the production of plant-based hamburgers. This approach aims to provide a valid protein alternative to meat or common plant-based products available on the market, with particular attention to the context of growing environmental concerns related to intensive farming and the search for sustainable solutions for food consumption. Through the use of algae, selected for their nutritional properties and culinary versatility, the aim is to explore the possibility of developing innovative and healthy food products capable of meeting the nutritional needs of the global population. The research aims to evaluate not only the organoleptic and sensory aspects of these burgers but also their nutritional characteristics such as moisture content, dry weight, protein, lipid, and antioxidant content. Furthermore, the intention is to examine the public's response through sensory analysis, with the goal of promoting greater awareness of the potential of these resources in the context of food and environmental sustainability, and to understand specific sensory preferences regarding the corresponding burgers in order to guide and optimize any future projects.

%
%\cleardoublepage
% !TeX encoding = UTF-8

\section{Materials and methods}
In this study, three types of algae commonly used in the food industry were selected in different percentages \tabref{tab:algae_percentages} to create burgers. Ingredients \tabref{tab:ingredients} were added to maintain color and nutritional properties of the algae. Subsequently, an initial sensory test was conducted to evaluate the organoleptic acceptability of the burgers. After this phase, a single formulation was chosen for each type of algae, based on the preferences of the subjects, to proceed with laboratory tests to analyze the nutritional profile of the burgers. The ash, moisture, lipid and protein contents\zxriv{}, were analyzed. Finally, a last sensory analysis was conducted to determine the preferences of the public.

\begin{table}[H]
	\centering
	\begin{tabular}{lc}
	\toprule
		\textbf{Algae}					& \textbf{\% in burgers} \\
	\midrule
		\species{Arthrospira platensis}	& \num{4}\%, \num{6}\% and \num{8}\% \\[\spbtwrows]
		\species{Chlorella vulgaris}	& \num{4}\%, \num{8}\% and \num{12}\% \\[\spbtwrows]
		\species{Palmaria palmata}		& \num{1,5}\% and \num{3}\% \\
	\bottomrule
\end{tabular}
	\caption{Percentage of algae used in burger formulation}
	\label{tab:algae_percentages}
\end{table}

\begin{table}[H]
	\centering
	\begin{tabular}{cccccc}
	\toprule
		\textbf{Legumes} & \textbf{Vegetables} & \textbf{Carbohydrates} & \textbf{Thickeners} & \textbf{Spieces} & \textbf{Fats} \\
	\midrule
		\makecell{Red\\beans}		& Spinach					& \makecell{Oat\\flakes}						& \makecell{Potato\\starch}	& \makecell{Smoked\\garlic}	& \makecell{Coconut\\oil} 	\\[0.75em]
		Lupines		& Carrots		& Quinoa					& Konjac		& Mustard						& 								\\[\spbtwrows]
		Lentils		& Beetrots		& Oat flour					& Guar gum		& Salt							& 								\\[\spbtwrows]
		Peas		& Courgettes	& Cous cous					& Xanthan gum	& Rosemary						& 								\\[\spbtwrows]
		Chickpeas	& Onions		& Potatoes					& 				& Lemon							& 								\\[\spbtwrows]
		 			& 				& 							& 				& Ginger						& 								\\[\spbtwrows]
		 			& 				& 							& 				& Tumeric						& 								\\[\spbtwrows]
		 			& 				& 							& 				& Curry							& 								\\
	\bottomrule
\end{tabular}
	\caption{Ingredients used to find the best burger formulation}
	\label{tab:ingredients}
\end{table}


\subsection{Algae used}
In this project, two microalgae in powders (\species{Arthrospira platensis} and \species{Chlorella vulgaris}) and one macroalga \species{(Palmaria palmata}), in flakes were used. They were sourced from companies producing these organisms on a large industrial scale. The production of these powders and flakes occurs after cultivation through photobioreactors. Once the algae have reached the desired maturity, characterized by the maximum concentration of biomass and the presence of desired nutrients, vitamins, and bioactive compounds, as well as optimal chlorophyll concentration, density, and total biomass, they are harvested and separated from the culture solution. This process can occur through centrifugation, where the algae, being denser than the solution, accumulate at the bottom of the centrifuge tube at high speed, or through filtration, where the culture solution passes through a membrane or filter that retains the algae and other solid particles. Subsequently, the collected biomass undergoes drying (a fundamental step to remove moisture), is skimmed to remove any impurities, and finally transformed into powder before packaging.


\subsubsection{\species{Arthrospira platensis}}
One of the two microalga used for this project is \species{A. platensis} (here after referred to with its common name \species{Spirulina}), a unicellular photosynthetic cyanobacterium considered the ancestor of higher plants \figref{fig:spirulina_views}. It is characterized by its spiral shape (hence the name Spirulina), blue-green coloration, and a cell wall devoid of cellulose, containing complex sugars such as glycogen in the membrane. It thrives in alkaline environments with a pH of \numrange[range-phrase={--}]{10}{12} and can be found in fresh, salt, and brackish waters~\parencite{mohan_spirulina_2014}. It has been used as food for centuries by various populations, primarily due to its nutritional characteristics, such as a protein content equal to \num{70}\% of its dry weight, which is particularly interesting from many perspectives, given that proteins are a fundamental component in the human diet and especially for vegetarian and vegan lifestyles. Spirulina contains a water-soluble blue pigment, phycocyanin, which gives it its characteristic color and possesses various properties. Phycocyanin is composed of two dissimilar protein subunits α and β, with a bilin chromophore attached to the α subunit and two to the β subunit. It is part of the phycobiliproteins, a small group of highly conserved chromoproteins that constitute the phycobilisome, a macromolecular protein complex with the purpose of light harvesting for the photosynthetic apparatus. Phycocyanin is also important for its antioxidant activity (e.g.~β-carotene), useful against free radicals. Additionally, Spirulina is considered a superfood, devoid of toxicity and endowed with beneficial properties against viral attacks, anemia, tumor growth, and malnutrition~\parencite{saranraj_spirulina_2014}. \species{A.~platensis} contains many fatty acids (such as \gls{gla}, omega-3, \gls{dha}, and \gls{epa} and vitamins, especially B12, C, and D2~\parencite{moons_drivers_2017}. Another advantage of this microalga is its easy digestion, precisely due to the absence of cellulose in the cell wall, which makes it an excellent food~\parencite{mohammadi_Spirulina_2022}. Other peculiarities related to potential applications for human health by Spirulina include:
\begin{itemize}
\item liver and kidney protection;
\item improvement of blood quality and prevention of anemia;
\item benefits for diabetes;
\item reduction of blood pressure;
\item removal of heavy metals from the body;
\item radioprotection;
\item prevention of liver and kidney toxicity;
\item antioxidant action;
\item immune protection;
\item relief in allergic reactions~\parencite{mohan_spirulina_2014};
\end{itemize}

\begin{figure}[H]
\centering
% Subfigure 1
	\subcaptionbox*{\label{subfig:spirulina_powder}}%
		{\includegraphics[width=0.55\textwidth]{images/spirulina_powder-mod-low}}%
\hfill
% Subfigure 2
	\subcaptionbox*{\phantomcaption\label{subfig:spirulina_micro}}%
		{\includegraphics[width=0.45\textwidth]{images/spirulina_micro-mod}}%
\caption{Powder form and microscopic view of \species{A.~platensis}}
\label{fig:spirulina_views}
\end{figure}

\subsubsection{\species{Chlorella vulgaris}}
The second microalga used is \species{C.~vulgaris}, a green alga with a diameter ranging \qtyrange{2}{10}{\micro\metre}. It is spherical, subspherical, or ellipsoidal in shape, and lacks flagella \figref{fig:chlorella_views}. It can exist as a single cell or form colonies of up to a maximum of \qty{64}{cells}. It has a single cup-shaped chloroplast, with or without the presence of pyrenoids (which store starch granules). Being a non-mobile microalga, it reproduces through the production of asexual autospores, by dividing the mother cell into 2–32\zxriv{} autospores or daughter cells. The cell wall of the mother cell breaks during the maturation of these autospores, and the debris of the mother cell becomes nourishment for the daughter cells in a process known as autosporulation~\parencite{ru_Chlorella_2020}. The name Chlorella is derived from the Greek word “chloros” (χλωρός), meaning green, and the Latin suffix “ella” referring to its microscopic size. It can be found in freshwater environments and sometimes in brackish environments, and it has also been used in the past as a source of nutrition by many populations and it’s still considered an excellent food today, also useful for medical treatments. As for \species{C.~vulgaris}, proteins are a fundamental component, representing \numrange{42}{58}\%\zxriv{} of the dry weight of the biomass. The most abundant pigment is chlorophyll, found in the thylakoids, and in addition, many carotenoids give it its typical yellow color. These pigments exhibit high antioxidant activity, contributing to the neutralization of free radicals and the inhibition of oxidative damage at the cellular level. Their ability to protect the retina from degenerative processes has also been observed, presumably through mechanisms that promote the stability of photoreceptors and the reduction of oxidative stress in eyes. Studies also indicate a potential role in regulating blood cholesterol levels, which could contribute to the prevention of chronic diseases such as cardiovascular and colon tumors. Additionally, these pigments are believed to enhance the immune system.

In \species{C.~vulgaris}, there are also many minerals that play important roles in human nutrition and health, such as potassium, against hypertension, magnesium, important for maintaining nerve activity and muscle contraction, and zinc (essential for enzymatic activity)~\parencite{safi_Morphology_2014}\zxriv{}. It is rich in iron, useful in preventing anemia and playing a significant role in respiration, energy production, DNA synthesis, and cell proliferation. Furthermore, it contains many water-soluble vitamins such as C and B (B1, B2, B6, B12), fat-soluble vitamins (A, D, E, and K), niacin, folic acid, biotin, and pantothenic acid. Vitamin D2 is crucial for bone health and calcium metabolism, reducing the risk of osteomalacia in adults and rickets in children, this organism is important as a supplement also in vegan and vegetarian diets as for B12.

Other potential applications for human health include:
\begin{itemize}
\item antihypertensive effects
\item anti-hypercholesterolemic and anti-hyperlipidemic effects
\item antidiabetic action
\item hepatoprotective action
\item detoxifying effect~\parencite{bito_Potential_2020}
\end{itemize}

\begin{figure}[H]
\centering
% Subfigure 1
	\subcaptionbox*{\label{subfig:chlorella_powder}}%
		{\includegraphics[width=0.45\textwidth]{images/chlorella_powder-mod}}%
\hfill
% Subfigure 2
	\subcaptionbox*{\label{subfig:chlorella_micro}}%
		{\includegraphics[width=0.45\textwidth]{images/chlorella_micro-mod}}%
\caption{Powder form and microscopic view of \species{C.~vulgaris}}
\label{fig:chlorella_views}
\end{figure}

\subsubsection{\species{Palmaria palmata}}
The macroalga used for this study is \species{P.~palmata}, a name derived from its characteristic hand-like shape, with flattened oblong lobes that extend from the center to the fronds \figref{fig:palmaria_views}. It belongs to the family of red algae and on average grows \qtyrange{10}{20}{\centi\metre} but can reach lengths of up to \qty{50}{\centi\metre} and a width of \qtyrange{3}{8}{\centi\metre}~\parencite{stevant_Concise_2023}. It is mostly found in shallow intertidal or subtidal zones, typical of the Atlantic Ocean, and thrives in cold and turbulent waters~\parencite{mouritsen_human_2013}. It is characterized by a pseudo-perennial life cycle, in which the fronds show a continuous process of growth, reproduction, and aging, while the substrate on which it grows can persist for several years. Like most macroalgae species, \species{P.~palmata} relies on the production and release of spores for reproduction, dissemination, and substrate colonization. To become fertile and induce spore maturation, \species{P.~palmata} has specific environmental requirements regarding temperature and light conditions, which influence the timing of reproduction in situ~\parencite{stevant_Concise_2023}. This macroalga is rich in proteins, comprising approximately 35\% of its dry weight, with variations due to seasonality and geographical origin. Although the lipid content of \species{P.~palmata} is considered relatively low (\numrange{0.3}{3.8}\% of dry weight), in line with many other macroalgae, it should be emphasized that these lipids are readily assimilated by the human body. Moreover, they are rich in \gls{epa}, a fatty acid that plays a crucial role in the prevention of non-communicable diseases. \gls{epa} is a type of omega-3 \gls{pufa} that significantly promotes cardiovascular and neurodegenerative health, in addition to possessing potent antioxidant and anti-inflammatory properties. In algae, \glspl{pufa} are mainly present in structural membrane lipids, such as phospholipids and glycolipids. Recent studies have associated these lipids with various bioactive properties, including anti-tumor, anti-inflammatory, antimicrobial, and antiviral activities. It also contains all essential amino acids and several amino acids like mycosporines, carotenoids such as α-carotene, β-carotene, and lutein, and sterols such as cholesterol and desmosterol. Recent studies have shown that extracts of this macroalga exhibit antioxidant and antiproliferative activity. \glsxtrlong{pufa}, particularly \gls{epa} and \gls{dha}, are known for their cardioprotective and anti-inflammatory properties. These compounds can downregulate the production of pro-inflammatory cytokines associated with metabolic syndrome, a set of conditions that increase the risk of heart disease, type 2 diabetes, and other related pathologies~\parencite{lopes_New_2019}.

\begin{figure}[H]
\centering
% Subfigure 1
	\subcaptionbox*{\label{subfig:palmaria_flakes}}%
		{\includegraphics[width=0.5\textwidth]{images/palmata_flakes-mod-low}}%
\hfill
% Subfigure 2
	\subcaptionbox*{\label{subfig:palmaria_micro}}%
		{\includegraphics[width=0.5\textwidth]{images/palmata_micro-2-rotate}}%
\caption{Flakes and microscopic view of \species{P.~palmata}}%
\label{fig:palmaria_views}
\end{figure}


\subsection{Development of the algae burgers}
During the hamburger creation phase, careful selection of ingredients was made to maintain the color and nutritional properties of the algae. For example, for burgers based on \species{A.~platensis}, green ingredients such as zucchini, spinach, peas, and broccoli were used, for those based on \species{C.~vulgaris}, carrots, turmeric, curry, and onions were added, and for burgers containing \species{P.~palmata}, beetroots, red beans, and red onions were used.

To ensure the completeness and nutritional balance of the meals, complementary ingredients such as oat flour, xanthan gum, konjac, guar gum, coconut oil, quinoa, couscous, and potatoes were integrated. These were also added to improve the texture of the burgers and provide additional nutritional benefits. This targeted selection of ingredients allowed for the creation of healthy, nutritious burgers with a balanced nutritional profile. Furthermore, the amount of algal biomass to be added to the final product was determined. During this process, various tests were conducted using different formulations for each algal spieces, varying the percentages. The algal percentage and the respective ingredients chosen for each burger are listed in the Tables~\ref{tab:formulation_spices-spirulina},~\ref{tab:formulation_spices-chlorella}~and~\ref{tab:formulation_spices-palmaria} and these were the formulations used for subsequent nutritional analyses.

\begin{table}[H]
\footnotesize
\centering
	\begin{subcaptionblock}{0.68\textwidth}
	\centering
		\begin{tabular}{lclclc}
	\toprule
	\belowrulesepcolor{colspir}
	\rowcolor{colspir}
		\multicolumn{2}{c}{\textbf{\species{A.~platensis} 4\%}} & \multicolumn{2}{c}{\textbf{\species{A.~platensis} 6\%}} & \multicolumn{2}{c}{\textbf{\species{A.~platensis} 8\%}} \\[\spheader]
	\rowcolor{colspir}
		\textbf{Ingredient} & \textbf{grams} & \textbf{Ingredient} & \textbf{grams} & \textbf{Ingredient} & \textbf{grams} \\
	\aboverulesepcolor{colspir}
	\midrule
		Peas									& \num{51}	& Peas									& \num{49}	& Peas									& \num{47} \\[\spbtwrows]
		Oat flour								& \num{10}	& Oat flour								& \num{10}	& Oat flour								& \num{10} \\[\spbtwrows]
		Oat flakes								& \num{8}	& Oat flakes							& \num{8}	& Oat flakes							& \num{8} \\[\spbtwrows]
		Courgettes								& \num{8}	& Courgettes							& \num{8}	& Courgettes							& \num{8} \\[\spbtwrows]
		Onion									& \num{7}	& Onion									& \num{7}	& Onion									& \num{7} \\[\spbtwrows]
		\makecell{Xanthan\\[\spbtwlines]gum}	& \num{2}	& \makecell{Xanthan\\[\spbtwlines]gum}	& \num{2}	& \makecell{Xanthan\\[\spbtwlines]gum}	& \num{2} \\[\spbtwrows]
		\makecell{Coconut\\[\spbtwlines]oil}	& \num{8}	& \makecell{Coconut\\[\spbtwlines]oil}	& \num{8}	& \makecell{Coconut\\[\spbtwlines]oil}	& \num{8} \\[\spbtwrows]
		Spices									& \num{2}	& Spices								& \num{2}	& Spices								& \num{2} \\[\spbtwrows]
		Microalga								& \num{4}	& Microalga								& \num{6}	& Microalga								& \num{8} \\[\spbtwrows]
		Tot										& \num{100}	& Tot									& \num{100}	& Tot									& \num{100} \\
	\bottomrule


%	\toprule
%		\multicolumn{2}{c}{\textbf{\species{A.~platensis} 4\%}} & \multicolumn{2}{c}{\textbf{\species{A.~platensis} 6\%}} & \multicolumn{2}{c}{\textbf{\species{A.~platensis} 8\%}} \\
%		\textbf{What?} & \textbf{grams} & \textbf{What?} & \textbf{grams} & \textbf{What?} & \textbf{grams} \\
%	\midrule
%		Peas		& \qty{51}{\gram}	& Peas			& \qty{49}{\gram}	& Peas			& \qty{47}{\gram} \\
%		Oat flour	& \qty{10}{\gram}	& Oat flour		& \qty{10}{\gram}	& Oat flour		& \qty{10}{\gram} \\
%		Oat flakes	& \qty{8}{\gram}	& Oat flakes	& \qty{8}{\gram}	& Oat flakes	& \qty{8}{\gram} \\
%		Courgettes	& \qty{8}{\gram}	& Courgettes	& \qty{8}{\gram}	& Courgettes	& \qty{8}{\gram} \\
%		Onion		& \qty{7}{\gram}	& Onion			& \qty{7}{\gram}	& Onion			& \qty{7}{\gram} \\
%		Xanthan gum	& \qty{2}{\gram}	& Xanthan gum	& \qty{2}{\gram}	& Xanthan gum	& \qty{2}{\gram} \\
%		Coconut oil	& \qty{8}{\gram}	& Coconut oil	& \qty{8}{\gram}	& Coconut oil	& \qty{8}{\gram} \\
%		Spices		& \qty{2}{\gram}	& Spices		& \qty{2}{\gram}	& Spices		& \qty{2}{\gram} \\
%		Microalga	& \qty{4}{\gram}	& Microalga		& \qty{6}{\gram}	& Microalga		& \qty{8}{\gram} \\
%		Tot			& \qty{100}{\gram}	& Tot			& \qty{100}{\gram}	& Tot			& \qty{100}{\gram} \\
%	\bottomrule
\end{tabular}%
	\end{subcaptionblock}%
\hspace*{\hbtwsfig}%
	\begin{subcaptionblock}[][18.91em][c]{0.235\textwidth}
	\centering
		\begin{tabular}{lc}
	\toprule
	\rowcolor{colspir}
		\textbf{\shortstack{Spice}} & \textbf{\Centerstack{Amount\\(grams)}} \\
	\midrule
		Salt		& \num{0,4} \\
		Rosemary	& \num{0,4} \\
		Garlic		& \num{0,3} \\
		Mustard		& \num{0,5} \\
		Lemon		& \num{0,4} \\
		Tot			& \num{2} \\
	\bottomrule
\end{tabular}%
	\end{subcaptionblock}%
\caption{Formulation of \species{A.~platensis} burgers for each percentage of algae and corresponding spices used}
\label{tab:formulation_spices-spirulina}
\end{table}

\begin{table}[H]
\footnotesize
\centering
	\begin{subcaptionblock}{0.68\textwidth}
	\centering
		\begin{tabular}{cccccc}
	\toprule
	\rowcolor{colchlo}
		\multicolumn{2}{c}{\textbf{\species{C.~vulgaris} 4\%}} & \multicolumn{2}{c}{\textbf{\species{C.~vulgaris} 8\%}} & \multicolumn{2}{c}{\textbf{\species{C.~vulgaris} 12\%}} \\[\spheader]
	\rowcolor{colchlo}
		\textbf{Ingredient} & \textbf{grams} & \textbf{Ingredient} & \textbf{grams} & \textbf{Ingredient} & \textbf{grams} \\
	\midrule
		Chickpeas								& \num{51}	& Chickpeas								& \num{47}	& Chickpeas								& \num{43} \\[\spbtwrows]
		Oat flakes								& \num{20}	& Oatflakes								& \num{20}	& Oatflakes								& \num{20} \\[\spbtwrows]
		Carrots									& \num{15}	& Carrots								& \num{15}	& Carrots								& \num{15} \\[\spbtwrows]
		\makecell{Xanthan\\[\spbtwlines]gum}	& \num{0}	& \makecell{Xanthan\\[\spbtwlines]gum}	& \num{0}	& \makecell{Xanthan\\[\spbtwlines]gum}	& \num{0} \\[\spbtwrows]
		\makecell{Coconut\\[\spbtwlines]oil}	& \num{8}	& \makecell{Coconut\\[\spbtwlines]oil}	& \num{8}	& \makecell{Coconut\\[\spbtwlines]oil}	& \num{8} \\[\spbtwrows]
		Spices									& \num{2}	& Spices								& \num{2}	& Spices								& \num{2} \\[\spbtwrows]
		Microalga								& \num{4}	& Microalga								& \num{8}	& Microalga								& \num{12} \\[\spbtwrows]
		Tot										& \num{100}	& Tot									& \num{100}	& Tot									& \num{100} \\[\spbtwrows]
	\bottomrule
\end{tabular}%
	\end{subcaptionblock}%
\hspace*{\hbtwsfig}%
	\begin{subcaptionblock}[][16.5em][c]{0.235\textwidth}
	\centering
		\begin{tabular}{lc}
	\toprule
		\textbf{\shortstack{Spice}} & \textbf{\Centerstack{Amount\\(grams)}} \\
	\midrule
		Salt		& \num{0,2} \\
		Rosemary	& \num{0,4} \\
		Garlic		& \num{0,4} \\
		Tumeric		& \num{0,15} \\
		Curry		& \num{0,15} \\
		Mustard		& \num{0,3} \\
		Lemon		& \num{0,4} \\
		Tot			& \num{2} \\
	\bottomrule
\end{tabular}%
	\end{subcaptionblock}%
\caption{Formulation of \species{C.~vulgaris} burgers for each percentage of algae and corresponding spices used}
\label{tab:formulation_spices-chlorella}
\end{table}

\begin{table}[H]
\footnotesize
\centering
	\begin{subcaptionblock}{0.5\textwidth}
	\centering
		\begin{tabular}{cccc}
	\toprule
		\multicolumn{2}{c}{\textbf{\species{P.~palmata} 1.5\%}} & \multicolumn{2}{c}{\textbf{\species{P.~palmata} 3\%}} \\[\spheader]
		\textbf{Ingredient} & \textbf{grams} & \textbf{Ingredient} & \textbf{grams} \\
	\midrule
		Red beans	& \num{57}		& Red beans		& \num{54} \\
		Cous cous	& \num{7}		& Cous cous		& \num{7} \\
		Oat flour	& \num{8}		& Oat flour		& \num{8} \\
		Beetroots	& \num{8}		& Beetroots		& \num{8} \\
		Red onion	& \num{7}		& Red onion		& \num{7,5} \\
		Xanthan gum	& \num{0,5}		& Xanthan gum	& \num{0,5} \\
		Coconut oil	& \num{9}		& Coconut oil	& \num{9} \\
		Spices		& \num{2}		& Spices		& \num{2} \\
		Microalga	& \num{1,5}		& Microalga		& \num{3} \\
		Tot			& \num{100}		& Tot			& \num{100} \\
	\bottomrule
\end{tabular}%
	\end{subcaptionblock}%
\hspace*{\hbtwsfig}%
	\begin{subcaptionblock}[][16em][c]{0.235\textwidth}
	\centering
		\begin{tabular}{lc}
	\toprule
	\rowcolor{colpalm}
		\textbf{\shortstack{Spice}} & \textbf{\Centerstack{Amount\\(grams)}} \\
	\midrule
		Salt		& \num{0,6} \\
		Rosemary	& \num{0,2} \\
		Garlic		& \num{0,4} \\
		Mustard		& \num{0,2} \\
		Ginger		& \num{0,2} \\
		Lemon		& \num{0,4} \\
		Tot			& \num{2} \\
	\bottomrule
\end{tabular}%
	\end{subcaptionblock}%
\caption{Formulation of \species{P.~palmata} burgers for each percentage of algae and corresponding spices used}
\label{tab:formulation_spices-palmaria}
\end{table}

\subsection{Nutritional analysis}
All analyses were repeated twice for each percentage of algae present in each burger formulation. In addition, a control sample for each type of burger was included, without addition of algae, in order to have a point of comparison for the conducted analyses. This approach allowed for the evaluation of the impact of different algal concentrations on the characteristics of the burgers, while the control sample provided a reference for assessing the accuracy, precision, and variability of the results obtained.


\subsubsection{Moisturized content}
To determine the moisture content of algae burgers at various percentages and control burgers, the~\cite{aoac_2000}\zxriv{} method was relied upon. First, aluminum plates were prepared and dried in an oven at \qty{105}{\degreeCelsius} for \qty{3}{hours} to remove any moisture residue. After this process, they were transferred to a desiccator for \qty{30}{minutes} to cool \figref{fig:desiccator}. This cooling phase is important to avoid any weight variations due to temperature conditions.

\begin{figure}[H]
	\centering
	\includegraphics[width=0.45\textwidth]{images/desiccator-mod}
	\caption{Desiccator}
	\label{fig:desiccator}
\end{figure}

Next, the dry plates were weighed to obtain their initial weight. For the analysis procedure, \num{22} samples of \qty{2}{\gram} each were prepared as follows: \num{2} samples of \species{A.~platensis} burgers for each algae percentage (4\%, 6\%, 8\%), \num{2} samples of \species{C.~vulgaris} burgers for each algae percentage (4\%, 8\%, 12\%), \num{2} samples of \species{P.~palmata} burgers for each algae percentage (1.5\%, 3\%), and 2 samples of control burgers without algae for each formulation (i.e., 2 with ingredient from \species{A.~platensis} burger, 2 with ingredient from \species{C.~vulgaris} burgers, 2 with ingredient from \species{P.~palmata} burgers). These samples were then cooked, homogenized, and spread onto the aluminum plates. The plates with the samples were then placed back in the oven at \qty{105}{\degreeCelsius} for \qty{3}{hours} to completely dry the sample \figref{fig:samples_in_oven_1}.

\begin{figure}[H]
	\centering
	\includegraphics[width=0.5\textwidth]{images/samples_in_oven_1-mod}
	\caption{Samples in the oven}
	\label{fig:samples_in_oven_1}
\end{figure}

All were then transferred to the desiccator for \qty{30}{minutes} for cooling. Subsequently, each plate containing the dried sample was weighed. The calculation to determine the moisture of each sample was performed using the following formula:\zxriv{}
\[
	\text{Moisture \%} = \frac{(W_1 - W_2) * 100}{W_1}
\]
where $ W_1 $ represents the weight of the sample before drying, and $ W_2 $ represents the weight of the sample after drying.


\subsubsection{Ash content}
To determine the ash content of algae burgers at various percentages and control burgers, the AOAC~2000 method (AOAC, 2000\zxriv{}) was relied upon. Firstly, an aluminum support and specially designed cups to contain the samples were created. These were then placed in an oven heated to a temperature of \qty{550}{\degreeCelsius} for an entire night to eliminate any impurities present in the samples \figref{subfig:samples_in_oven_2}. This heating phase is essential to ensure that the samples are completely dried and free from any volatile material that could influence the analysis results. The samples resulting from the moisture content analysis were recovered and used for these analyses and then placed in the oven heated to \qty{550}{\degreeCelsius} for an entire night and then dried again before being weighed again \figref{subfig:residue_ash}.

\begin{figure}[H]
\centering
% Subfigure 1
	\subcaptionbox%
	{Samples in the oven\label{subfig:samples_in_oven_2}}%
		{\includegraphics[width=0.45\textwidth]{images/samples_in_oven_2-cut-mod}}%
\hspace*{0.1\textwidth}%
% Subfigure 2
	\subcaptionbox%
		{Residue of ash content after combustion\label{subfig:residue_ash}}%
		{\includegraphics[width=0.45\textwidth]{images/residue_ash-mod}}%
\caption{}
%\label{fig:}
\end{figure}

This process allows determining the weight of the residual ashes after the complete combustion of the organic materials present in the samples. The calculation of the ash content is then performed using the following formula:
\[
\text{Ash \%} = \frac{W_\text{A} * 100}{W_2}
\]
where $ W_\text{A} $ represents the weight in grams of the residual ashes after combustion and $ W_2 $ the initial weight of the analyzed sample, and therefore in our case, the weight resulting from the drying at \qty{105}{\degreeCelsius} for the moisture content.


\subsubsection{Lipid content}
%For the determination of lipid content, the modified Bligh and Dyer gravimetric method~\parencite{bligh_RAPID_1959} was used. First, microtubes were prepared for lipid weighing by washing and drying them at \qty{60}{\degreeCelsius} for at least \qty{3}{hours}, followed by placing them in a desiccator for another \qty{3}{hours} before proceeding with weighing on a precision balance. Once the weight of each microtube was obtained, the actual methodology can be carried out.
For the determination of lipid content, the modified Bligh and Dyer gravimetric method \parencite{bligh_RAPID_1959} was used. First, microtubes were prepared for lipid weighing by washing and drying them at \qty{60}{\degreeCelsius} for at least \qty{3}{hours}, followed by placing them in a desiccator for another \qty{3}{hours} before proceeding with weighing on a precision balance. Once the weight of each microtube was obtained, the actual methodology can be carried out.

\zxriv{}\qty{30}{\milli\gram} of lyophilized biomass were accurately weighed and transferred into \qty{2}{\milli\metre} microtubes. This step requires precision, as even small variations in the sample quantity can affect the analysis results.

Next, \qty{0.5}{\milli\litre} of glass beads were introduced into the microtubes for extraction. Then, \qty{0.8}{\milli\litre} of distilled water were added, and the mixture is allowed to react for at least \qty{15}{minutes}. This resting period is crucial as it allows the samples to soften, making lipid extraction more efficient. Working under a fume hood to ensure safety and reduce the risk of contamination, the following steps were followed sequentially:
\begin{itemize}
\item adding \qty{0,5}{\milli\litre} of methanol and \qty{0,2}{\milli\litre} of chloroform.
\item homogenizing the mixture using a vortex and then in a beadmill \figref{fig:beadmill} at \qty{30}{\hertz} for \qty{3}{minutes}. This ensures uniform distribution of solvents and samples.
\item once homogenized, transferring everything (including glass beads and biomass) into a glass centrifuge tube and cleaning the used Eppendorf with a mixture of \qty{1,5}{\milli\litre} of methanol and \qty{0,75}{\milli\litre} of chloroform, then adding this solution to the tube.
\item mixing again, this time using only the vortex.
\item adding additional chloroform, \qty{1}{\milli\litre}.
\item homogenizing with a vortex.
\item adding distilled water, \qty{1}{\milli\litre}.
\item homogenizing the samples twice with a vortex.
\end{itemize}

\begin{figure}[H]
	\centering
	\includegraphics[width=0.6\textwidth]{images/beadmill-mod}
	\caption{Beadmill}
	\label{fig:beadmill}
\end{figure}

After this extraction phase, the samples were centrifuged at 2000~×~\textsl{g}
%\footnote{\zariv{} \href{https://www.fishersci.it/it/it/scientific-products/centrifuge-guide/centrifuge-applications-tools/rpm-rcf-calculator.html}{1} \href{https://www.sigmaaldrich.com/IT/it/support/calculators-and-apps/g-force-calculator}{2} \href{https://tomy.amuzainc.com/blog/optimizing-centrifuge-speed/}{3}}
for \qty{10}{minutes} at room temperature. This process separates the lipids from the solvent-sample mixture. Then, the chloroform containing the extracted lipids is pipetted into new tubes using a Pasteur pipette. The appearance of the extraction should be translucent \figref{fig:extraction_example}.

\begin{figure}[H]
	\centering
	\includegraphics[width=0.5\textwidth]{images/extraction_example-mod}
	\caption{Example of \species{C.~vulgaris} extraction}
	\label{fig:extraction_example}
\end{figure}

Otherwise, centrifugation is repeated to ensure complete separation. Next, a known amount of chloroform, usually between \qtyrange{0.5}{1}{\milli\litre}, is pipetted into the microtubes for subsequent weighing. The microtubes are then placed in a dry bath at \qty{60}{\degreeCelsius} \figref{fig:dry_bath_fume_hood} to allow complete evaporation of chloroform, all of course under a fume hood.

\begin{figure}[H]
	\centering
	\includegraphics[width=0.5\textwidth]{images/dry_bath_fume_hood-mod}
	\caption{Dry bath at \qty{60}{\degreeCelsius} in a fume hood}
	\label{fig:dry_bath_fume_hood}
\end{figure}

Following this phase, the tubes were transferred to the desiccator for at least \qty{3}{hours} to ensure they are completely dry before weighing on a precision balance. It is important to use tweezers to handle the tubes to avoid contamination and ensure accurate results. This detailed and rigorous procedure allows us to obtain reliable data on the lipid content of our samples, which will be obtained following this calculation:\zxriv{}
\[
	\text{Lipid \%}
	=
	\frac
		{\frac{(W_f - W_t) * Ch}{E_{Ch}}}
		{Ws}
		* 100
\]
where $ Wf $ is the final weight, $ Wt $ is the weight of the glass tubes, $ Ch $ is the total chloroform (\qty{2}{\milli\litre}), $ ECh $ is the evaporated chloroform, and $ Ws $ is the weight of the samples (\qty{30}{\milli\gram}).

\subsubsection{Protein content}
For the final type of analysis, we focused on the elemental analysis of the dried samples to determine the carbon (C), hydrogen (H), and nitrogen (N) content, and subsequently estimate the protein content.
Burger samples were dried by lyophilization to remove moisture and ensure the accuracy of the analysis. The dry samples were ground into a fine powder using a ball mill (RETSCH MM 300). This step is crucial to ensure adequate sample homogeneity, reducing variations in analytical results. Approximately \qtyrange{2}{3}{\milli\gram} of the obtained powder were taken and subjected to a CHN elemental analyzer (Elementar Vario EL III) to quantify the total carbon, hydrogen, and nitrogen content. The analyzer uses a high-temperature combustion process to decompose the sample, allowing the measurement of the produced gases (\ch{CO2}, \ch{H2O}, \ch{N2}) to determine the concentration of each element.
The sample was combusted in an oxygen-rich environment at around \qty{1000}{\degreeCelsius}. During combustion, the nitrogen present in the sample is converted to \ch{N2} gas. The produced gases passed through various filters and chemical reagents to remove potential interferents such as \ch{CO2} and \ch{H2O}, ensuring that only pure nitrogen reached the detector. The \ch{N2} gas was measured using a thermal conductivity detector. The detected nitrogen concentration is directly proportional to the protein content in the sample.
The protein content was estimated using the Dumas method, as specified in
\zxriv{}ISO 16634 and AOAC 990.03 standards (Horwitz \& Latimer, 2006)\zxriv{}
. This method involves multiplying the total nitrogen content by a conversion factor of 6.25, according to FAO guidelines~\parencite{fao_2003}:
\[
    \text{Protein (\%)} = \text{Total nitrogen} * 6.25
\]

This factor assumes that most proteins contain about \num{16}\% nitrogen \mbox{(1/0.16 = 6.25)}. Therefore, the factor 6.25 is an estimate based on the average nitrogen content in proteins but can vary depending on the specific protein composition in the analyzed samples.

\subsection{Statistical Analysis}
\subsubsection{Univariate Analysis}
\Gls{anova} was used to test for differences in all the considered variables (i.e., ash, moisture, lipid, and proteins contents) among different formulations separately for each algal species. The design for the consisted of the single factor ‘Composition\%’, fixed, with \numrange[range-phrase={--}]{3}{4} levels corresponding to the control, and the different \%\zxriv{} of algae contained in the burger formulation. Prior to analysis, the assumption of normality of the response variables was tested with the Shapiro–Wilk test. The Cochran’s C-test was used to check for the homogeneity of variances ~\parencite{underwood_Experiments_1996} and data were $ \log(x + 1) $ transformed to stabilize variance if required. Post-hoc pairwise t-test were performed to examine specific differences in the investigated variables among compositions.

\subsubsection{Multivariate Analysis}
Distance-based permutational multivariate analysis of variance (PERMANOVA;~\cite{anderson_new_2001}) was used to test for differences in the overall set of nutritional factors among burgers containing the different \% of the algae. Also in this case, the analysis included the single factor ‘Treatment’, with eight levels (corresponding to the \num{3} compositions for \species{Spirulina} and \species{C.~vulgaris}, and the two compositions of \species{P.~palmata}). The analysis was based on Euclidean distance and tests were done using \num{999} permutations. Post-hoc pairwise t-test were performed among levels. However, due to the limited number of samples P-values were obtained using a Monte Carlo random sample from the asymptotic permutation distribution~\parencite{anderson_Generalized_2003}. A canonical \gls{cap}~\parencite{anderson_CANONICAL_2003} was performed for factor ‘Treatment’ to depict patterns of variation among different burger compositions. Correlations of individual variables with the two canonical axes were represented as lines in a projection biplot.

\subsection{Sensory Analysis}
\subsubsection{Preparation of Initial Formulations}
Before the nutritional analyses, preliminary sensory analyses were conducted to identify the most preferred burger formulations among consumers. Thirteen \species{Spirulina} burgers, nine \species{C.~vulgaris} burgers, and eight \species{P.~palmata} burgers with different ingredient percentages and mixes were tasted. From this initial assessment, the two most appreciated combinations were identified, and a subsequent sensory analysis was performed to select a single formulation for each algae type, which would then be subjected to nutritional analyses and further final sensory evaluations.

\subsubsection{First Sensory Analysis}
A sample of \num{28} subjects was given a questionnaire concerning six different burgers, divided as follows: two formulations with different ingredients for each alga (\num{4}\% \species{Spirulina}, \num{4}\% \species{C.~vulgaris} and \num{1.5}\% \species{P.~palmata}). The questions addressed taste, smell, color, and texture of each algae pair, and a final preference judgment between the two.
The formulations selected from this first sensory analysis, as reported in Tables 3, 4, and 5, were then subjected to the aforementioned nutritional analyses.

\subsubsection{Last Sensory Analysis}
Four \qty{100}{\gram} burgers were prepared for each algae percentage, totaling \num{32} burgers. These were cooked, cut into small pieces, and divided into tasting samples on designated plates \figref{fig:last_sensory_analysis}. The \species{P.~palmata} burgers were coded as 1A for \num{1.5}\% and 1B for \num{3}\%. The \species{Spirulina} burgers were coded as 2A, 2B, 2C for \num{4}\%, \num{6}\%, and \num{8}\%, respectively. Finally, the \species{C.~vulgaris} burgers were coded as 3A for \num{4}\%, 3B for \num{8}\%, and 3C for \num{12}\%.
Napkins and glasses of water were provided, with the requirement to rinse the mouth after each tasting. The analysis was conducted with \num{41} participants, each seated at desks with side panels to isolate tasters and prevent any form of communication and judgment interference. These desks featured a shaded window through which the various burgers were passed for tasting, avoiding any bias or prejudice in the responses. Each subject was given a QR~code to access the questionnaire, which included general questions about gender, age, origin, dietary habits, and specific questions about the burgers. Initially, an image of the burger was shown for an aesthetic judgment, followed by tasting questions regarding smell, texture, and taste. At the end of each section, participants were asked to indicate their preferred burger code and whether they would consider buying it. At the end of the questionnaire, they were asked to indicate their overall favorite burger among all the tasted types, adding any comments or final suggestions if they had any.

\begin{figure}[H]
	\centering
	\includegraphics[width=1\textwidth]{images/last_sensory_analysis-mod}
	\caption{Last sensory analysis}
	\label{fig:last_sensory_analysis}
\end{figure}



%\cleardoublepage
%% !TeX encoding = UTF-8


%
%
%\cleardoublepage
%\input{.tex}
%
%
%
\cleardoublepage
%\emergencystretch 0.95em
\printbibliography[heading=bibintoc]

\end{document}

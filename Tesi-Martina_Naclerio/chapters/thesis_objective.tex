% !TeX encoding = UTF-8

\section{Thesis objective}
Hamburgers are one of the most popular and beloved dishes worldwide, but they are often associated with high saturated fat content and low nutritional value. Therefore, there is an interest in developing healthier and more sustainable food alternatives. Algae, in particular, represent a nutrient-rich source and are considered a promising food to improve human health and environmental sustainability. However, incorporating these organisms into hamburgers requires a careful evaluation of preparation methodologies and the effects on the nutritional and sensory profile of the final product. The aim of this research is thus to investigate the effect of adding algae on the production of plant-based hamburgers. This approach aims to provide a valid protein alternative to meat or common plant-based products available on the market, with particular attention to the context of growing environmental concerns related to intensive farming and the search for sustainable solutions for food consumption. Through the use of algae, selected for their nutritional properties and culinary versatility, the aim is to explore the possibility of developing innovative and healthy food products capable of meeting the nutritional needs of the global population. The research aims to evaluate not only the organoleptic and sensory aspects of these burgers but also their nutritional characteristics such as moisture content, dry weight, protein, lipid, and antioxidant content. Furthermore, the intention is to examine the public's response through sensory analysis, with the goal of promoting greater awareness of the potential of these resources in the context of food and environmental sustainability, and to understand specific sensory preferences regarding the corresponding burgers in order to guide and optimize any future projects.

% !TeX encoding = UTF-8

\section{Materials and methods}
In this study, three types of algae commonly used in the food industry were selected in different percentages \tabref{tab:algae_percentages} to create burgers. Ingredients \tabref{tab:ingredients} were added to maintain color and nutritional properties of the algae. Subsequently, an initial sensory test was conducted to evaluate the organoleptic acceptability of the burgers. After this phase, a single formulation was chosen for each type of algae, based on the preferences of the subjects, to proceed with laboratory tests to analyze the nutritional profile of the burgers. The ash, moisture, lipid and protein contents\zxriv{}, were analyzed. Finally, a last sensory analysis was conducted to determine the preferences of the public.

\begin{table}[H]
	\centering
	\begin{tabular}{lc}
	\toprule
		\textbf{Algae}					& \textbf{\% in burgers} \\
	\midrule
		\species{Arthrospira platensis}	& \num{4}\%, \num{6}\% and \num{8}\% \\[\spbtwrows]
		\species{Chlorella vulgaris}	& \num{4}\%, \num{8}\% and \num{12}\% \\[\spbtwrows]
		\species{Palmaria palmata}		& \num{1,5}\% and \num{3}\% \\
	\bottomrule
\end{tabular}
	\caption{Percentage of algae used in burger formulation}
	\label{tab:algae_percentages}
\end{table}

\begin{table}[H]
	\centering
	\begin{tabular}{cccccc}
	\toprule
		\textbf{Legumes} & \textbf{Vegetables} & \textbf{Carbohydrates} & \textbf{Thickeners} & \textbf{Spieces} & \textbf{Fats} \\
	\midrule
		\makecell{Red\\beans}		& Spinach					& \makecell{Oat\\flakes}						& \makecell{Potato\\starch}	& \makecell{Smoked\\garlic}	& \makecell{Coconut\\oil} 	\\[0.75em]
		Lupines		& Carrots		& Quinoa					& Konjac		& Mustard						& 								\\[\spbtwrows]
		Lentils		& Beetrots		& Oat flour					& Guar gum		& Salt							& 								\\[\spbtwrows]
		Peas		& Courgettes	& Cous cous					& Xanthan gum	& Rosemary						& 								\\[\spbtwrows]
		Chickpeas	& Onions		& Potatoes					& 				& Lemon							& 								\\[\spbtwrows]
		 			& 				& 							& 				& Ginger						& 								\\[\spbtwrows]
		 			& 				& 							& 				& Tumeric						& 								\\[\spbtwrows]
		 			& 				& 							& 				& Curry							& 								\\
	\bottomrule
\end{tabular}
	\caption{Ingredients used to find the best burger formulation}
	\label{tab:ingredients}
\end{table}


\subsection{Algae used}
In this project, two microalgae in powders (\species{Arthrospira platensis} and \species{Chlorella vulgaris}) and one macroalga \species{(Palmaria palmata}), in flakes were used. They were sourced from companies producing these organisms on a large industrial scale. The production of these powders and flakes occurs after cultivation through photobioreactors. Once the algae have reached the desired maturity, characterized by the maximum concentration of biomass and the presence of desired nutrients, vitamins, and bioactive compounds, as well as optimal chlorophyll concentration, density, and total biomass, they are harvested and separated from the culture solution. This process can occur through centrifugation, where the algae, being denser than the solution, accumulate at the bottom of the centrifuge tube at high speed, or through filtration, where the culture solution passes through a membrane or filter that retains the algae and other solid particles. Subsequently, the collected biomass undergoes drying (a fundamental step to remove moisture), is skimmed to remove any impurities, and finally transformed into powder before packaging.


\subsubsection{\species{Arthrospira platensis}}
One of the two microalga used for this project is \species{A. platensis} (here after referred to with its common name \species{Spirulina}), a unicellular photosynthetic cyanobacterium considered the ancestor of higher plants \figref{fig:spirulina_views}. It is characterized by its spiral shape (hence the name Spirulina), blue-green coloration, and a cell wall devoid of cellulose, containing complex sugars such as glycogen in the membrane. It thrives in alkaline environments with a pH of \numrange[range-phrase={--}]{10}{12} and can be found in fresh, salt, and brackish waters~\parencite{mohan_spirulina_2014}. It has been used as food for centuries by various populations, primarily due to its nutritional characteristics, such as a protein content equal to \num{70}\% of its dry weight, which is particularly interesting from many perspectives, given that proteins are a fundamental component in the human diet and especially for vegetarian and vegan lifestyles. Spirulina contains a water-soluble blue pigment, phycocyanin, which gives it its characteristic color and possesses various properties. Phycocyanin is composed of two dissimilar protein subunits α and β, with a bilin chromophore attached to the α subunit and two to the β subunit. It is part of the phycobiliproteins, a small group of highly conserved chromoproteins that constitute the phycobilisome, a macromolecular protein complex with the purpose of light harvesting for the photosynthetic apparatus. Phycocyanin is also important for its antioxidant activity (e.g.~β-carotene), useful against free radicals. Additionally, Spirulina is considered a superfood, devoid of toxicity and endowed with beneficial properties against viral attacks, anemia, tumor growth, and malnutrition~\parencite{saranraj_spirulina_2014}. \species{A.~platensis} contains many fatty acids (such as \gls{gla}, omega-3, \gls{dha}, and \gls{epa} and vitamins, especially B12, C, and D2~\parencite{moons_drivers_2017}. Another advantage of this microalga is its easy digestion, precisely due to the absence of cellulose in the cell wall, which makes it an excellent food~\parencite{mohammadi_Spirulina_2022}. Other peculiarities related to potential applications for human health by Spirulina include:
\begin{itemize}
\item liver and kidney protection;
\item improvement of blood quality and prevention of anemia;
\item benefits for diabetes;
\item reduction of blood pressure;
\item removal of heavy metals from the body;
\item radioprotection;
\item prevention of liver and kidney toxicity;
\item antioxidant action;
\item immune protection;
\item relief in allergic reactions~\parencite{mohan_spirulina_2014};
\end{itemize}

\begin{figure}[H]
\centering
% Subfigure 1
	\subcaptionbox*{\label{subfig:spirulina_powder}}%
		{\includegraphics[width=0.55\textwidth]{images/spirulina_powder-mod-low}}%
\hfill
% Subfigure 2
	\subcaptionbox*{\phantomcaption\label{subfig:spirulina_micro}}%
		{\includegraphics[width=0.45\textwidth]{images/spirulina_micro-mod}}%
\caption{Powder form and microscopic view of \species{A.~platensis}}
\label{fig:spirulina_views}
\end{figure}

\subsubsection{\species{Chlorella vulgaris}}
The second microalga used is \species{C.~vulgaris}, a green alga with a diameter ranging \qtyrange{2}{10}{\micro\metre}. It is spherical, subspherical, or ellipsoidal in shape, and lacks flagella \figref{fig:chlorella_views}. It can exist as a single cell or form colonies of up to a maximum of \qty{64}{cells}. It has a single cup-shaped chloroplast, with or without the presence of pyrenoids (which store starch granules). Being a non-mobile microalga, it reproduces through the production of asexual autospores, by dividing the mother cell into 2–32\zxriv{} autospores or daughter cells. The cell wall of the mother cell breaks during the maturation of these autospores, and the debris of the mother cell becomes nourishment for the daughter cells in a process known as autosporulation~\parencite{ru_Chlorella_2020}. The name Chlorella is derived from the Greek word “chloros” (χλωρός), meaning green, and the Latin suffix “ella” referring to its microscopic size. It can be found in freshwater environments and sometimes in brackish environments, and it has also been used in the past as a source of nutrition by many populations and it’s still considered an excellent food today, also useful for medical treatments. As for \species{C.~vulgaris}, proteins are a fundamental component, representing \numrange{42}{58}\%\zxriv{} of the dry weight of the biomass. The most abundant pigment is chlorophyll, found in the thylakoids, and in addition, many carotenoids give it its typical yellow color. These pigments exhibit high antioxidant activity, contributing to the neutralization of free radicals and the inhibition of oxidative damage at the cellular level. Their ability to protect the retina from degenerative processes has also been observed, presumably through mechanisms that promote the stability of photoreceptors and the reduction of oxidative stress in eyes. Studies also indicate a potential role in regulating blood cholesterol levels, which could contribute to the prevention of chronic diseases such as cardiovascular and colon tumors. Additionally, these pigments are believed to enhance the immune system.

In \species{C.~vulgaris}, there are also many minerals that play important roles in human nutrition and health, such as potassium, against hypertension, magnesium, important for maintaining nerve activity and muscle contraction, and zinc (essential for enzymatic activity)~\parencite{safi_Morphology_2014}\zxriv{}. It is rich in iron, useful in preventing anemia and playing a significant role in respiration, energy production, DNA synthesis, and cell proliferation. Furthermore, it contains many water-soluble vitamins such as C and B (B1, B2, B6, B12), fat-soluble vitamins (A, D, E, and K), niacin, folic acid, biotin, and pantothenic acid. Vitamin D2 is crucial for bone health and calcium metabolism, reducing the risk of osteomalacia in adults and rickets in children, this organism is important as a supplement also in vegan and vegetarian diets as for B12.

Other potential applications for human health include:
\begin{itemize}
\item antihypertensive effects
\item anti-hypercholesterolemic and anti-hyperlipidemic effects
\item antidiabetic action
\item hepatoprotective action
\item detoxifying effect~\parencite{bito_Potential_2020}
\end{itemize}

\begin{figure}[H]
\centering
% Subfigure 1
	\subcaptionbox*{\label{subfig:chlorella_powder}}%
		{\includegraphics[width=0.45\textwidth]{images/chlorella_powder-mod}}%
\hfill
% Subfigure 2
	\subcaptionbox*{\label{subfig:chlorella_micro}}%
		{\includegraphics[width=0.45\textwidth]{images/chlorella_micro-mod}}%
\caption{Powder form and microscopic view of \species{C.~vulgaris}}
\label{fig:chlorella_views}
\end{figure}

\subsubsection{\species{Palmaria palmata}}
The macroalga used for this study is \species{P.~palmata}, a name derived from its characteristic hand-like shape, with flattened oblong lobes that extend from the center to the fronds \figref{fig:palmaria_views}. It belongs to the family of red algae and on average grows \qtyrange{10}{20}{\centi\metre} but can reach lengths of up to \qty{50}{\centi\metre} and a width of \qtyrange{3}{8}{\centi\metre}~\parencite{stevant_Concise_2023}. It is mostly found in shallow intertidal or subtidal zones, typical of the Atlantic Ocean, and thrives in cold and turbulent waters~\parencite{mouritsen_human_2013}. It is characterized by a pseudo-perennial life cycle, in which the fronds show a continuous process of growth, reproduction, and aging, while the substrate on which it grows can persist for several years. Like most macroalgae species, \species{P.~palmata} relies on the production and release of spores for reproduction, dissemination, and substrate colonization. To become fertile and induce spore maturation, \species{P.~palmata} has specific environmental requirements regarding temperature and light conditions, which influence the timing of reproduction in situ~\parencite{stevant_Concise_2023}. This macroalga is rich in proteins, comprising approximately 35\% of its dry weight, with variations due to seasonality and geographical origin. Although the lipid content of \species{P.~palmata} is considered relatively low (\numrange{0.3}{3.8}\% of dry weight), in line with many other macroalgae, it should be emphasized that these lipids are readily assimilated by the human body. Moreover, they are rich in \gls{epa}, a fatty acid that plays a crucial role in the prevention of non-communicable diseases. \gls{epa} is a type of omega-3 \gls{pufa} that significantly promotes cardiovascular and neurodegenerative health, in addition to possessing potent antioxidant and anti-inflammatory properties. In algae, \glspl{pufa} are mainly present in structural membrane lipids, such as phospholipids and glycolipids. Recent studies have associated these lipids with various bioactive properties, including anti-tumor, anti-inflammatory, antimicrobial, and antiviral activities. It also contains all essential amino acids and several amino acids like mycosporines, carotenoids such as α-carotene, β-carotene, and lutein, and sterols such as cholesterol and desmosterol. Recent studies have shown that extracts of this macroalga exhibit antioxidant and antiproliferative activity. \glsxtrlong{pufa}, particularly \gls{epa} and \gls{dha}, are known for their cardioprotective and anti-inflammatory properties. These compounds can downregulate the production of pro-inflammatory cytokines associated with metabolic syndrome, a set of conditions that increase the risk of heart disease, type 2 diabetes, and other related pathologies~\parencite{lopes_New_2019}.

\begin{figure}[H]
\centering
% Subfigure 1
	\subcaptionbox*{\label{subfig:palmaria_flakes}}%
		{\includegraphics[width=0.5\textwidth]{images/palmata_flakes-mod-low}}%
\hfill
% Subfigure 2
	\subcaptionbox*{\label{subfig:palmaria_micro}}%
		{\includegraphics[width=0.5\textwidth]{images/palmata_micro-2-rotate}}%
\caption{Flakes and microscopic view of \species{P.~palmata}}%
\label{fig:palmaria_views}
\end{figure}


\subsection{Development of the algae burgers}
During the hamburger creation phase, careful selection of ingredients was made to maintain the color and nutritional properties of the algae. For example, for burgers based on \species{A.~platensis}, green ingredients such as zucchini, spinach, peas, and broccoli were used, for those based on \species{C.~vulgaris}, carrots, turmeric, curry, and onions were added, and for burgers containing \species{P.~palmata}, beetroots, red beans, and red onions were used.

To ensure the completeness and nutritional balance of the meals, complementary ingredients such as oat flour, xanthan gum, konjac, guar gum, coconut oil, quinoa, couscous, and potatoes were integrated. These were also added to improve the texture of the burgers and provide additional nutritional benefits. This targeted selection of ingredients allowed for the creation of healthy, nutritious burgers with a balanced nutritional profile. Furthermore, the amount of algal biomass to be added to the final product was determined. During this process, various tests were conducted using different formulations for each algal spieces, varying the percentages. The algal percentage and the respective ingredients chosen for each burger are listed in the Tables~\ref{tab:formulation_spices-spirulina},~\ref{tab:formulation_spices-chlorella}~and~\ref{tab:formulation_spices-palmaria} and these were the formulations used for subsequent nutritional analyses.

\begin{table}[H]
\footnotesize
\centering
	\begin{subcaptionblock}{0.68\textwidth}
	\centering
		\begin{tabular}{lclclc}
	\toprule
	\belowrulesepcolor{colspir}
	\rowcolor{colspir}
		\multicolumn{2}{c}{\textbf{\species{A.~platensis} 4\%}} & \multicolumn{2}{c}{\textbf{\species{A.~platensis} 6\%}} & \multicolumn{2}{c}{\textbf{\species{A.~platensis} 8\%}} \\[\spheader]
	\rowcolor{colspir}
		\textbf{Ingredient} & \textbf{grams} & \textbf{Ingredient} & \textbf{grams} & \textbf{Ingredient} & \textbf{grams} \\
	\aboverulesepcolor{colspir}
	\midrule
		Peas									& \num{51}	& Peas									& \num{49}	& Peas									& \num{47} \\[\spbtwrows]
		Oat flour								& \num{10}	& Oat flour								& \num{10}	& Oat flour								& \num{10} \\[\spbtwrows]
		Oat flakes								& \num{8}	& Oat flakes							& \num{8}	& Oat flakes							& \num{8} \\[\spbtwrows]
		Courgettes								& \num{8}	& Courgettes							& \num{8}	& Courgettes							& \num{8} \\[\spbtwrows]
		Onion									& \num{7}	& Onion									& \num{7}	& Onion									& \num{7} \\[\spbtwrows]
		\makecell{Xanthan\\[\spbtwlines]gum}	& \num{2}	& \makecell{Xanthan\\[\spbtwlines]gum}	& \num{2}	& \makecell{Xanthan\\[\spbtwlines]gum}	& \num{2} \\[\spbtwrows]
		\makecell{Coconut\\[\spbtwlines]oil}	& \num{8}	& \makecell{Coconut\\[\spbtwlines]oil}	& \num{8}	& \makecell{Coconut\\[\spbtwlines]oil}	& \num{8} \\[\spbtwrows]
		Spices									& \num{2}	& Spices								& \num{2}	& Spices								& \num{2} \\[\spbtwrows]
		Microalga								& \num{4}	& Microalga								& \num{6}	& Microalga								& \num{8} \\[\spbtwrows]
		Tot										& \num{100}	& Tot									& \num{100}	& Tot									& \num{100} \\
	\bottomrule


%	\toprule
%		\multicolumn{2}{c}{\textbf{\species{A.~platensis} 4\%}} & \multicolumn{2}{c}{\textbf{\species{A.~platensis} 6\%}} & \multicolumn{2}{c}{\textbf{\species{A.~platensis} 8\%}} \\
%		\textbf{What?} & \textbf{grams} & \textbf{What?} & \textbf{grams} & \textbf{What?} & \textbf{grams} \\
%	\midrule
%		Peas		& \qty{51}{\gram}	& Peas			& \qty{49}{\gram}	& Peas			& \qty{47}{\gram} \\
%		Oat flour	& \qty{10}{\gram}	& Oat flour		& \qty{10}{\gram}	& Oat flour		& \qty{10}{\gram} \\
%		Oat flakes	& \qty{8}{\gram}	& Oat flakes	& \qty{8}{\gram}	& Oat flakes	& \qty{8}{\gram} \\
%		Courgettes	& \qty{8}{\gram}	& Courgettes	& \qty{8}{\gram}	& Courgettes	& \qty{8}{\gram} \\
%		Onion		& \qty{7}{\gram}	& Onion			& \qty{7}{\gram}	& Onion			& \qty{7}{\gram} \\
%		Xanthan gum	& \qty{2}{\gram}	& Xanthan gum	& \qty{2}{\gram}	& Xanthan gum	& \qty{2}{\gram} \\
%		Coconut oil	& \qty{8}{\gram}	& Coconut oil	& \qty{8}{\gram}	& Coconut oil	& \qty{8}{\gram} \\
%		Spices		& \qty{2}{\gram}	& Spices		& \qty{2}{\gram}	& Spices		& \qty{2}{\gram} \\
%		Microalga	& \qty{4}{\gram}	& Microalga		& \qty{6}{\gram}	& Microalga		& \qty{8}{\gram} \\
%		Tot			& \qty{100}{\gram}	& Tot			& \qty{100}{\gram}	& Tot			& \qty{100}{\gram} \\
%	\bottomrule
\end{tabular}%
	\end{subcaptionblock}%
\hspace*{\hbtwsfig}%
	\begin{subcaptionblock}[][18.91em][c]{0.235\textwidth}
	\centering
		\begin{tabular}{lc}
	\toprule
	\rowcolor{colspir}
		\textbf{\shortstack{Spice}} & \textbf{\Centerstack{Amount\\(grams)}} \\
	\midrule
		Salt		& \num{0,4} \\
		Rosemary	& \num{0,4} \\
		Garlic		& \num{0,3} \\
		Mustard		& \num{0,5} \\
		Lemon		& \num{0,4} \\
		Tot			& \num{2} \\
	\bottomrule
\end{tabular}%
	\end{subcaptionblock}%
\caption{Formulation of \species{A.~platensis} burgers for each percentage of algae and corresponding spices used}
\label{tab:formulation_spices-spirulina}
\end{table}

\begin{table}[H]
\footnotesize
\centering
	\begin{subcaptionblock}{0.68\textwidth}
	\centering
		\begin{tabular}{cccccc}
	\toprule
	\rowcolor{colchlo}
		\multicolumn{2}{c}{\textbf{\species{C.~vulgaris} 4\%}} & \multicolumn{2}{c}{\textbf{\species{C.~vulgaris} 8\%}} & \multicolumn{2}{c}{\textbf{\species{C.~vulgaris} 12\%}} \\[\spheader]
	\rowcolor{colchlo}
		\textbf{Ingredient} & \textbf{grams} & \textbf{Ingredient} & \textbf{grams} & \textbf{Ingredient} & \textbf{grams} \\
	\midrule
		Chickpeas								& \num{51}	& Chickpeas								& \num{47}	& Chickpeas								& \num{43} \\[\spbtwrows]
		Oat flakes								& \num{20}	& Oatflakes								& \num{20}	& Oatflakes								& \num{20} \\[\spbtwrows]
		Carrots									& \num{15}	& Carrots								& \num{15}	& Carrots								& \num{15} \\[\spbtwrows]
		\makecell{Xanthan\\[\spbtwlines]gum}	& \num{0}	& \makecell{Xanthan\\[\spbtwlines]gum}	& \num{0}	& \makecell{Xanthan\\[\spbtwlines]gum}	& \num{0} \\[\spbtwrows]
		\makecell{Coconut\\[\spbtwlines]oil}	& \num{8}	& \makecell{Coconut\\[\spbtwlines]oil}	& \num{8}	& \makecell{Coconut\\[\spbtwlines]oil}	& \num{8} \\[\spbtwrows]
		Spices									& \num{2}	& Spices								& \num{2}	& Spices								& \num{2} \\[\spbtwrows]
		Microalga								& \num{4}	& Microalga								& \num{8}	& Microalga								& \num{12} \\[\spbtwrows]
		Tot										& \num{100}	& Tot									& \num{100}	& Tot									& \num{100} \\[\spbtwrows]
	\bottomrule
\end{tabular}%
	\end{subcaptionblock}%
\hspace*{\hbtwsfig}%
	\begin{subcaptionblock}[][16.5em][c]{0.235\textwidth}
	\centering
		\begin{tabular}{lc}
	\toprule
		\textbf{\shortstack{Spice}} & \textbf{\Centerstack{Amount\\(grams)}} \\
	\midrule
		Salt		& \num{0,2} \\
		Rosemary	& \num{0,4} \\
		Garlic		& \num{0,4} \\
		Tumeric		& \num{0,15} \\
		Curry		& \num{0,15} \\
		Mustard		& \num{0,3} \\
		Lemon		& \num{0,4} \\
		Tot			& \num{2} \\
	\bottomrule
\end{tabular}%
	\end{subcaptionblock}%
\caption{Formulation of \species{C.~vulgaris} burgers for each percentage of algae and corresponding spices used}
\label{tab:formulation_spices-chlorella}
\end{table}

\begin{table}[H]
\footnotesize
\centering
	\begin{subcaptionblock}{0.5\textwidth}
	\centering
		\begin{tabular}{cccc}
	\toprule
		\multicolumn{2}{c}{\textbf{\species{P.~palmata} 1.5\%}} & \multicolumn{2}{c}{\textbf{\species{P.~palmata} 3\%}} \\[\spheader]
		\textbf{Ingredient} & \textbf{grams} & \textbf{Ingredient} & \textbf{grams} \\
	\midrule
		Red beans	& \num{57}		& Red beans		& \num{54} \\
		Cous cous	& \num{7}		& Cous cous		& \num{7} \\
		Oat flour	& \num{8}		& Oat flour		& \num{8} \\
		Beetroots	& \num{8}		& Beetroots		& \num{8} \\
		Red onion	& \num{7}		& Red onion		& \num{7,5} \\
		Xanthan gum	& \num{0,5}		& Xanthan gum	& \num{0,5} \\
		Coconut oil	& \num{9}		& Coconut oil	& \num{9} \\
		Spices		& \num{2}		& Spices		& \num{2} \\
		Microalga	& \num{1,5}		& Microalga		& \num{3} \\
		Tot			& \num{100}		& Tot			& \num{100} \\
	\bottomrule
\end{tabular}%
	\end{subcaptionblock}%
\hspace*{\hbtwsfig}%
	\begin{subcaptionblock}[][16em][c]{0.235\textwidth}
	\centering
		\begin{tabular}{lc}
	\toprule
	\rowcolor{colpalm}
		\textbf{\shortstack{Spice}} & \textbf{\Centerstack{Amount\\(grams)}} \\
	\midrule
		Salt		& \num{0,6} \\
		Rosemary	& \num{0,2} \\
		Garlic		& \num{0,4} \\
		Mustard		& \num{0,2} \\
		Ginger		& \num{0,2} \\
		Lemon		& \num{0,4} \\
		Tot			& \num{2} \\
	\bottomrule
\end{tabular}%
	\end{subcaptionblock}%
\caption{Formulation of \species{P.~palmata} burgers for each percentage of algae and corresponding spices used}
\label{tab:formulation_spices-palmaria}
\end{table}

\subsection{Nutritional analysis}
All analyses were repeated twice for each percentage of algae present in each burger formulation. In addition, a control sample for each type of burger was included, without addition of algae, in order to have a point of comparison for the conducted analyses. This approach allowed for the evaluation of the impact of different algal concentrations on the characteristics of the burgers, while the control sample provided a reference for assessing the accuracy, precision, and variability of the results obtained.


\subsubsection{Moisturized content}
To determine the moisture content of algae burgers at various percentages and control burgers, the~\cite{aoac_2000}\zxriv{} method was relied upon. First, aluminum plates were prepared and dried in an oven at \qty{105}{\degreeCelsius} for \qty{3}{hours} to remove any moisture residue. After this process, they were transferred to a desiccator for \qty{30}{minutes} to cool \figref{fig:desiccator}. This cooling phase is important to avoid any weight variations due to temperature conditions.

\begin{figure}[H]
	\centering
	\includegraphics[width=0.45\textwidth]{images/desiccator-mod}
	\caption{Desiccator}
	\label{fig:desiccator}
\end{figure}

Next, the dry plates were weighed to obtain their initial weight. For the analysis procedure, \num{22} samples of \qty{2}{\gram} each were prepared as follows: \num{2} samples of \species{A.~platensis} burgers for each algae percentage (4\%, 6\%, 8\%), \num{2} samples of \species{C.~vulgaris} burgers for each algae percentage (4\%, 8\%, 12\%), \num{2} samples of \species{P.~palmata} burgers for each algae percentage (1.5\%, 3\%), and 2 samples of control burgers without algae for each formulation (i.e., 2 with ingredient from \species{A.~platensis} burger, 2 with ingredient from \species{C.~vulgaris} burgers, 2 with ingredient from \species{P.~palmata} burgers). These samples were then cooked, homogenized, and spread onto the aluminum plates. The plates with the samples were then placed back in the oven at \qty{105}{\degreeCelsius} for \qty{3}{hours} to completely dry the sample \figref{fig:samples_in_oven_1}.

\begin{figure}[H]
	\centering
	\includegraphics[width=0.5\textwidth]{images/samples_in_oven_1-mod}
	\caption{Samples in the oven}
	\label{fig:samples_in_oven_1}
\end{figure}

All were then transferred to the desiccator for \qty{30}{minutes} for cooling. Subsequently, each plate containing the dried sample was weighed. The calculation to determine the moisture of each sample was performed using the following formula:\zxriv{}
\[
	\text{Moisture \%} = \frac{(W_1 - W_2) * 100}{W_1}
\]
where $ W_1 $ represents the weight of the sample before drying, and $ W_2 $ represents the weight of the sample after drying.


\subsubsection{Ash content}
To determine the ash content of algae burgers at various percentages and control burgers, the AOAC~2000 method (AOAC, 2000\zxriv{}) was relied upon. Firstly, an aluminum support and specially designed cups to contain the samples were created. These were then placed in an oven heated to a temperature of \qty{550}{\degreeCelsius} for an entire night to eliminate any impurities present in the samples \figref{subfig:samples_in_oven_2}. This heating phase is essential to ensure that the samples are completely dried and free from any volatile material that could influence the analysis results. The samples resulting from the moisture content analysis were recovered and used for these analyses and then placed in the oven heated to \qty{550}{\degreeCelsius} for an entire night and then dried again before being weighed again \figref{subfig:residue_ash}.

\begin{figure}[H]
\centering
% Subfigure 1
	\subcaptionbox%
	{Samples in the oven\label{subfig:samples_in_oven_2}}%
		{\includegraphics[width=0.45\textwidth]{images/samples_in_oven_2-cut-mod}}%
\hspace*{0.1\textwidth}%
% Subfigure 2
	\subcaptionbox%
		{Residue of ash content after combustion\label{subfig:residue_ash}}%
		{\includegraphics[width=0.45\textwidth]{images/residue_ash-mod}}%
\caption{}
%\label{fig:}
\end{figure}

This process allows determining the weight of the residual ashes after the complete combustion of the organic materials present in the samples. The calculation of the ash content is then performed using the following formula:
\[
\text{Ash \%} = \frac{W_\text{A} * 100}{W_2}
\]
where $ W_\text{A} $ represents the weight in grams of the residual ashes after combustion and $ W_2 $ the initial weight of the analyzed sample, and therefore in our case, the weight resulting from the drying at \qty{105}{\degreeCelsius} for the moisture content.


\subsubsection{Lipid content}
%For the determination of lipid content, the modified Bligh and Dyer gravimetric method~\parencite{bligh_RAPID_1959} was used. First, microtubes were prepared for lipid weighing by washing and drying them at \qty{60}{\degreeCelsius} for at least \qty{3}{hours}, followed by placing them in a desiccator for another \qty{3}{hours} before proceeding with weighing on a precision balance. Once the weight of each microtube was obtained, the actual methodology can be carried out.
For the determination of lipid content, the modified Bligh and Dyer gravimetric method \parencite{bligh_RAPID_1959} was used. First, microtubes were prepared for lipid weighing by washing and drying them at \qty{60}{\degreeCelsius} for at least \qty{3}{hours}, followed by placing them in a desiccator for another \qty{3}{hours} before proceeding with weighing on a precision balance. Once the weight of each microtube was obtained, the actual methodology can be carried out.

\zxriv{}\qty{30}{\milli\gram} of lyophilized biomass were accurately weighed and transferred into \qty{2}{\milli\metre} microtubes. This step requires precision, as even small variations in the sample quantity can affect the analysis results.

Next, \qty{0.5}{\milli\litre} of glass beads were introduced into the microtubes for extraction. Then, \qty{0.8}{\milli\litre} of distilled water were added, and the mixture is allowed to react for at least \qty{15}{minutes}. This resting period is crucial as it allows the samples to soften, making lipid extraction more efficient. Working under a fume hood to ensure safety and reduce the risk of contamination, the following steps were followed sequentially:
\begin{itemize}
\item adding \qty{0,5}{\milli\litre} of methanol and \qty{0,2}{\milli\litre} of chloroform.
\item homogenizing the mixture using a vortex and then in a beadmill \figref{fig:beadmill} at \qty{30}{\hertz} for \qty{3}{minutes}. This ensures uniform distribution of solvents and samples.
\item once homogenized, transferring everything (including glass beads and biomass) into a glass centrifuge tube and cleaning the used Eppendorf with a mixture of \qty{1,5}{\milli\litre} of methanol and \qty{0,75}{\milli\litre} of chloroform, then adding this solution to the tube.
\item mixing again, this time using only the vortex.
\item adding additional chloroform, \qty{1}{\milli\litre}.
\item homogenizing with a vortex.
\item adding distilled water, \qty{1}{\milli\litre}.
\item homogenizing the samples twice with a vortex.
\end{itemize}

\begin{figure}[H]
	\centering
	\includegraphics[width=0.6\textwidth]{images/beadmill-mod}
	\caption{Beadmill}
	\label{fig:beadmill}
\end{figure}

After this extraction phase, the samples were centrifuged at 2000~×~\textsl{g}
%\footnote{\zariv{} \href{https://www.fishersci.it/it/it/scientific-products/centrifuge-guide/centrifuge-applications-tools/rpm-rcf-calculator.html}{1} \href{https://www.sigmaaldrich.com/IT/it/support/calculators-and-apps/g-force-calculator}{2} \href{https://tomy.amuzainc.com/blog/optimizing-centrifuge-speed/}{3}}
for \qty{10}{minutes} at room temperature. This process separates the lipids from the solvent-sample mixture. Then, the chloroform containing the extracted lipids is pipetted into new tubes using a Pasteur pipette. The appearance of the extraction should be translucent \figref{fig:extraction_example}.

\begin{figure}[H]
	\centering
	\includegraphics[width=0.5\textwidth]{images/extraction_example-mod}
	\caption{Example of \species{C.~vulgaris} extraction}
	\label{fig:extraction_example}
\end{figure}

Otherwise, centrifugation is repeated to ensure complete separation. Next, a known amount of chloroform, usually between \qtyrange{0.5}{1}{\milli\litre}, is pipetted into the microtubes for subsequent weighing. The microtubes are then placed in a dry bath at \qty{60}{\degreeCelsius} \figref{fig:dry_bath_fume_hood} to allow complete evaporation of chloroform, all of course under a fume hood.

\begin{figure}[H]
	\centering
	\includegraphics[width=0.5\textwidth]{images/dry_bath_fume_hood-mod}
	\caption{Dry bath at \qty{60}{\degreeCelsius} in a fume hood}
	\label{fig:dry_bath_fume_hood}
\end{figure}

Following this phase, the tubes were transferred to the desiccator for at least \qty{3}{hours} to ensure they are completely dry before weighing on a precision balance. It is important to use tweezers to handle the tubes to avoid contamination and ensure accurate results. This detailed and rigorous procedure allows us to obtain reliable data on the lipid content of our samples, which will be obtained following this calculation:\zxriv{}
\[
	\text{Lipid \%}
	=
	\frac
		{\frac{(W_f - W_t) * Ch}{E_{Ch}}}
		{Ws}
		* 100
\]
where $ Wf $ is the final weight, $ Wt $ is the weight of the glass tubes, $ Ch $ is the total chloroform (\qty{2}{\milli\litre}), $ ECh $ is the evaporated chloroform, and $ Ws $ is the weight of the samples (\qty{30}{\milli\gram}).

\subsubsection{Protein content}
For the final type of analysis, we focused on the elemental analysis of the dried samples to determine the carbon (C), hydrogen (H), and nitrogen (N) content, and subsequently estimate the protein content.
Burger samples were dried by lyophilization to remove moisture and ensure the accuracy of the analysis. The dry samples were ground into a fine powder using a ball mill (RETSCH MM 300). This step is crucial to ensure adequate sample homogeneity, reducing variations in analytical results. Approximately \qtyrange{2}{3}{\milli\gram} of the obtained powder were taken and subjected to a CHN elemental analyzer (Elementar Vario EL III) to quantify the total carbon, hydrogen, and nitrogen content. The analyzer uses a high-temperature combustion process to decompose the sample, allowing the measurement of the produced gases (\ch{CO2}, \ch{H2O}, \ch{N2}) to determine the concentration of each element.
The sample was combusted in an oxygen-rich environment at around \qty{1000}{\degreeCelsius}. During combustion, the nitrogen present in the sample is converted to \ch{N2} gas. The produced gases passed through various filters and chemical reagents to remove potential interferents such as \ch{CO2} and \ch{H2O}, ensuring that only pure nitrogen reached the detector. The \ch{N2} gas was measured using a thermal conductivity detector. The detected nitrogen concentration is directly proportional to the protein content in the sample.
The protein content was estimated using the Dumas method, as specified in
\zxriv{}ISO 16634 and AOAC 990.03 standards (Horwitz \& Latimer, 2006)\zxriv{}
. This method involves multiplying the total nitrogen content by a conversion factor of 6.25, according to FAO guidelines~\parencite{fao_2003}:
\[
    \text{Protein (\%)} = \text{Total nitrogen} * 6.25
\]

This factor assumes that most proteins contain about \num{16}\% nitrogen \mbox{(1/0.16 = 6.25)}. Therefore, the factor 6.25 is an estimate based on the average nitrogen content in proteins but can vary depending on the specific protein composition in the analyzed samples.

\subsection{Statistical Analysis}
\subsubsection{Univariate Analysis}
\Gls{anova} was used to test for differences in all the considered variables (i.e., ash, moisture, lipid, and proteins contents) among different formulations separately for each algal species. The design for the consisted of the single factor ‘Composition\%’, fixed, with \numrange[range-phrase={--}]{3}{4} levels corresponding to the control, and the different \%\zxriv{} of algae contained in the burger formulation. Prior to analysis, the assumption of normality of the response variables was tested with the Shapiro–Wilk test. The Cochran’s C-test was used to check for the homogeneity of variances ~\parencite{underwood_Experiments_1996} and data were $ \log(x + 1) $ transformed to stabilize variance if required. Post-hoc pairwise t-test were performed to examine specific differences in the investigated variables among compositions.

\subsubsection{Multivariate Analysis}
Distance-based permutational multivariate analysis of variance (PERMANOVA;~\cite{anderson_new_2001}) was used to test for differences in the overall set of nutritional factors among burgers containing the different \% of the algae. Also in this case, the analysis included the single factor ‘Treatment’, with eight levels (corresponding to the \num{3} compositions for \species{Spirulina} and \species{C.~vulgaris}, and the two compositions of \species{P.~palmata}). The analysis was based on Euclidean distance and tests were done using \num{999} permutations. Post-hoc pairwise t-test were performed among levels. However, due to the limited number of samples P-values were obtained using a Monte Carlo random sample from the asymptotic permutation distribution~\parencite{anderson_Generalized_2003}. A canonical \gls{cap}~\parencite{anderson_CANONICAL_2003} was performed for factor ‘Treatment’ to depict patterns of variation among different burger compositions. Correlations of individual variables with the two canonical axes were represented as lines in a projection biplot.

\subsection{Sensory Analysis}
\subsubsection{Preparation of Initial Formulations}
Before the nutritional analyses, preliminary sensory analyses were conducted to identify the most preferred burger formulations among consumers. Thirteen \species{Spirulina} burgers, nine \species{C.~vulgaris} burgers, and eight \species{P.~palmata} burgers with different ingredient percentages and mixes were tasted. From this initial assessment, the two most appreciated combinations were identified, and a subsequent sensory analysis was performed to select a single formulation for each algae type, which would then be subjected to nutritional analyses and further final sensory evaluations.

\subsubsection{First Sensory Analysis}
A sample of \num{28} subjects was given a questionnaire concerning six different burgers, divided as follows: two formulations with different ingredients for each alga (\num{4}\% \species{Spirulina}, \num{4}\% \species{C.~vulgaris} and \num{1.5}\% \species{P.~palmata}). The questions addressed taste, smell, color, and texture of each algae pair, and a final preference judgment between the two.
The formulations selected from this first sensory analysis, as reported in Tables 3, 4, and 5, were then subjected to the aforementioned nutritional analyses.

\subsubsection{Last Sensory Analysis}
Four \qty{100}{\gram} burgers were prepared for each algae percentage, totaling \num{32} burgers. These were cooked, cut into small pieces, and divided into tasting samples on designated plates \figref{fig:last_sensory_analysis}. The \species{P.~palmata} burgers were coded as 1A for \num{1.5}\% and 1B for \num{3}\%. The \species{Spirulina} burgers were coded as 2A, 2B, 2C for \num{4}\%, \num{6}\%, and \num{8}\%, respectively. Finally, the \species{C.~vulgaris} burgers were coded as 3A for \num{4}\%, 3B for \num{8}\%, and 3C for \num{12}\%.
Napkins and glasses of water were provided, with the requirement to rinse the mouth after each tasting. The analysis was conducted with \num{41} participants, each seated at desks with side panels to isolate tasters and prevent any form of communication and judgment interference. These desks featured a shaded window through which the various burgers were passed for tasting, avoiding any bias or prejudice in the responses. Each subject was given a QR~code to access the questionnaire, which included general questions about gender, age, origin, dietary habits, and specific questions about the burgers. Initially, an image of the burger was shown for an aesthetic judgment, followed by tasting questions regarding smell, texture, and taste. At the end of each section, participants were asked to indicate their preferred burger code and whether they would consider buying it. At the end of the questionnaire, they were asked to indicate their overall favorite burger among all the tasted types, adding any comments or final suggestions if they had any.

\begin{figure}[H]
	\centering
	\includegraphics[width=1\textwidth]{images/last_sensory_analysis-mod}
	\caption{Last sensory analysis}
	\label{fig:last_sensory_analysis}
\end{figure}


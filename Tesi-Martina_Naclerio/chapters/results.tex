% !TeX encoding = UTF-8

\section{Results}
\subsection{Univariate analysis}
\species{Spirulina}

\gls{anova} detected significant difference among different composition of \species{Spirulina} burgers for all the investigated variables, except for lipid contents \tabref{tab:anova-spirulina}.

\begin{table}[H]
	\centering
	\scriptsize
	\begin{tabular}{l@{\hspace{1ex}}ccccccccccccc}
	\toprule
		 &  & \multicolumn{3}{c}{\textbf{Ash content}} & \multicolumn{3}{c}{\textbf{Moisture}} & \multicolumn{3}{c}{\textbf{Lipid}} & \multicolumn{3}{c}{\textbf{Protein}} \\[\spheader]
		\textbf{Source} & \textbf{d.f.} & \textbf{MS} & \textbf{F} & \textbf{P} & \textbf{MS} & \textbf{F} & \textbf{P} & \textbf{MS} & \textbf{F} & \textbf{P} & \textbf{MS} & \textbf{F} & \textbf{P} \\
	\midrule
		\textbf{Composition \%} & \num{3} & \num{4.38} & \num{38.17} & \num{0.002} & \num{48.54} & \num{45,69} & \num{0,001} & \num{1.39} & \num{0.61} & \num{0.643} & \num{9.73} & \num{28.41} & \num{0.004} \\[\spbtwrowsA]
		\textbf{Transformation} &  & \multicolumn{3}{c}{none} & \multicolumn{3}{c}{none}& \multicolumn{3}{c}{none}& \multicolumn{3}{c}{none} \\[\spbtwrowsA]
		\textbf{Shapiro-Wilk test} &  & \multicolumn{3}{c}{$ W = \num{0,977}^{\text{NS}} $} & \multicolumn{3}{c}{$ W = \num{0,972}^{\text{NS}} $} & \multicolumn{3}{c}{$ W = \num{0,872}^{\text{NS}} $} & \multicolumn{3}{c}{$ W = \num{0,974}^{\text{NS}} $} \\[\spbtwrowsA]
		\textbf{Cochran's \emph{C}-test} &  & \multicolumn{3}{c}{$ C = \num{0,747}^{\text{NS}} $} & \multicolumn{3}{c}{$ C = \num{0,735}^{\text{NS}} $} & \multicolumn{3}{c}{$ C = \num{0,394}^{\text{NS}} $} & \multicolumn{3}{c}{$ C = \num{0,646}^{\text{NS}} $} \\[\spbtwrowsA]
		\textbf{Pairwise test} &  & \multicolumn{3}{c}{Ctrl = \num{4}\% < \num{6}\% = \num{8}\%} & \multicolumn{3}{c}{Ctrl > \num{4}\% = \num{6}\% = \num{8}\%}& \multicolumn{3}{c}{-}& \multicolumn{3}{c}{Ctrl = \num{4}\% = \num{6}\% < \num{8}\%} \\
	\bottomrule
\end{tabular}
	\caption{ANOVA results of nutritional content for \species{Spirulina} burgers}
	\label{tab:anova-spirulina}
\end{table}

For the ash content, pairwise tests indicated that the control burger did not differ significantly from the burger containing 4\%, both having a significant lower ash content with respect to burgers with 6\% and 8\%, which in turn were comparable (\tab{}~\ref{tab:anova-spirulina}, \fig{}~\ref{subfig:spirulina_ashes}).
For moisture content, the control was significantly higher compared to 4\%, 6\%, and 8\%, which do not differ significantly from each other (\tab{}~\ref{tab:anova-spirulina}, \fig{}~\ref{subfig:spirulina_moisture}). No significant differences among the different groups were detected for lipid content (\tab{}~\ref{tab:anova-spirulina}, \fig{}~\ref{subfig:spirulina_lipids}). Finally, for the protein content, there were no significant difference between the control and the burgers with 4\% and 6\%, all of them containing a significantly higher amount of protein compared to the 8\% burger (\tab{}~\ref{tab:anova-spirulina}, \fig{}~\ref{subfig:spirulina_proteins}).

% !TeX encoding = UTF-8

\begin{figure}[H]
\centering
% Subfigure 1
	\subcaptionbox%
	{Ashes content percentage\label{subfig:spirulina_ashes}}%
		{\includegraphics[width=\subfigwth]{../plots/mpl/barplots_anova/barplots/barplots-outs/spirulina_ashes.pdf}}%
\hspace*{\hbtwsfig}%
% Subfigure 2
	\subcaptionbox%
	{Moisture content percentage\label{subfig:spirulina_moisture}}%
		{\includegraphics[width=\subfigwth]{../plots/mpl/barplots_anova/barplots/barplots-outs/spirulina_moisture.pdf}}%
\\[2ex]
% Subfigure 3
	\subcaptionbox%
	{Lipids content percentage\label{subfig:spirulina_lipids}}%
		{\includegraphics[width=\subfigwth]{../plots/mpl/barplots_anova/barplots/barplots-outs/spirulina_lipids.pdf}}%
\hspace*{\hbtwsfig}%
% Subfigure 4
	\subcaptionbox%
	{Proteins content percentage\label{subfig:spirulina_proteins}}%
		{\includegraphics[width=\subfigwth]{../plots/mpl/barplots_anova/barplots/barplots-outs/spirulina_proteins.pdf}}%
\caption%
{ANOVA results fot \species{Spirulina}}
\label{fig:spirulinabla}
\end{figure}


\species{C.~vulgaris}

For \species{C.~vulgaris} burgers, \gls{anova} detected significant differences for ash and moisture content, but not for lipid and protein contents (\tab{}~\ref{tab:anova-chlorella}).

\begin{table}[H]
	\centering
	\scriptsize
	\begin{tabular}{l@{\hspace{1ex}}ccc@{\hspace{1ex}}ccc@{\hspace{1ex}}ccc@{\hspace{1ex}}ccc@{\hspace{1ex}}c}
	\toprule
		 &  & \multicolumn{3}{c}{\textbf{Ash content}} & \multicolumn{3}{c}{\textbf{Moisture}} & \multicolumn{3}{c}{\textbf{Lipid}} & \multicolumn{3}{c}{\textbf{Protein}} \\[\spheader]
		\textbf{Source} & \textbf{d.f.} & \textbf{MS} & \textbf{F} & \textbf{P} & \textbf{MS} & \textbf{F} & \textbf{P} & \textbf{MS} & \textbf{F} & \textbf{P} & \textbf{MS} & \textbf{F} & \textbf{P} \\
	\midrule
		\textbf{Composition \%} & \num{3} & \num{0,32} & \num{11.45} & \num{0.02} & \num{87.45} & \num{254,4} & \num[output-exponent-marker = e]{5.09e-05} & \num{3,19} & \num{3,57} & \num{0.126} & \num{0.45} & \num{5,46} & \num{0.067} \\[\spbtwrowsA]
		\textbf{Transformation} &  & \multicolumn{3}{c}{none} & \multicolumn{3}{c}{none}& \multicolumn{3}{c}{none}& \multicolumn{3}{c}{none} \\[\spbtwrowsA]
		\textbf{Shapiro-Wilk test} &  & \multicolumn{3}{c}{$ W = \num{0,906}^{\text{NS}} $} & \multicolumn{3}{c}{$ W = \num{0,967}^{\text{NS}} $} & \multicolumn{3}{c}{$ W = \num{0,989}^{\text{NS}} $} & \multicolumn{3}{c}{$ W = \num{0,973}^{\text{NS}} $} \\[\spbtwrowsA]
		\textbf{Cochran's \emph{C}-test} &  & \multicolumn{3}{c}{$ C = \num{0,455}^{\text{NS}} $} & \multicolumn{3}{c}{$ C = \num{0,818}^{\text{NS}} $} & \multicolumn{3}{c}{$ C = \num{0,721}^{\text{NS}} $} & \multicolumn{3}{c}{$ C = \num{0,617}^{\text{NS}} $} \\[\spbtwrowsA]
		\textbf{Pairwise test} &  & \multicolumn{3}{c}{Ctrl = \num{4}\% < \num{8}\% = \num{12}\%} & \multicolumn{3}{c}{Ctrl > \num{4}\% = \num{6}\% = \num{8}\%}& \multicolumn{3}{c}{-}& \multicolumn{3}{c}{-} \\
	\bottomrule
\end{tabular}
	\caption{ANOVA results of nutritional content for \species{C.~vulgaris} burgers}
	\label{tab:anova-chlorella}
\end{table}

For both ash and moisture, control burgers had significantly higher values with respect to burgers containing \species{C.~vulgaris} burgers at all the analyzed \%, which instead did not differ (\tab{}~\ref{tab:anova-chlorella}, Figures~\ref{subfig:chlorella_ashes}~\ref{subfig:chlorella_moisture}). No significant difference among groups were detected for lipids and proteins (\tab{}~\ref{tab:anova-chlorella}, Figures~\ref{subfig:chlorella_lipids}~\ref{subfig:chlorella_proteins}).
%However, for protein content, the burgers containing \species{C.~vulgaris} appeared to have a higher protein \% if compared with control burgers (Fig .19).

% !TeX encoding = UTF-8

\begin{figure}[H]
\centering
% Subfigure 1
	\subcaptionbox%
	{Ashes content percentage\label{subfig:chlorella_ashes}}%
		{\includegraphics[width=\subfigwth]{../plots/mpl/barplots_anova/barplots/barplots-outs/chlorella_ashes.pdf}}%
\hspace*{\hbtwsfig}%
% Subfigure 2
	\subcaptionbox%
	{Moisture content percentage\label{subfig:chlorella_moisture}}%
		{\includegraphics[width=\subfigwth]{../plots/mpl/barplots_anova/barplots/barplots-outs/chlorella_moisture.pdf}}%
\\[2ex]
% Subfigure 3
	\subcaptionbox%
	{Lipids content percentage\label{subfig:chlorella_lipids}}%
		{\includegraphics[width=\subfigwth]{../plots/mpl/barplots_anova/barplots/barplots-outs/chlorella_lipids.pdf}}%
\hspace*{\hbtwsfig}%
% Subfigure 4
	\subcaptionbox%
	{Proteins content percentage\label{subfig:chlorella_proteins}}%
		{\includegraphics[width=\subfigwth]{../plots/mpl/barplots_anova/barplots/barplots-outs/chlorella_proteins.pdf}}%
\caption%
{ANOVA results fot \species{Chlorella}}
\label{fig:chlorellabla}
\end{figure}


\species{P.~palmata}

Also in this case, \gls{anova} detected significant differences among different compositions only for ash and moisture content (\tab{}~\ref{tab:anova-palmaria}).

\begin{table}[H]
	\centering
	\scriptsize
	\begin{tabular}{l@{\hspace{1ex}}ccccccccccccc}
	\toprule
		 &  & \multicolumn{3}{c}{\textbf{Ash content}} & \multicolumn{3}{c}{\textbf{Moisture}} & \multicolumn{3}{c}{\textbf{Lipid}} & \multicolumn{3}{c}{\textbf{Protein}} \\[\spheader]
		\textbf{Source} & \textbf{d.f.} & \textbf{MS} & \textbf{F} & \textbf{P} & \textbf{MS} & \textbf{F} & \textbf{P} & \textbf{MS} & \textbf{F} & \textbf{P} & \textbf{MS} & \textbf{F} & \textbf{P} \\
	\midrule
		\textbf{Composition \%} & \num{2} & \num{0,84} & \num{42.35} & \num{0.006} & \num{61.54} & \num{24,21} & \num{0,014} & \num{10.83} & \num{8.09} & \num{0.062} & \num{0.01} & \num{9.12} & \num{0.053} \\[\spbtwrowsA]
		\textbf{Transformation} &  & \multicolumn{3}{c}{none} & \multicolumn{3}{c}{none}& \multicolumn{3}{c}{none}& \multicolumn{3}{c}{$ \log (x + 1) $} \\[\spbtwrowsA]
		\textbf{Shapiro-Wilk test} &  & \multicolumn{3}{c}{$ W = \num{0,992}^{\text{NS}} $} & \multicolumn{3}{c}{$ W = \num{0,908}^{\text{NS}} $} & \multicolumn{3}{c}{$ W = \num{0,941}^{\text{NS}} $} & \multicolumn{3}{c}{$ W = \num{0,924}^{\text{NS}} $} \\[\spbtwrowsA]
		\textbf{Cochran's \emph{C}-test} &  & \multicolumn{3}{c}{$ W = \num{0,826}^{\text{NS}} $} & \multicolumn{3}{c}{$ W = \num{0,590}^{\text{NS}} $} & \multicolumn{3}{c}{$ W = \num{0,963}^{\text{NS}} $} & \multicolumn{3}{c}{$ W = \num{0,953}^{\text{NS}} $} \\[\spbtwrowsA]
		\textbf{Pairwise test} &  & \multicolumn{3}{c}{Ctrl = \num{1.5}\% < 3\%} & \multicolumn{3}{c}{Ctrl > \num{1.5}\% = \num{3}\%}& \multicolumn{3}{c}{-}& \multicolumn{3}{c}{-} \\
	\bottomrule
\end{tabular}
	\caption{ANOVA results of nutritional content for \species{P.~palmata} burgers}
	\label{tab:anova-palmaria}
\end{table}

Pairwise tests indicated that the control group and the \num{1.5}\% group did not differ significantly in their ash content, while the burgers with \num{3}\% \species{P.~palmata} had a significantly higher ash content compared to the others (\tab{}~\ref{tab:anova-palmaria}, \fig{}~\ref{subfig:palmaria_ashes}). Conversely, for moisture content, the control group had a higher value of mosture compared to the other two groups, which are not significantly different from each other (\tab{}~\ref{tab:anova-palmaria}, \fig{}~\ref{subfig:palmaria_moisture}). No significant differences were detected for lipid (\tab{}~\ref{tab:anova-palmaria}, \fig{}~\ref{subfig:palmaria_lipids}) and protein (\tab{}~\ref{tab:anova-palmaria}, \fig{}~\ref{subfig:palmaria_proteins}) content between the control burger and those with \num{1.5}\% and \num{3}\% content.

% !TeX encoding = UTF-8

\begin{figure}[H]
\centering
% Subfigure 1
	\subcaptionbox%
	{Ashes bla\label{subfig:palmaria_ashes}}%
		{\includegraphics[width=0.5\textwidth]{../plots/mpl/barplots_anova/barplots/barplots-outs/palmaria_ashes.pdf}}%
\hfill
% Subfigure 2
	\subcaptionbox%
	{Moisture bla\label{subfig:palmaria_moisture}}%
		{\includegraphics[width=0.5\textwidth]{../plots/mpl/barplots_anova/barplots/barplots-outs/palmaria_moisture.pdf}}%
\\[2ex]
% Subfigure 3
	\subcaptionbox%
	{Lipids bla\label{subfig:palmaria_lipids}}%
		{\includegraphics[width=0.5\textwidth]{../plots/mpl/barplots_anova/barplots/barplots-outs/palmaria_lipids.pdf}}%
\hfill
% Subfigure 4
	\subcaptionbox%
	{Proteins bla\label{subfig:palmaria_proteins}}%
		{\includegraphics[width=0.5\textwidth]{../plots/mpl/barplots_anova/barplots/barplots-outs/palmaria_proteins.pdf}}%
\caption%
{Palmaria bla}
\label{fig:palmariabla}
\end{figure}


\subsection{Multivariate analysis}
PERMANOVA detected significance differences among treatments, corresponding to different burgers containing varying \zxriv{}\% of the three algal species \tabref{tab:permanova}. The analysis reveals significant differences in the overall nutritional values among the various formulations examined. Specifically, comparisons between different formulations indicated that main differences were among the three species considered. Conversely, variations in percentages within the same species do not significantly impact the overall nutritional values.
\begin{table}[H]
	\centering
	\small
	\begin{tabular}{lccccccc}
	\toprule
		\textbf{Spice}	& \textbf{d.f.}	& \textbf{SS}	& \textbf{MS}	& \textbf{\emph{Pseudo-F}}	& \textbf{\emph{P(perm)}}	& \textbf{\emph{Unique perms}}	& \textbf{P(MC)} \\
	\midrule
		Tr				& \num{6}		& \num{56,57}	& \num{9,43}	& \num{24,75}		& \num{0,001}		& \num{997}				& \num{0,001} \\[\spbtwrowsA]
		Res				& \num{6}		& \num{3,43}	& \num{0,38}	& 					& 					& 						&  \\[\spbtwrowsA]
		Total			& \num{15}		& \num{60}		& 				& 					& 					& 						&  \\[\spbtwrowsA]
	\bottomrule
\end{tabular}
	\caption{PERMANOVA results}
	\label{tab:permanova}
\end{table}

The \gls{cap} biplot \figref{fig:cap_biplot} provided a clear visualization of these differences, showing how the various formulations cluster based on the algal species, with relatively small variations within easch species. All variables contributed to these pattern of difference, with Spirulina burgers characterized by high nutitrional values with respect to burgers containing the other two species.

\begin{figure}[H]
\centering
\includegraphics[width=0.6\linewidth]{images/cap-mod}
\caption{CAP biplot}
\label{fig:cap_biplot}
\end{figure}

\subsection{Sensory Analysis}
\subsubsection{First Sensory Analysis}
\species{Spirulina}

For the two types of \species{Spirulina} burgers offered for tasting, divided as in \tab{}~\ref{subtab:burgers_ingredients-spirulina}, it can be noted that burger 1B is preferred by the tasters with a \num{75}\% preference compared to burger 1A \figref{subfig:sens_analysis-spirulina}.

\begin{figure}[H]
\scriptsize
\centering
	\begin{minipage}[b]{0.475\textwidth}%
	\centering%
		\begin{tabular}{cccc}
	\toprule
	\rowcolor{colspir}
		\multicolumn{4}{c}{\textbf{\species{A.~platensis} 4\%}} \\[\spheader]
		\multicolumn{2}{c}{\textbf{1A}} & \multicolumn{2}{c}{\textbf{1B}} \\[\spheader]
		\textbf{Ingredient} & \textbf{grams} & \textbf{Ingredient} & \textbf{grams} \\
	\midrule
		Lupines			& \num{48}	& Lupines + lentils & \num{56} \\[\spbtwrows]
		Oat flour		& \num{15}	& Oat flour			& \num{10} \\[\spbtwrows]
		Carrots			& \num{10}	& Oat flakes		& \num{5} \\[\spbtwrows]
		Spinach			& \num{10}	& Courgettes		& \num{15} \\[\spbtwrows]
						& 			& Onion				& \num{7} \\[\spbtwrows]
		Xanthan gum		& \num{2}	& Xanthan gum		& \num{2} \\[\spbtwrows]
		Cocconut oil	& \num{8}	& Cocconut oil		& \num{8} \\[\spbtwrows]
		Spices			& \num{3}	& Spices			& \num{2} \\[\spbtwrows]
		Microalga		& \num{4}	& Microalga			& \num{4} \\[\spbtwrows]
		Tot				& \num{100}	& Tot				& \num{100} \\
	\bottomrule
\end{tabular}%
	\captionof{table}{Subdivision of ingredients for \species{Spirulina} 1A and 1B tasting burgers\label{subtab:burgers_ingredients-spirulina}}%
	\end{minipage}%
\hspace*{0.05\textwidth}%
	\begin{minipage}[b]{0.4162\textwidth}%
	\centering%
		\includegraphics[width=0.865\textwidth]{../plots/mpl/forms/jot_forms/jot_forms-outs/burger_1-trim}%
	\caption{Sensory analysis results for \species{Spirulina} 1A and 1B tasting burgers\label{subfig:sens_analysis-spirulina}}%
	\end{minipage}%
\end{figure}

\species{C.~vulgaris}

For the tasting of the \species{C.~vulgaris} burgers, divided like in \tab{}~\ref{subtab:burgers_ingredients-chlorella}, as we can see from the chart \figref{subfig:sens_analysis-chlorella}, in this case the preferred burger is the type 2A, again with \num{75}\% preference.

\begin{figure}[H]
\scriptsize
\centering
	\begin{minipage}[b]{0.475\textwidth}%
	\centering%
		\begin{tabular}{lclc}
	\toprule
	\belowrulesepcolor{colchlo}
	\rowcolor{colchlo}
		\multicolumn{4}{c}{\textbf{\species{C.~vulgaris} 4\%}} \\[\spheader]
	\rowcolor{colchlo}
		\multicolumn{2}{c}{\textbf{2A}} & \multicolumn{2}{c}{\textbf{2B}} \\[\spheader]
	\rowcolor{colchlo}
		\textbf{Ingredient} & \textbf{grams} & \textbf{Ingredient} & \textbf{grams} \\
	\aboverulesepcolor{colchlo}
	\midrule
		Chickpeas	& \num{51}	& Lupines + lentils & \num{56} \\[\spbtwrows]
		Oat flakes	& \num{20}	& Cous cous			& \num{10} \\[\spbtwrows]
		Carrots		& \num{15}	& Oat flour			& \num{5} \\[\spbtwrows]
					&			& Carrots			& \num{15} \\[\spbtwrows]
		Xanthan gum	& \num{0}	& Xanthan gum		& \num{0.5} \\[\spbtwrows]
		Coconut oil	& \num{8}	& Coconut oil		& \num{8} \\[\spbtwrows]
		Spices		& \num{2}	& Spices			& \num{2} \\[\spbtwrows]
		Microalga	& \num{4}	& Microalga			& \num{4} \\[\spbtwrows]
		Tot			& \num{100}	& Tot				& \num{100} \\
	\bottomrule
\end{tabular}%
	\captionof{table}{Subdivision of ingredients for \species{C.~vulgaris} 2A and 2B tasting burgers\label{subtab:burgers_ingredients-chlorella}}%
	\end{minipage}%
\hspace*{0.05\textwidth}%
	\begin{minipage}[b]{0.4162\textwidth}%
	\centering%
		\includegraphics[width=0.865\textwidth]{../plots/mpl/forms/jot_forms/jot_forms-outs/burger_2-trim}%
	\caption{Sensory analysis results for \species{C.~vulgaris} 2A and 2B tasting burgers\label{subfig:sens_analysis-chlorella}}%
	\end{minipage}%
\end{figure}

\species{P.~palmata}

Finally, for the \species{P.~palmata} burgers, divided as shown in the \tab{}~\ref{subtab:burgers_ingredients-palmaria} in this case the preferred type among the tasters was 3A with \num{65}\% preference \figref{subfig:sens_analysis-palmaria}.

\begin{figure}[H]
\scriptsize
\centering
	\begin{minipage}[b]{0.475\textwidth}%
	\centering%
		\begin{tabular}{lclc}
	\toprule
	\belowrulesepcolor{colpalm}
	\rowcolor{colpalm}
		\multicolumn{4}{c}{\textbf{\species{P.~palmata} \num{3}\%}} \\[\spheader]
	\rowcolor{colpalm}
		\multicolumn{2}{c}{\textbf{3A}} & \multicolumn{2}{c}{\textbf{3B}} \\[\spheader]
	\rowcolor{colpalm}
		\textbf{Ingredient} & \textbf{grams} & \textbf{Ingredient} & \textbf{grams} \\
	\aboverulesepcolor{colpalm}
	\midrule
		Red beans	& \num{54}	& Lentils 			& \num{60} \\[\spbtwrows]
		Cous cous	& \num{7}	& Potato			& \num{5} \\[\spbtwrows]
		Oat flour	& \num{8}	& Quinoa			& \num{5} \\[\spbtwrows]
		Beetroots	& \num{8}	& Beetroots			& \num{7} \\[\spbtwrows]
		Red onion	& \num{7.5}	& Carrots			& \num{7} \\[\spbtwrows]
		Xanthan gum	& \num{0.5}	& Xanthan gum		& \num{0.5} \\[\spbtwrows]
		Coconut oil	& \num{9}	& Coconut oil		& \num{8} \\[\spbtwrows]
		Spices		& \num{2}	& Spices			& \num{5} \\[\spbtwrows]
		Microalga	& \num{3}	& Microalga			& \num{3} \\[\spbtwrows]
		Tot			& \num{100}	& Tot				& \num{100} \\
	\bottomrule
\end{tabular}%
	\captionof{table}{Subdivision of ingredients for \species{P.~palmata} 3A and 3B tasting burgers\label{subtab:burgers_ingredients-palmaria}}%
	\end{minipage}%
\hspace*{0.05\textwidth}%
	\begin{minipage}[b]{0.4162\textwidth}%
	\centering%
		\includegraphics[width=0.865\textwidth]{../plots/mpl/forms/jot_forms/jot_forms-outs/burger_3-trim}%
	\caption{Sensory analysis results for \species{P.~palmata} 3A and 3B tasting burgers\label{subfig:sens_analysis-palmaria}}%
	\end{minipage}%
\end{figure}

\subsubsection{Last sensory analysis}
Regarding the final sensory analyses, we obtained information about the subjects diet, noting that they predominantly followed an omnivorous diet, with only \num{5} out of \num{36} being vegetarians and none being vegans \figref{fig:diet-pieplot}.

\begin{figure}[H]
\centering
\includegraphics[width=0.5\linewidth]{../plots/mpl/forms/google_forms/google_forms-outs/diet.pdf}
\caption{Tasters diet classification}
\label{fig:diet-pieplot}
\end{figure}


Regarding the algae species, we see that for \species{P.~palmata} the preferred type of burger was 1A with \num{1.5}\% algae 
\figref{subfig:burger_prefs-palmata}. Concerning the question about the “intention to buy”, the response about this burger was mostly positive \figref{subfig:buy_survey-palmata}.

For \species{Spirulina}, the preferred formulation among the tasters was 2A with the lowest algae percentage of \num{4}\% \figref{subfig:burger_prefs-spirulina}. About the response of purpose to purchase the feedback for this burger was relative mixed \figref{subfig:buy_survey-spirulina}.

Finally, for the \species{C.~vulgaris} burgers, the preference was for code 3C, with the highest percentage of \num{12}\% \figref{subfig:burger_prefs-chlorella}. Regarding the “intention to buy” question for this formulation was also mainly positive \figref{subfig:buy_survey-chlorella}.


\begin{figure}[H]
\centering
% Subfigure 1
	\subcaptionbox%
	{\species{P.~palmata}\label{subfig:burger_prefs-palmata}}%
		{\includegraphics[width=\tresubfigwth]{../plots/mpl/forms/google_forms/google_forms-outs/sample_1-prefer.pdf}}%
\hspace*{\trehbtwsfig}%
% Subfigure 2
	\subcaptionbox%
	{\species{Spirulina}\label{subfig:burger_prefs-spirulina}}%
		{\includegraphics[width=\tresubfigwth]{../plots/mpl/forms/google_forms/google_forms-outs/sample_2-prefer.pdf}}%
\hspace*{\trehbtwsfig}%
% Subfigure 3
	\subcaptionbox%
	{\species{C.~vulgaris}\label{subfig:burger_prefs-chlorella}}%
		{\includegraphics[width=\tresubfigwth]{../plots/mpl/forms/google_forms/google_forms-outs/sample_3-prefer.pdf}}%
\caption{Burger preferences}
\label{fig:burger_prefs}
\end{figure}

\begin{figure}[H]
\centering
% Subfigure 1
	\subcaptionbox%
	{\species{P.~palmata}\label{subfig:buy_survey-palmata}}%
		{\includegraphics[width=\tresubfigwth]{../plots/mpl/forms/google_forms/google_forms-outs/sample_1-buy.pdf}}%
\hspace*{\trehbtwsfig}%
% Subfigure 2
	\subcaptionbox%
	{\species{Spirulina}\label{subfig:buy_survey-spirulina}}%
		{\includegraphics[width=\tresubfigwth]{../plots/mpl/forms/google_forms/google_forms-outs/sample_2-buy.pdf}}%
\hspace*{\trehbtwsfig}%
% Subfigure 3
	\subcaptionbox%
	{\species{C.~vulgaris}\label{subfig:buy_survey-chlorella}}%
		{\includegraphics[width=\tresubfigwth]{../plots/mpl/forms/google_forms/google_forms-outs/sample_3-buy.pdf}}%
\caption{Results of the “intention to buy” survey}
\label{fig:buy_survey}
\end{figure}

If we consider all three algae species and their respective percentages, we see that the burgers preferred by the public are those made with \species{P.~palmata} and \species{C.~vulgaris}, specifically the 1A formulation with \num{1.5}\%, and the 3C formulation with \num{12}\% \figref{fig:last_sensory_barplot}.

\begin{figure}[H]
\centering
\includegraphics[width=0.8\linewidth]{../plots/mpl/forms/the_one/the_one-outs/the_one.pdf}
\caption{Last sensory analysis preferences burger}
\label{fig:last_sensory_barplot}
\end{figure}


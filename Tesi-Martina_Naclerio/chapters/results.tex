% !TeX encoding = UTF-8

\section{Results}
\subsection{ANOVA}
\paragraph{\species{Spirulina}}
\gls{anova} detected significant difference among different composition of \species{Spirulina} burgers for 

\begin{table}[H]
	\caption{ANOVA results of nutritional content for \species{Spirulina} burgers}
	\label{tab:anova-spirulina}
	\centering
	\scriptsize
	\begin{tabular}{l@{\hspace{1ex}}ccccccccccccc}
	\toprule
		 &  & \multicolumn{3}{c}{\textbf{Ash content}} & \multicolumn{3}{c}{\textbf{Moisture}} & \multicolumn{3}{c}{\textbf{Lipid}} & \multicolumn{3}{c}{\textbf{Protein}} \\[\spheader]
		\textbf{Source} & \textbf{d.f.} & \textbf{MS} & \textbf{F} & \textbf{P} & \textbf{MS} & \textbf{F} & \textbf{P} & \textbf{MS} & \textbf{F} & \textbf{P} & \textbf{MS} & \textbf{F} & \textbf{P} \\
	\midrule
		\textbf{Composition \%} & \num{3} & \num{4.38} & \num{38.17} & \num{0.002} & \num{48.54} & \num{45,69} & \num{0,001} & \num{1.39} & \num{0.61} & \num{0.643} & \num{9.73} & \num{28.41} & \num{0.004} \\[\spbtwrowsA]
		\textbf{Transformation} &  & \multicolumn{3}{c}{none} & \multicolumn{3}{c}{none}& \multicolumn{3}{c}{none}& \multicolumn{3}{c}{none} \\[\spbtwrowsA]
		\textbf{Shapiro-Wilk test} &  & \multicolumn{3}{c}{$ W = \num{0,977}^{\text{NS}} $} & \multicolumn{3}{c}{$ W = \num{0,972}^{\text{NS}} $} & \multicolumn{3}{c}{$ W = \num{0,872}^{\text{NS}} $} & \multicolumn{3}{c}{$ W = \num{0,974}^{\text{NS}} $} \\[\spbtwrowsA]
		\textbf{Cochran's \emph{C}-test} &  & \multicolumn{3}{c}{$ C = \num{0,747}^{\text{NS}} $} & \multicolumn{3}{c}{$ C = \num{0,735}^{\text{NS}} $} & \multicolumn{3}{c}{$ C = \num{0,394}^{\text{NS}} $} & \multicolumn{3}{c}{$ C = \num{0,646}^{\text{NS}} $} \\[\spbtwrowsA]
		\textbf{Pairwise test} &  & \multicolumn{3}{c}{Ctrl = \num{4}\% < \num{6}\% = \num{8}\%} & \multicolumn{3}{c}{Ctrl > \num{4}\% = \num{6}\% = \num{8}\%}& \multicolumn{3}{c}{-}& \multicolumn{3}{c}{Ctrl = \num{4}\% = \num{6}\% < \num{8}\%} \\
	\bottomrule
\end{tabular}
\end{table}

all the investigated variables, except for lipid contents (\tab{}~\ref{tab:anova-spirulina}). For the ash content, pairwise tests indicated that the control burger did not differ significantly from the burger containing 4\%, both having a significant lower ash content with respect to burgers with 6\% and 8\%, which in turn were comparable (Tab. 5, Fig. 12).
For moisture content, the control was significantly higher compared to 4\%, 6\%, and 8\%, which do not differ significantly from each other (Tab. 5, Fig. 13). No significant differences among the different groups were detected for lipid content (Tab. 5, Fig. 14). Finally, for the protein content, there were no significant difference between the control and the burgers with 4\% and 6\%, all of them containing a significantly higher amount of protein compared to the 8\% burger (\tab{}~\ref{tab:anova-spirulina}, Fig. 15).

% !TeX encoding = UTF-8

\begin{figure}[H]
\centering
% Subfigure 1
	\subcaptionbox%
	{Ashes content percentage\label{subfig:spirulina_ashes}}%
		{\includegraphics[width=\subfigwth]{../plots/mpl/barplots_anova/barplots/barplots-outs/spirulina_ashes.pdf}}%
\hspace*{\hbtwsfig}%
% Subfigure 2
	\subcaptionbox%
	{Moisture content percentage\label{subfig:spirulina_moisture}}%
		{\includegraphics[width=\subfigwth]{../plots/mpl/barplots_anova/barplots/barplots-outs/spirulina_moisture.pdf}}%
\\[2ex]
% Subfigure 3
	\subcaptionbox%
	{Lipids content percentage\label{subfig:spirulina_lipids}}%
		{\includegraphics[width=\subfigwth]{../plots/mpl/barplots_anova/barplots/barplots-outs/spirulina_lipids.pdf}}%
\hspace*{\hbtwsfig}%
% Subfigure 4
	\subcaptionbox%
	{Proteins content percentage\label{subfig:spirulina_proteins}}%
		{\includegraphics[width=\subfigwth]{../plots/mpl/barplots_anova/barplots/barplots-outs/spirulina_proteins.pdf}}%
\caption%
{ANOVA results fot \species{Spirulina}}
\label{fig:spirulinabla}
\end{figure}


\paragraph{\species{C.~vulgaris}}
For \species{C.~vulgaris} burgers, \gls{anova} detected significant differences for ash and moisture content, 

\begin{table}[H]
	\caption{ANOVA results of nutritional content for \species{C.~vulgaris} burgers}
	\label{tab:anova-chlorella}
	\centering
	\scriptsize
	\begin{tabular}{l@{\hspace{1ex}}ccc@{\hspace{1ex}}ccc@{\hspace{1ex}}ccc@{\hspace{1ex}}ccc@{\hspace{1ex}}c}
	\toprule
		 &  & \multicolumn{3}{c}{\textbf{Ash content}} & \multicolumn{3}{c}{\textbf{Moisture}} & \multicolumn{3}{c}{\textbf{Lipid}} & \multicolumn{3}{c}{\textbf{Protein}} \\[\spheader]
		\textbf{Source} & \textbf{d.f.} & \textbf{MS} & \textbf{F} & \textbf{P} & \textbf{MS} & \textbf{F} & \textbf{P} & \textbf{MS} & \textbf{F} & \textbf{P} & \textbf{MS} & \textbf{F} & \textbf{P} \\
	\midrule
		\textbf{Composition \%} & \num{3} & \num{0,32} & \num{11.45} & \num{0.02} & \num{87.45} & \num{254,4} & \num[output-exponent-marker = e]{5.09e-05} & \num{3,19} & \num{3,57} & \num{0.126} & \num{0.45} & \num{5,46} & \num{0.067} \\[\spbtwrowsA]
		\textbf{Transformation} &  & \multicolumn{3}{c}{none} & \multicolumn{3}{c}{none}& \multicolumn{3}{c}{none}& \multicolumn{3}{c}{none} \\[\spbtwrowsA]
		\textbf{Shapiro-Wilk test} &  & \multicolumn{3}{c}{$ W = \num{0,906}^{\text{NS}} $} & \multicolumn{3}{c}{$ W = \num{0,967}^{\text{NS}} $} & \multicolumn{3}{c}{$ W = \num{0,989}^{\text{NS}} $} & \multicolumn{3}{c}{$ W = \num{0,973}^{\text{NS}} $} \\[\spbtwrowsA]
		\textbf{Cochran's \emph{C}-test} &  & \multicolumn{3}{c}{$ W = \num{0,818}^{\text{NS}} $} & \multicolumn{3}{c}{$ W = \num{0,721}^{\text{NS}} $} & \multicolumn{3}{c}{$ W = \num{0,617}^{\text{NS}} $} & \multicolumn{3}{c}{$ W = \num{0,953}^{\text{NS}} $} \\[\spbtwrowsA]
		\textbf{Pairwise test} &  & \multicolumn{3}{c}{Ctrl = \num{4}\% < \num{8}\% = \num{12}\%} & \multicolumn{3}{c}{Ctrl > \num{4}\% = \num{6}\% = \num{8}\%}& \multicolumn{3}{c}{-}& \multicolumn{3}{c}{Ctrl = \num{4}\% = \num{8}\% < \num{12}\%} \\
	\bottomrule
\end{tabular}
\end{table}

but not for lipid and protein contents (Tab. 6 ). For both ash and moisture, control burgers had significantly higher values with respect to burgers containing \species{C.~vulgaris} burgers at all the analyzed \%, which instead did not differ (Tab. 6, Figg. 16-17). No significant difference among groups were detected for lipids and proteins (Tab. 6). However, for protein content, the burgers containing \species{C.~vulgaris} appeared to have a higher protein \% if compared with control burgers (Fig .19).

% !TeX encoding = UTF-8

\begin{figure}[H]
\centering
% Subfigure 1
	\subcaptionbox%
	{Ashes bla\label{subfig:chlorella_ashes}}%
		{\includegraphics[width=0.5\textwidth]{../plots/mpl/barplots_anova/barplots/barplots-outs/chlorella_ashes.pdf}}%
\hfill
% Subfigure 2
	\subcaptionbox%
	{Moisture bla\label{subfig:chlorella_moisture}}%
		{\includegraphics[width=0.5\textwidth]{../plots/mpl/barplots_anova/barplots/barplots-outs/chlorella_moisture.pdf}}%
\\[2ex]
% Subfigure 3
	\subcaptionbox%
	{Lipids bla\label{subfig:chlorella_lipids}}%
		{\includegraphics[width=0.5\textwidth]{../plots/mpl/barplots_anova/barplots/barplots-outs/chlorella_lipids.pdf}}%
\hfill
% Subfigure 4
	\subcaptionbox%
	{Proteins bla\label{subfig:chlorella_proteins}}%
		{\includegraphics[width=0.5\textwidth]{../plots/mpl/barplots_anova/barplots/barplots-outs/chlorella_proteins.pdf}}%
\caption%
{Chlorella bla}
\label{fig:chlorellabla}
\end{figure}


\paragraph{\species{P.~palmata}}
Also in this case, \gls{anova} detected significant differences among different compositions only

\begin{table}[H]
	\caption{ANOVA results of nutritional content for \species{P.~palmata} burgers}
	\label{tab:anova-palmaria}
	\centering
	\scriptsize
	\begin{tabular}{l@{\hspace{1ex}}ccccccccccccc}
	\toprule
		 &  & \multicolumn{3}{c}{\textbf{Ash content}} & \multicolumn{3}{c}{\textbf{Moisture}} & \multicolumn{3}{c}{\textbf{Lipid}} & \multicolumn{3}{c}{\textbf{Protein}} \\[\spheader]
		\textbf{Source} & \textbf{d.f.} & \textbf{MS} & \textbf{F} & \textbf{P} & \textbf{MS} & \textbf{F} & \textbf{P} & \textbf{MS} & \textbf{F} & \textbf{P} & \textbf{MS} & \textbf{F} & \textbf{P} \\
	\midrule
		\textbf{Composition \%} & \num{2} & \num{0,84} & \num{42.35} & \num{0.006} & \num{61.54} & \num{24,21} & \num{0,014} & \num{10.83} & \num{8.09} & \num{0.062} & \num{0.01} & \num{9.12} & \num{0.053} \\[\spbtwrowsA]
		\textbf{Transformation} &  & \multicolumn{3}{c}{none} & \multicolumn{3}{c}{none}& \multicolumn{3}{c}{none}& \multicolumn{3}{c}{$ \log (x + 1) $} \\[\spbtwrowsA]
		\textbf{Shapiro-Wilk test} &  & \multicolumn{3}{c}{$ W = \num{0,992}^{\text{NS}} $} & \multicolumn{3}{c}{$ W = \num{0,908}^{\text{NS}} $} & \multicolumn{3}{c}{$ W = \num{0,941}^{\text{NS}} $} & \multicolumn{3}{c}{$ W = \num{0,924}^{\text{NS}} $} \\[\spbtwrowsA]
		\textbf{Cochran's \emph{C}-test} &  & \multicolumn{3}{c}{$ C = \num{0,826}^{\text{NS}} $} & \multicolumn{3}{c}{$ C = \num{0,590}^{\text{NS}} $} & \multicolumn{3}{c}{$ C = \num{0,963}^{\text{NS}} $} & \multicolumn{3}{c}{$ C = \num{0,953}^{\text{NS}} $} \\[\spbtwrowsA]
		\textbf{Pairwise test} &  & \multicolumn{3}{c}{Ctrl = \num{1.5}\% < 3\%} & \multicolumn{3}{c}{Ctrl > \num{1.5}\% = \num{3}\%}& \multicolumn{3}{c}{-}& \multicolumn{3}{c}{-} \\
	\bottomrule
\end{tabular}
\end{table}

for ash and moisture content (Tab. 7). Pairwise tests indicated that the control group and the 1.5\% group did not differ significantly in their ash content, while the burgers with 3\% \species{P.~palmata} had a significantly higher ash content compared to the others (Fig. 20). Conversely, for moisture content, the control group had a higher value of mosture compared to the other two groups, which are not significantly different from each other (Fig. 21). No significant differences were detected for lipid (Fig. 22) and protein (Fig. 23) content between the control burger and those with 1.5\% and 3\% content.

% !TeX encoding = UTF-8

\begin{figure}[H]
\centering
% Subfigure 1
	\subcaptionbox%
	{Ashes bla\label{subfig:palmaria_ashes}}%
		{\includegraphics[width=0.5\textwidth]{../plots/mpl/barplots_anova/barplots/barplots-outs/palmaria_ashes.pdf}}%
\hfill
% Subfigure 2
	\subcaptionbox%
	{Moisture bla\label{subfig:palmaria_moisture}}%
		{\includegraphics[width=0.5\textwidth]{../plots/mpl/barplots_anova/barplots/barplots-outs/palmaria_moisture.pdf}}%
\\[2ex]
% Subfigure 3
	\subcaptionbox%
	{Lipids bla\label{subfig:palmaria_lipids}}%
		{\includegraphics[width=0.5\textwidth]{../plots/mpl/barplots_anova/barplots/barplots-outs/palmaria_lipids.pdf}}%
\hfill
% Subfigure 4
	\subcaptionbox%
	{Proteins bla\label{subfig:palmaria_proteins}}%
		{\includegraphics[width=0.5\textwidth]{../plots/mpl/barplots_anova/barplots/barplots-outs/palmaria_proteins.pdf}}%
\caption%
{Palmaria bla}
\label{fig:palmariabla}
\end{figure}


GRAFICI

\subsection{PERMANOVA}
PERMANOVA detected significance differences among treatments, corresponding to different burgers containing varying \% of the three algal species. The analysis reveals significant differences in the overall nutritional values among the various formulations examined. Specifically, comparisons between different formulations indicated that main differences were among the three species considered. Conversely, variations in percentages within the same species do not significantly impact the overall nutritional values.
The CAP biplot (Fig. 24) provided a clear visualization of these differences, showing how the various formulations cluster based on the algal species, with relatively small variations within easch species. All variables contributed to these pattern of difference, with Spirulina burgers characterized by high nutitrional values with respect to burgers containing the other two species.

\subsection{Sensory Analysis}
\subsubsection{First Sensory Analysis}
For the two types of \species{Spirulina} burgers offered for tasting, divided as follows (\tab{}~\ref{tab:burgers_ingredients-spirulina}):
\begin{table}[H]
	\caption{Subdivision of ingredients for \species{Spirulina} A and B tasting burgers}
	\label{tab:burgers_ingredients-spirulina}
	\centering
	\begin{tabular}{cccc}
	\toprule
	\rowcolor{colspir}
		\multicolumn{4}{c}{\textbf{\species{A.~platensis} 4\%}} \\[\spheader]
		\multicolumn{2}{c}{\textbf{1A}} & \multicolumn{2}{c}{\textbf{1B}} \\[\spheader]
		\textbf{Ingredient} & \textbf{grams} & \textbf{Ingredient} & \textbf{grams} \\
	\midrule
		Lupines			& \num{48}	& Lupines + lentils & \num{56} \\[\spbtwrows]
		Oat flour		& \num{15}	& Oat flour			& \num{10} \\[\spbtwrows]
		Carrots			& \num{10}	& Oat flakes		& \num{5} \\[\spbtwrows]
		Spinach			& \num{10}	& Courgettes		& \num{15} \\[\spbtwrows]
						& 			& Onion				& \num{7} \\[\spbtwrows]
		Xanthan gum		& \num{2}	& Xanthan gum		& \num{2} \\[\spbtwrows]
		Cocconut oil	& \num{8}	& Cocconut oil		& \num{8} \\[\spbtwrows]
		Spices			& \num{3}	& Spices			& \num{2} \\[\spbtwrows]
		Microalga		& \num{4}	& Microalga			& \num{4} \\[\spbtwrows]
		Tot				& \num{100}	& Tot				& \num{100} \\
	\bottomrule
\end{tabular}
\end{table}

it can be noted that Burger B is preferred by the tasters with a 75\% preference compared to Burger A (Fig. 25).

PIEPLOT

As for the tasting of the \species{C.~vulgaris} burgers, divided as follows (\tab{}~\ref{tab:burgers_ingredients-chlorella}):
\begin{table}[H]
	\caption{Subdivision of ingredients for \species{C.~vulgaris} A and B tasting burgers}
	\label{tab:burgers_ingredients-chlorella}
	\centering
	\begin{tabular}{lclc}
	\toprule
	\belowrulesepcolor{colchlo}
	\rowcolor{colchlo}
		\multicolumn{4}{c}{\textbf{\species{C.~vulgaris} 4\%}} \\[\spheader]
	\rowcolor{colchlo}
		\multicolumn{2}{c}{\textbf{2A}} & \multicolumn{2}{c}{\textbf{2B}} \\[\spheader]
	\rowcolor{colchlo}
		\textbf{Ingredient} & \textbf{grams} & \textbf{Ingredient} & \textbf{grams} \\
	\aboverulesepcolor{colchlo}
	\midrule
		Chickpeas	& \num{51}	& Lupines + lentils & \num{56} \\[\spbtwrows]
		Oat flakes	& \num{20}	& Cous cous			& \num{10} \\[\spbtwrows]
		Carrots		& \num{15}	& Oat flour			& \num{5} \\[\spbtwrows]
					&			& Carrots			& \num{15} \\[\spbtwrows]
		Xanthan gum	& \num{0}	& Xanthan gum		& \num{0.5} \\[\spbtwrows]
		Coconut oil	& \num{8}	& Coconut oil		& \num{8} \\[\spbtwrows]
		Spices		& \num{2}	& Spices			& \num{2} \\[\spbtwrows]
		Microalga	& \num{4}	& Microalga			& \num{4} \\[\spbtwrows]
		Tot			& \num{100}	& Tot				& \num{100} \\
	\bottomrule
\end{tabular}
\end{table}

as we can see from the chart (Fig. 26), in this case, the preferred burger is of type A, always with 75\% preference.

PIEPLOT

Finally, for the \species{P.~palmata} burgers divided as shown in the table (\tab{}~\ref{tab:burgers_ingredients-palmaria})
\begin{table}[H]
	\caption{Subdivision of ingredients for \species{P.~palmata} A and B tasting burgers}
	\label{tab:burgers_ingredients-palmaria}
	\centering
	\begin{tabular}{lclc}
	\toprule
	\belowrulesepcolor{colpalm}
	\rowcolor{colpalm}
		\multicolumn{4}{c}{\textbf{\species{P.~palmata} \num{3}\%}} \\[\spheader]
	\rowcolor{colpalm}
		\multicolumn{2}{c}{\textbf{3A}} & \multicolumn{2}{c}{\textbf{3B}} \\[\spheader]
	\rowcolor{colpalm}
		\textbf{Ingredient} & \textbf{grams} & \textbf{Ingredient} & \textbf{grams} \\
	\aboverulesepcolor{colpalm}
	\midrule
		Red beans	& \num{54}	& Lentils 			& \num{60} \\[\spbtwrows]
		Cous cous	& \num{7}	& Potato			& \num{5} \\[\spbtwrows]
		Oat flour	& \num{8}	& Quinoa			& \num{5} \\[\spbtwrows]
		Beetroots	& \num{8}	& Beetroots			& \num{7} \\[\spbtwrows]
		Red onion	& \num{7.5}	& Carrots			& \num{7} \\[\spbtwrows]
		Xanthan gum	& \num{0.5}	& Xanthan gum		& \num{0.5} \\[\spbtwrows]
		Coconut oil	& \num{9}	& Coconut oil		& \num{8} \\[\spbtwrows]
		Spices		& \num{2}	& Spices			& \num{5} \\[\spbtwrows]
		Microalga	& \num{3}	& Microalga			& \num{3} \\[\spbtwrows]
		Tot			& \num{100}	& Tot				& \num{100} \\
	\bottomrule
\end{tabular}
\end{table}

in this case as well, the preferred type among the tasters was type A with 65\% preference (Fig. 27).

\subsubsection{Last sensory analysis}
Regarding the final sensory analyses, if we analyze the individual species, we see that for \species{P.~palmata}, the preferred type of burger was 1A with 1.5\% algae (Fig. 28).

PIEPLOT

For \species{Spirulina}, the preferred formulation among the tasters was 2A with the lowest algae percentage of 4\% (Fig. 29).

PIEPLOT

Finally, for the \species{C.~vulgaris} burgers, the preference was for code 3C, with the highest percentage of 12\% (Fig. 30).

PIEPLOT

If we consider all three algae species and their respective percentages, we see that the burgers preferred by the public are those made with P. palmata, specifically the 1A formulation with 1.5\% (Fig. 31).

HISTOGRAM

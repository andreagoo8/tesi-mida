\section{Discussion}
The burger that differed the most from the control, providing improvements of the nutritional features is the one based on \species{Spirulina}, particularly the one with the highest percentage, which is 8\%. Examining the ash content, an important parameter for assessing the nutritional value of the food in terms of essential micronutrients~\parencite{harris_Ash_2017}, the burger with the addition of \species{Spirulina}, especially at 8\%, showed a higher ash content, leading to a significant enhancement of this aspect. Ash represents the residual minerals after the combustion of a food, and a higher quantity of these indicates a higher content of essential minerals such as calcium, iron, and magnesium.
Conversely, the moisture content is lower for all three \species{Spirulina} formulations. A high value of this parameter always causes a deterioration in the overall quality of the food~\parencite{roudaut_Moisture_2010}, and for this reason, the implementation of algae brings an improvement in this regard, contributing to greater stability and shelf-life of the product.
The lipid content does not vary significantly between formulations and generally remains within the recommended ranges for a balanced diet. Foods with a total fat content of less than \qty{3}{\gram} per \qty{100}{\gram} and \qty{20}{\gram} per \qty{100}{\gram} are considered to be low and moderate fat content respectively (WHO, 2003). \species{Spirulina} burgers therefore remain within the ideal ranges, contributing to maintaining a balanced diet without excessive fat intake.
The protein content, essential for health and crucial bodily functions, from tissue building to metabolic regulation~\parencite{chang_Protein_2017} was significantly improved by the addition of \species{Spirulina}, especially at higher concentrations. This makes \species{Spirulina} an excellent alternative protein source, particularly useful in vegetarian and vegan diets.
Regarding the other two types of algae, they generally provide improvements compared to the control, except for the protein content of \species{P.~palmata}, which decreased with its addition.
This may be due to various factors, including the possible presence of compounds that inhibit the absorption or digestion of proteins, or the fact that the protein content of \species{P.~palmata} is inherently lower compared to \species{Spirulina}, with 30\% versus 70\% of dry weight~\parencite{fleurence_Seaweed_2004}.
Another hypothesis concerns the specific composition of the parts of the algae used in the formulation of the product. It is known that in algae, as in many other plants, proteins are not distributed uniformly throughout all parts of the plant. Specifically, the fronds of \species{P.~palmata}, which are the main parts of the algae exposed to sunlight, tend to have a higher concentration of proteins compared to other parts such as stems or roots~\parencite{galland-irmouli_Nutritional_1999}.
This is because the fronds perform photosynthesis and require more enzymes and structural proteins to support this metabolic activity. In the production of the burgers, it is possible that the process of reducing them to flakes included a significant proportion of parts of the algae that are less rich in proteins, such as stems or other supporting structures. This may have diluted the overall protein content of the algae used. To improve the protein content of burgers based on \species{P.~palmata}, it may be useful to examine and more carefully select the parts of the algae used. The predominant use of fronds, which are richer in proteins, could increase the protein content of the final product. Alternatively, it might be possible to improve the flake production process to ensure a more uniform and high-protein composition.
Considering the whole set of nutritional factors, the algae that significantly enhanced the nutritional composition is \species{Spirulina}, confirming this species as the best choice as food additive in the conducted experiment.
However, combining these outcomes with the results of sensory analysis, we found that the public’s taste preferences did not align with the nutritionally best burger. Instead, the sampled population expressed a preference for the \species{P.~palmata} burger at 1.5\% concentration. Nevertheless, these results cannot be considered negative, as the 8\% Spirulina burger still received positive feedback, with a 71.79\% likelihood of purchase intention.
In the future, it is planned to make organoleptic improvements to the \species{Spirulina} burgers. To facilitate greater acceptance and dissemination of these highly nutritious foods, it is essential to improve their sensory characteristics, particularly the taste. Enhancing this aspect could involve modulating the flavor to make it more pleasant and appealing without compromising the nutritional benefits.
Overall, the sensory analysis results are extremely positive in terms of public acceptance. Despite the predominance of individuals with an omnivorous diet, they did not show skepticism towards these alternative foods. This suggests a growing openness to new protein sources and innovative foods, regardless of pre-existing dietary habits. However, it may be useful to expand the sensory analysis to a larger group of people from different parts of the world to understand how preferences vary among different subjects in relation to cultural diversity. This would allow for obtaining more representative data and identifying any differences in preferences due to cultural preferences.
Future studies are necessary to include further nutritional analyses, such as fiber content and antioxidant content. Importance of fiber in the human diet is widely recognized due to the numerous health benefits they offer. A generous intake of dietary fiber has been associated with reduced risk of developing several serious diseases, including coronary heart disease, stroke, hypertension, diabetes, obesity, and some gastrointestinal disorders. Fiber plays a crucial role in improving serum lipid concentrations, lowering blood pressure, and enhancing glycemic control in diabetics. Moreover, it promotes bowel regularity, supports weight loss, and appears to improve immune system function. Despite these benefits, most of the population consumes less than half of the recommended daily amount of fiber \zxriv{(28-36 g/day)}~\parencite{anderson_Health_2009}. Antioxidants play a crucial role in the human diet due to their ability to neutralize free radicals, unstable molecules that can cause cellular damage and contribute to the development of various chronic diseases. Among the major diseases antioxidants can help prevent are cardiovascular diseases, cancer, diabetes, and neurodegenerative diseases. Free radicals can arise from normal metabolic processes as well as external factors, and antioxidants, acting as scavengers, protect cells and tissues from oxidative damage. The importance of antioxidants in the diet is further underscored by their ability to improve food quality, prolong its preservation, and also have positive effects on the immune system~\parencite{zehiroglu_importance_2019}. Several algae, such as \species{Spirulina}, may have a high antioxidant activity when added to food and could represent an important source of these antioxidant molecules in human diet~\parencite{stunda-zujeva_Comparison_2023}. Additionally, it would be interesting to provide further insights into the nutritional analyses carried out in this study, exploring the difference in lipid content for saturated and unsaturated fatty acids separately, to obtain a more complete view of the properties of the foods being examined. Knowledge of the quantity of saturated and unsaturated fatty acids in the human diet is crucial for maintaining good health and preventing various diseases. We know that saturated fatty acids have been associated with an increased risk of cardiovascular diseases. Dietary recommendations suggest reducing the intake of saturated fats to less than 10\% of daily calories and increasing the intake of unsaturated fats, particularly omega-3 fatty acids, which are known to be abundant in algae~\parencite{lunn_health_2006}. 
For the analysis of ash content as well, further research is recommended on the specific mineral content, as these micronutrients are essential for the human diet. Macrominerals such as sodium, potassium, and chlorine are crucial for maintaining electrolyte balance and facilitating cellular signal transmission, movement, and transport of metabolites across cellular membranes. Together with trace elements, they participate in vital processes such as enzyme activation, modulation of genetic transcription, and the antioxidant capacity of certain metalloenzymes. Therefore, a thorough understanding of the types of minerals present in foods is essential for formulating precise dietary recommendations and preventing nutritional deficiencies that can compromise human health. For the analysis of ash content as well, further research is recommended on the specific mineral content, as these micronutrients are essential for the human diet. Macrominerals such as sodium, potassium, and chlorine are crucial for maintaining electrolyte balance and facilitating cellular signal transmission, movement, and transport of metabolites across cellular membranes. Together with trace elements, they participate in vital processes such as enzyme activation, modulation of genetic transcription, and the antioxidant capacity of certain metalloenzymes.
Therefore, a thorough understanding of the types of minerals present in foods is essential for formulating precise dietary recommendations and preventing nutritional deficiencies that can compromise human health.

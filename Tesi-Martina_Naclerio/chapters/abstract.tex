% !TeX encoding = UTF-8

\section*{Abstract}
\addcontentsline{toc}{section}{Abstract}
Algae are promising organisms for the production of sustainable products to be used as raw materials for food, animal feed, chemicals, biofuels, and much more. In this period of change and demographic growth, it will be necessary to focus on the potential of these organisms as a sustainable food resource, also considering the European environmental regulations and commitment towards an Ecological Transition and a new concept of Blue Economy.
Initiatives such as the “EU Green Deal”, “Farm to Fork” and the “EU Algae Strategy” are just some examples promoting the adoption of more sustainable and environmentally friendly dietary practices. % \zxriv{Andrea usare corsivo?}
From an overview of the characteristics of algae, their cultivation methods, and their multiple uses, the benefits they offer for both human health and the ecosystem will be evident. Through an experimental project, the integration of three types of algae into plant-based burgers was examined: \species{Spirulina}, \species{C.~vulgaris}, and \species{P.~palmata}. Analyzing the nutritional and sensory aspects of these food alternatives, it is noticeable that their implementation has led to an improvement in nutritional profile, especially in \species{Spirulina}-based burgers, with a positive degree of acceptance by the public for each proposed formulation, despite the predominantly omnivorous diet of the sample individuals. This gives hope that innovative food proposals, in addition to benefiting human health, can outline a path towards a more conscious and sustainable future and offer support against the increasingly inevitable battle against climate change.

\begin{otherlanguage}{italian}
\section*{Riassunto}
Le alghe sono organismi promettenti per la produzione di prodotti sostenibili da utilizzare come materie prime per alimenti, mangimi, prodotti chimici e biocarburanti. In questo periodo di cambiamenti e di crescita demografica bisognerà concentrarsi sulle potenzialità di questi organismi come risorsa alimentare sostenibile guardando anche alle normative europee ambientali e all'impegno per una Transizione Ecologica e un nuovo concetto di Blue Economy. Il “Green Deal”, il “Farm to Fork” e il progetto “EU Algae Strategy” sono solo alcune iniziative che promuovono l'adozione di pratiche alimentari più sostenibili e rispettose dell'ambiente. Attraverso una panoramica sulle caratteristiche delle alghe, i loro metodi di coltivazione e i molteplici utilizzi, saranno ben evidenti i benefici che queste offrono sia per la salute umana che per l'ecosistema. Attraverso un progetto sperimentale, è stata esaminata l'integrazione in burger a base vegetale di tre tipologie di alga: \species{Spirulina}, \species{C.~vulgaris} e \species{P.~palmata}. Andando ad analizzare gli aspetti nutrizionali e sensoriali di queste alternative alimentari, può notare come l’implementazione di queste abbia portato ad un miglioramento sotto il profilo nutrizionale, specialmente nei burger a base di \species{Spirulina}, con un positivo grado di accettazione da parte del pubblico per ogni formulazione proposta, nonostante la dieta del campione di individui fosse maggiormente onnivora. Questo fa sperare che innovative proposte alimentari, oltre ad apportare beneficio per la salute umana, possano delineare una via verso un futuro consapevole e più sostenibile e offrire un supporto contro una sempre più inevitabile battaglia ai cambiamenti climatici.
\end{otherlanguage}
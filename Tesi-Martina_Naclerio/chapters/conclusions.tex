\section{Conclusions}
In conclusion, we can confirm the importance of enriching the human diet with these alternative foods, which are much healthier compared to the currently most used protein sources by the global population. Moreover, emphasizing the importance not only from a human perspective but also, and especially, from an environmental and ecological standpoint that these solutions can bring. Algae emerge as a promising resource to replace both traditional plant and animal raw materials, significantly contributing to environmental sustainability and greenhouse gas emission mitigation. Agriculture, forestry, and other land uses account for up to \num{23}\% of total greenhouse gas emissions (GHG), making them major contributors to global warming~\parencite{gonzalez_Meat_2020}.
Due to their ability to grow rapidly without the need for intensive agricultural land, algae present an advantageous alternative to conventional crops.

In the food industry, algae can serve as sources of proteins, carbohydrates, and micronutrients, replacing ingredients derived from intensive crops like soybeans and maize. A 2014 study~\parencite{ullah_Algal_2014} found that algae can produce biomass annually that is 167 times greater than corn using the same amount of land. The use of algae reduces environmental impact associated with deforestation and intensive use of water and synthetic fertilizers linked to terrestrial crops. Algae are highly resource-efficient and can grow on non-arable land using non-potable water, such as seawater or brackish water, thereby reducing competition with traditional agriculture for water and land resources~\parencite{diaz_Developing_2023}.
Additionally, these organisms may improve environmental conditions in areas where they are cultured.

Studies indicate that algae cultivation can also improve water quality by absorbing excess nutrients and releasing oxygen during photosynthesis, thus mitigating eutrophication and enhancing aquatic ecosystem health~\parencite{espinosa-ramirez_Algae_2023}. Nutritionally, algae are rich in proteins, lipids, carbohydrates, essential fatty acids, vitamins, and minerals, making them ideal not only as dietary supplements but also as ingredients for meat alternatives. Integrating algae into the human diet can contribute to achieving the sustainable development goals of the United Nations' Agenda 2030. Algae cultivation requires no artificial fertilizers and boasts much higher productivity compared to terrestrial crops. Moreover, their cultivation does not compete with arable land use, a critical factor in a context of population growth and limited agricultural resources~\parencite{boukid_Algae_2022}. In summary, integrating algae into food and industrial production systems offers significant potential to improve environmental sustainability, reduce greenhouse gas emissions, and promote the bioeconomy. However, technological challenges and consumer acceptance hurdles remain to be overcome for their full-scale adoption in production.
\section{Introduction}
The following project provides an overview of the benefits and applications of algae in the Blue Economy. By examining this concept, along with the Ecological Transition and strategies implemented at the European level, I explored some of the diverse applications of algae, particularly focusing on their use in human nutrition. These organisms represent one of the most abundant and sustainable resources on Earth and play a crucial role in promoting the Blue Economy and ecological transition by offering a wide range of commercial opportunities and applications. Algae can be utilized to produce food, biofuels, cosmetics, agricultural fertilizers, and much more, thereby helping to reduce dependence on fossil resources and minimize the negative environmental impact of human activities.


\subsection{European Regulations}
The \gls{unfccc} is the primary international agreement on climate action adopted in Rio in 1992. When it was realized that collective action was necessary to protect people and the environment and to limit greenhouse gas emissions, the Kyoto Protocol was concluded to reduce these emissions. On November 4, 2016 the Paris Agreement entered into force, representing the renewed commitment of nations to combat and limit global warming, with the aim of creating a zero-climate-impact society by 2050~\parencite{paris_agreement}.

In addition, in 2015, the Agenda 2030 was born, a program of action comprising 17 \glspl{sdg} to be achieved in the environmental, economic, social, and institutional spheres, a goal that calls for global efforts aimed at ensuring a better present and future for our planet, committing to eliminating or finding solutions to the issues that concern us globally. This Agenda is based on five key concepts: people, prosperity, peace, partnership, and planet, and its objectives are all closely interconnected. The \glspl{sdg} are universal and as such must be achieved by all countries, which will need to define their own strategies in order to implement them~\parencite{agenda_2030}.

In 2019, three years later, the European Parliament declared a climate emergency and established a key strategy of the European Union, the Green Deal. It consists of a series of action plans aimed at making Europe the first continent to achieve climate neutrality by 2050 (in line with the Paris Agreement) and leading it towards a green transition by implementing economic, energy, industrial, and agricultural initiatives. In addition to climate neutrality, the Green Deal has two other main purposes: circular economy and environmental protection. This project includes legislations and initiatives to achieve the established objectives. The first climate-related legislation is the “Fit for 55\%” package, which aims to adapt the current European regulations to align with the goals of the Green Deal. Following this, achieving climate neutrality becomes a legal obligation for the EU, with a commitment to reduce net greenhouse gas emissions by 55\% by 2030 compared to 1990 levels. Furthermore, adaptation strategies to climate change are implemented to enhance understanding and effects, and to create appropriate solutions to address them, such as adapting civil protection through prevention, preparedness, response, and recovery.

There is also an industrial strategy, the objective of which is to support the industry in its role as an accelerator and driver of change, innovation, and growth, guiding us towards a green transformation. We must not overlook the action plan for the circular economy, a crucial step for economic growth, an increase in sustainable products, and better resource use, both for production and consumption. Additionally, it is essential to consider that achieving climate neutrality will not be equitable for all; for some member states, it will be more challenging, especially those dependent on fossil fuels or with high-carbon-intensive industries. This is why the EU has introduced a mechanism for a just transition, providing financial and technical support to these regions. Regarding greenhouse gas emissions, we know that 75\% of them are attributable to the production and use of energy. In this regard, the Green Deal proposes objectives for affordable and secure energy through the use of cleaner sources such as renewables, also revising and amending current legislation accordingly.

Further lines of action are: biodiversity, where the Green Deal aims to restore biodiversity in Europe by 2030 through ecosystem restoration actions, reducing pesticide use, expanding land and marine areas, and increasing funding for the implementation of these goals; chemicals, which are also approached from a sustainability perspective to preserve the environment and human health; forests and deforestation, which is linked to the biodiversity plan and the reduction of greenhouse gas emissions. This involves offering financial incentives and proposing the planting of new trees to address deforestation issues~\parencite{green_deal}.

In May 2020, the European Commission presented the “Farm to Fork” strategy as one of the key actions of the Green Deal, also aligning with the Sustainable Development Goals mentioned in the Agenda 2030, aiming to make the current food system more sustainable. The objectives of this initiative consider food security from various aspects, including the adoption of organic farming, a plan for food safety, the promotion of environmentally friendly agricultural practices that allow carbon sequestration in soils, sustainable aquaculture facilities, and proper labeling and marketing standards for products. All of this is aimed at guiding Europe towards a fair, healthy, and environmentally respectful food system~\parencite{farm_to_fork}.


\subsection{Ecological Transition and Blue Economy}
When discussing the Ecological Transition, we cannot overlook the \gls{ngeu} recovery plan, of which it is a key focus, linking with the perspectives of the Green Deal. This project aims to create a healthier and greener future for Europe through a wide range of investments, facilitated by the \gls{rrf} and the \gls{nrrp}.

\gls{ngeu} focuses on six key sectors, one of which is the Ecological Transition, providing momentum to the green transition by promoting renewable energies, sustainable mobility, enhancing waste recycling, and much more.

In more detail, Europe commits to:\footnote{\zxriv{ho messo in minuscolo le iniziali dei punti}}
\begin{itemize}
\item improve the quality of water in rivers and seas, reduce polluting waste and plastics, plant billions of trees, and repopulate the world with bees;
\item make agriculture more environmentally friendly so that our food is healthier;
\item create green spaces in our cities and utilize renewable energies more effectively~\parencite{next_gen_eu}.
\end{itemize}

The ecological transition is a process aimed at transforming production and consumption patterns to make the economy more sustainable and environmentally friendly. This involves adopting practices and technologies that reduce environmental impact, promoting energy efficiency, the use of renewable energies, and the adoption of sustainable natural resource management strategies. It is based on principles of equity, social justice, and environmental responsibility, aiming to reconcile economic development with environmental preservation and therefore plays a fundamental role in addressing current challenges related to it. Through the promotion of sustainable practices and policies, the ecological transition can contribute to mitigating climate change, reducing pollution, preserving biodiversity, and ensuring a better future for future generations. Additionally, it fosters the creation of new jobs in renewable energy, energy efficiency, and sustainable technology sectors, stimulating innovation and the green economy.

An example of an approach to the responsible use of natural sources is introduced by the strategy "Innovating for Sustainable Growth: A Bioeconomy for Europe" (a project connected to the Green Deal and Farm to Fork), initiated in 2012 and subsequently updated in 2018.

Our economy, based on fossil fuels, has reached its limits; therefore, we need to revolutionize the way we produce and consume, respecting our planet. The purpose of the bioeconomy is precisely this: it covers all sectors focused on biological resources and connects terrestrial and marine ecosystems with a view to sustainability. Furthermore, to build a zero-emission future, a bioeconomy aiming at the use of renewable resources for energy production is necessary~\parencite{eucommission_sustainable_2018}.

To achieve the goals of the Green Deal, increasing attention is being paid to the role that marine ecosystems and the blue economy can play and the opportunities they can offer as a source of resources. The Agenda 2030 recognizes that without healthy oceans, life on our planet is at risk, and without these resources, societies lose the ability to sustain themselves. This is why European thinking is shifting from the simple concept of the bioeconomy to the idea of a sustainable blue bioeconomy, which promotes the responsible management of oceans and marine resources to ensure economic benefits without compromising ecosystem health. This can contribute to carbon neutrality through the development of renewable energies produced at sea (offshore) and by making maritime transport and ports more environmentally friendly. Through better use of marine resources and the selection of alternative sources of food and feed, the blue economy can help alleviate the pressure on the climate and natural resources from food production~\parencite{sustainable_blue_economy_2021}.

In line with the "Farm to Fork" and projects for a blue bioeconomy, in November 2022, the European Commission announced the EU Algae Strategy, "Towards a strong and sustainable algae sector," highlighting the role of algae as an alternative protein source in the perspective of a sustainable food system~\parencite{blue_bioeconomy}.

To fully exploit the opportunities offered by the algae sector in the EU, it is necessary to promote the integration and development of markets related to their applications and consequently promote increased cultivation and production of them throughout the territory.

To this end, this strategy identifies 23 actions aimed at:
\begin{enumerate}
\item improving governance framework and regulations;
\item enhancing the business environment;
\item addressing knowledge gaps, research, technology, and innovation;
\item increasing social awareness and market acceptance of algae and algae-based products in the EU~\parencite{kuech_future_2023}.
\end{enumerate}

Algae indeed have significant exploitable potential useful for addressing the issues affecting our world. The aim is to create a high-level industry that plays a significant global role, both in terms of climate change, food security, and supporting the marine ecosystem and economic growth~\parencite{seaweed_manifesto}.

\subsection{Algae}
Algae are photoautotrophic organisms that have played and continue to play a crucial role in shaping the planet's biosphere. It is believed that four billion years ago, the Earth's atmosphere was completely different from what we know today; it was a hostile environment devoid of oxygen where life was almost impossible. This changed when the first cyanobacteria appeared, utilizing light and pigments such as chlorophyll a and phycobiliproteins to transform \ch{CO2}, producing the oxygen that enabled life to thrive and creating a world rich in diverse living beings~\parencite{mayfield_algae_2021}.

For centuries, algae have been utilized by humans for various purposes, serving economic and social roles. For example, algae were an important resource in prehistoric times, likely for food or commercial purposes. References dating back around \num{5000} years demonstrate their use in traditional Chinese and Aboriginal medicine. Their use in human diet dates back to the \zxriv{4th} century in Japan~\parencite{jacquin_Selected_2014} and ancient Greeks utilized these organisms as fertilizers, while their use as fertilizer was a common practice for ancient Romans~\parencite{jacquin_Selected_2014}.

Algae are a group of autotrophic, unicellular organisms that, thanks to their ability to adapt to various environmental conditions, can be found in diverse habitats ranging from extreme conditions to more favorable environments. They can be found in saline oceans, freshwater lakes, rivers, and ponds, as well as in various types of soil and rocks, or in areas with freezing temperatures (such as the Himalayas) to those with the hottest temperatures (deserts). They are also often found in symbiotic association with plants and animals~\parencite{rindi_Diversity_2007}. Algae are often found in symbiotic association with plants and animals. They possess a flagellum and have a well-developed nucleus, a cell wall, and a chloroplast containing chlorophyll and other pigments. Their shape varies from small unicellular organisms (\qty{1}{\micro\metre}) to large multicellular forms such as kelp (approximately \qty{60}{\metre})~\parencite{sahoo_Algae_2015} and based on this, they are classified into two different groups: Microalgae, visible only under a microscope, and Macroalgae, visible to the naked eye. Microalgae are traditionally classified based on their cytological and morphological aspects, type of reserve metabolites, components of the cell wall, and pigments. For example, cyanobacteria (blue-green algae) contain chlorophyll a and blue phycocyanins, while the brownish colorations given by xanthophyll pigments are typical of diatoms. On the other hand, macroalgae are classified based on their chemical and morphological characteristics, particularly the presence of specific pigments. In this case, we have red algae (Rhodophyceae), characterized by phycoerythrin and phycocyanin; brown algae (Phaeophyceae) for the presence of fucoxanthin, and green algae (Chlorophyceae) containing chlorophyll a and b~\parencite{scieszka_Algae_2019}.

These organisms possess a vegetative body called a thallus, which lacks the differentiation of typical terrestrial plants into roots, stems, and leaves, and the peculiar vascular system. Furthermore, this is a distinctive organ among the various species. Algae, therefore, based on the morphology of the thallus, exhibit different shapes and consistencies: they can be filamentous, cartilaginous, laminar, spongy, calcareous, or they can have a cylindrical axis or even be flattened.

Moreover, these organisms exhibit two types of reproduction: sexual and asexual. The former requires a fusion between sex cells through meiosis and is divided into isogamy if both gametes are mobile and anisogamy if the male gamete encounters the female one. As for asexual reproduction, which does not require fertilization between two gametes, it can occur through thallus fragmentation, propagules, or by spores~\parencite{pereira_Macroalgae_2021}.


\subsection{Algae as a renewable resource: cultivation methods, uses, and benefits for humans and the environment}
Green cities play a crucial role in the battle against the climate emergency and in ensuring a sustainable future for the next generations. Therefore, in facing future challenges, cities must be at the forefront, focusing their efforts on the development of ecological communities. One of the paramount issues is the integration of renewable energy sources and the implementation of an intelligent energy distribution system, ensuring a fair distribution of energy resources~\parencite{chew_Algae_2021}.

Algae are a crucial group both for the ecosystem and for the organisms inhabiting it. In fact, besides being photosynthetic, they are also an important direct source of nutrition for many species, including humans. Apart from their significant nutritional qualities, algae have considerable ecological importance. They play a fundamental role in transitioning towards a more sustainable and environmentally responsible society, representing a renewable resource of particular interest due to their ability to grow rapidly and regenerate continuously. This capability offers the potential to reduce dependence on non-renewable energies and limit the environmental impact associated with their extraction and combustion. Furthermore, algae offer multiple possibilities for use in critical sectors such as food, energy, and industry, making them a valuable asset for sustainable development.


\subsubsection{Cultivation system}
\zxriv{Rivedere posizionamento figure}

Referring to the energy use of algae, a \gls{lca} study has highlighted how the impacts from the cultivation phase are the main contributors to environmental exacerbations~\parencite{clarens_Environmental_2010}. In this regard, it is appropriate to have an overview of the different methods of algal cultivation. These are mainly divided into open (\zxriv{Fig. 1A}) and closed systems. The former are preferred for large-scale commercial production, as they are less expensive to install, manage, and maintain~\parencite{roselet_Comparison_2013}. Nevertheless, due to their open nature, they are subject to risks of contamination and evaporation, limiting control over environmental parameters such as temperature, salinity, and irradiation, and require extensive space. On the other hand, closed systems are characterized by more precise control over environmental conditions, which improves control over species composition and growth conditions. This leads to an overall higher biomass and/or lipid yield and a higher energy density of harvested algae, reducing space requirements, increasing light availability, and decreasing contamination issues. However, closed systems are more complex and expensive to build and operate, making them less suitable for large-scale commercial production as they are difficult to scale up to meet production demands~\parencite{resurreccion_Comparison_2012}. Furthermore, they are subject to bio-fouling, overheating, growth of benthic algae, cleaning issues, and high accumulation of dissolved oxygen, leading to growth limitations~\parencite{narala_Comparison_2016}.

\begin{figure}[H]
\centering
% Subfigure 1
	\subcaptionbox%
		[]%
		{Open Pond culture system\label{subfig:}}%
		{\includegraphics[width=\tresubfigwth]{example-image}}%
\hspace*{\trehbtwsfig}%
% Subfigure 2
	\subcaptionbox%
		[]%
		{Photobioreactor cultivation system\label{subfig:}}%
		{\includegraphics[width=\tresubfigwth]{example-image}}%
\hspace*{\trehbtwsfig}%
% Subfigure 3
	\subcaptionbox%
		[]%
		{Two-stage microalgae cultivation system (hybrid)\label{subfig:}}%
		{\includegraphics[width=\tresubfigwth]{example-image}}%
\caption%
[]%
{Algae culture system}
\label{fig:}
\end{figure}


In terms of algal cultivation methods, there are various techniques utilized, including cultivation in open ponds, tubular or flat-panel photobioreactors, and hybrid systems that combine elements of both. Each method has its own characteristics in terms of efficiency, cost, and environmental control, and the choice depends on the specific project requirements and available resources.

Open ponds are large shallow tanks where algae are cultivated using sunlight and nutrients present in the water. They fall under open systems because they are not completely enclosed and are exposed to the surrounding environment. They are easy to manage and construct but are, of course, susceptible to contamination and environmental fluctuations.

Photobioreactors (\zxriv{Fig. 1B}) are closed systems consisting of transparent tubes or panels containing algae, allowing for greater control of growth conditions such as temperature, light, and nutrients. They offer more precise environmental control and greater protection from external contamination, but they are more expensive to build and require higher maintenance.

Hybrid systems (\zxriv{Fig. 1C}) combine elements of both open and closed systems to harness the advantages of each, thus separating biomass growth from lipid accumulation. Algae are initially cultivated in an open environment, such as ponds or lakes, to utilize sunlight and reduce initial costs. Subsequently, the algae can be transferred to closed photobioreactors for a more controlled growth phase. In this configuration, the open phase provides ample area for initial cultivation and easy access to sunlight, while the closed phase allows for greater control of growth conditions such as temperature, \ch{CO2} concentration, and nutrition. A recent LCA found that hybrid cultivation has a reduced environmental impact compared to open and closed systems, making it preferable to these alternatives~\parencite{narala_Comparison_2016}.


\subsection{Algae, ecological importance: uses and benefits}
Algae play a fundamental role in maintaining environmental balance and promoting the health of our planet, offering a range of essential benefits for the ecosystem.

One example is their role as renewable resources in biofuel production, following the rapid increase in oil prices, depletion of reserves, and growing awareness of the environmental damage caused by the use of fossil fuels. Attention is shifting towards the development of alternative technologies that are more robust and secure. Furthermore, as the population grows, so does the energy demand, making it increasingly challenging to meet this demand with current energy sources~\parencite{faruk_role_2023}.

In the biofuel market, the potential of algae is increasingly growing, thanks to their characteristics as energy producers. They have a simple cellular structure and are rich in lipids (\zxriv{from 40\% to 80\%} of dry weight), producing a large quantity compared to traditional crops. Moreover, they have a very fast reproductive rate, and algal biofuels are biodegradable, non-toxic, and sulfur-free. Additionally, algae can convert almost all the energy contained in biomass residues and waste into methane and hydrogen~\parencite{suganya_Macroalgae_2016}. We know that the combustion of fossil fuels releases \ch{CO2} and other greenhouse gases, whereas algae are capable of absorbing \ch{CO2} from gases emitted by factories and power plants, producing significant amounts of biomass. To convert it, this is dehydrated and subjected to acid pretreatment, after which its carbohydrates are broken down into monomers. The sugars released are then converted into other products, including ethanol or combustible hydrocarbons or chemicals~\parencite{salami_AlgaeBased_2021}.

Estimates from the Pacific Northwest National Laboratory model have suggested that algal biofuels, particularly biodiesel, have the potential to meet up to 17\% of the demand for transportation fuel~\parencite{dalrymple_Wastewater_2013}. By cultivating algae as a source of fuel, it is possible to slow down the increase in atmospheric and oceanic \ch{CO2}, thereby limiting the rise in global temperatures~\parencite{raven_possible_2017}. Additionally, these organisms can enhance the sequestration of organic carbon for extended periods, depositing it on the seabed and in aquatic sediments. Thus, along with the reduction of \ch{CO2} and temperature increases, ocean acidification can also be mitigated~\parencite{prasad_Role_2021}.

In addition to providing lipids for fuel production, algae farming and cultivation also play a role in mitigating these changes. In fact, it is possible for a 1~hectare algae pond to sequester one ton of \ch{CO2} per day~\parencite{proksch_growing_2013}.

Algal cultivation offers numerous agricultural advantages compared to conventional plants, including:
\begin{enumerate}
\item high yield per unit of land, allowing for optimized land use;
\item efficient water usage for biomass production;
\item full utilization of the plant for various applications;
\item significant production of proteins, lipids, and vitamins per unit of land;
\item utilization of carbon as the primary resource for growth~\parencite{sivakumar_role_2013s}.
\end{enumerate}

Furthermore, algae exhibit greater energy efficiency compared to food crops, yet they do not compete with them. Moreover, their photosynthesis occurs at a rate three to five times faster than that of plants~\parencite{salami_AlgaeBased_2021}.

Algal cultivation, not competing with others, indeed does not require traditional agricultural land; they can be grown in marine or lagoon environments, with wastewater or saline water, thereby reducing the impact on freshwater resources~\parencite{narala_Comparison_2016}. Additionally, algae play an important role as fertilizers for agriculture. They are rich in minerals and nutrients useful for this purpose and have the ability to absorb and retain water in the soil, particularly beneficial in dry and arid lands. Moreover, algae can fix atmospheric nitrogen, a crucial component for plant growth, making it available for them. Another favorable aspect is that these organisms confer greater resistance to diseases and insects~\parencite{garima_diverse_2015}.

Furthermore, it has been observed that crops of fruits, vegetables, and flowers treated with algal fertilizers exhibited increased vigor, improved nutrient absorption, higher yields, successful seed germination, and better preservation~\parencite{rupawalla_Algae_2021}.

Another advantageous use of algae is in the treatment of wastewater, which is also responsible for greenhouse gas emissions. In recent years, to reduce environmental impact, there has been a need to research processes that reduce energy consumption and seek eco-friendly and sustainable alternatives to replace those already in use.

The strategic use of algae in wastewater treatment is based on their ability to utilize both organic and inorganic carbon sources, as well as the nitrogen and phosphorus elements present in inorganic form in wastewater, to fuel their growth process. This mechanism leads to a decrease in the concentration of such compounds within the water body. In this synergy, the main benefit of integrating algae into the wastewater treatment process lies in the production of oxygen through photosynthesis. This \ch{O2} is crucial for the heterotrophic bacteria involved in the biodegradation process of organic materials present in wastewater, thereby accelerating the degradation rate of carbon compounds~\parencite{mohsenpour_Integrating_2021}.

Algae utilize pollutants as energy and nutrient resources for their growth, producing biomass. The process of bioremediation of pollutants by microorganisms, included in treatment operations, occurs through two main mechanisms: bioaccumulation and bioabsorption, followed by subsequent biodegradation. In bioaccumulation, pollutants are absorbed inside the cell to be metabolized, while in bioabsorption, pollutants bind to the cell membrane forming complexes, which are then removed from the environment. The latter mechanism, in particular, represents an effective strategy for the removal of heavy metals from wastewater, as algal cells can incorporate such metals into their growth and development. Algal varieties demonstrate the ability to metabolize a wide range of pollutants, both organic and inorganic, present in wastewater, producing intermediate metabolites through the action of algal enzymes. The algal biomass generated during this process can be considered a source of renewable energy, indicating algae-mediated wastewater treatment as a highly recommended practice~\parencite{bhatt_Algae_2022}.

Recently, there has been a growing interest in using algae for the production of bioplastics, offering a biomass rich in hydrocarbons useful for extracting high-purity cellulose, a key material in their production. These organisms are a promising renewable source, capable of doubling their biomass in a single day and growing in a variety of environments without the need for cultivable land. Scientific research is focusing on potential algal species and the compounds derived from them, such as \gls{pha}, which can be used to develop bioplastics with various industrial applications, including cosmetics, pharmaceuticals, food packaging, and medicine. Furthermore, studies are exploring the possibilities of using these organisms to create a sustainable circular economy, thus contributing to reducing the use of petroleum-derived plastics~\parencite{dang_Current_2022}. Therefore, we can say that algae play a fundamental role in human life and in the mitigation of pollution and critical effects caused by global changes.


\subsubsection{Algae, properties and uses for human health and in future food}
Scientific studies have demonstrated how algae play a crucial role in promoting human health, thanks to their biochemical composition rich in nutrients and bioactive compounds. Among the many areas where these organisms prove useful, cosmetics stand out, where they serve as antioxidants, thickeners, and binding agents. By stimulating skin elasticity and renewal, algae have anti-aging and anti-cellulite properties, can cleanse, tone, and detoxify the skin, as well as increase its brightness and hydration. They have anti-inflammatory and regenerative properties, thus contributing to reducing wrinkles by acting as moisturizers, and additionally have softening and shining effects on hair~\parencite{garima_diverse_2015}.

It is well known that algae are a rich source of metabolites with different activities beneficial to human health, including anti-HIV, antifungal, antitumor, antimalarial, and antimicrobial properties. They are rich in antioxidants, which prevent oxidative damage by eliminating free radicals and reactive oxygen species, thus limiting the onset and formation of tumor cells. Moreover, they are extremely useful compounds in the fight against chemical agents and diseases such as atherosclerosis, cardiovascular disorders, aging processes, and cancer. 

Research on the pharmacological properties of algae has become increasingly important, especially considering the growing resistance to antibiotics and the need for new sources of antimicrobial drugs. Many of these organisms produce antibiotic substances capable of inhibiting bacteria, viruses, fungi, and other organisms. It seems that the antibiotic characteristic depends on various factors, including the specific algae species, microorganisms, season, and growth conditions~\parencite{raja_biological_2013}.

It has been demonstrated that some pigments, such as phycocyanin found in cyanobacteria, are beneficial in the treatment of diseases such as Alzheimer's and Parkinson's. They can also treat ulcers and reduce the onset of heart diseases~\parencite{subhashini_Molecular_2004}. Algae have many other medical properties in addition to those listed so far, and they also have crucial nutritional functions beneficial to human health. Considering omega-3 fatty acids, it has long been known that these polyunsaturated lipids play a fundamental role in human physiology, influencing various aspects of health, including the development of the nervous system and cardiovascular function. It has always been thought that these oils were produced by fish, but this is not the case; algae play a crucial role in the production of these fatty acids, which are then transferred along the food chain to reach humans through the consumption of fish products~\parencite{mayfield_algae_2021}.

Thanks to their nutritional value, algae are commonly used as dietary supplements, and their inclusion in the diet provides a healthy protein intake. Additionally, these organisms promote the body's detoxification process, preserve the integrity of the gastric mucosa, and facilitate digestion~\parencite{scieszka_Algae_2019}. To consider algae as a new renewable food source, a crucial factor to consider is obviously their nutritional content, which varies based on various growth factors such as environment, light, and temperature. Moreover, it also varies from species to species.

One of the most abundant compounds in algae is protein; in fact, approximately 50\% of the \zxriv{dry weight (dw)} of various species can consist of them. The highest protein contents are found in \zxriv{Spirulina} (50–70\% dw) and \species{\zxriv{Chlorella}} (51–58\% dw). These macromolecules are essential for the human diet and provide most of the nitrogen and amino acids needed by humans. Essential amino acids are referred to, which are fundamental organic molecules that cannot be synthesized endogenously and must therefore be incorporated through diet. Algal proteins are rich in essential amino acids, making them a complete protein source~\parencite{torres-tiji_Microalgae_2020}.

The lipids found in algae are relatively modest in quantity, constituting only about 1-5\% of their dry weight. Despite their modest presence, they provide a significant contribution to human health as a low-energy food source. Approximately half of the lipids present consist of polyunsaturated fatty acids, including \gls{epa} and \gls{aa}. These have been shown to have beneficial effects on health, helping to regulate blood pressure, blood clotting, and reducing the risk of chronic diseases such as cardiovascular diseases, osteoporosis, and diabetes. Red and brown algae are particularly rich in these fatty acids, while green algae tend to contain predominantly hexadecatetraenoic, oleic, and palmitic fatty acids. We know that lipids and fatty acids are essential components of cells and are precursors to many essential molecules, making their adequate intake fundamental to the human diet.

Algae are also rich in essential minerals and trace elements, which are crucial for health as they contribute to tissue formation and regulate many vital reactions as cofactors of metalloenzymes. Minerals are absorbed by these organisms through their cell surface polysaccharides, allowing them to accumulate significant amounts of minerals from the surrounding environment. The mineral composition of algae varies depending on the type, season, geographic location, and cultivation method, but they can contain calcium, phosphorus, magnesium, potassium, sulfur, and sodium. Therefore, these organisms represent a valuable source of minerals and trace elements and are used as dietary supplements to ensure an adequate intake of essential nutrients.

Vitamins, considered fundamental catalysts for human metabolism, must be obtained through the diet as the human body has limited capacity to synthesize them independently. Algae have been identified as a rich source of various vitamins, including vitamin A, C, B-group vitamins (B1 thiamine, B2 riboflavin, B3 niacin, B6 pyridoxine, B12 cobalamin), and vitamins E and K, along with a variety of carotenoids. Red and brown algae are particularly rich in B-group vitamins, while brown algae tend to have a higher content of vitamin E compared to others. Generally, the levels of vitamins in these organisms vary depending on the species, harvesting period, and environmental conditions. Algae, therefore, can constitute a significant source of vitamins in the human diet, contributing to maintaining an optimal nutritional status~\parencite{tiwari_Seaweed_2015}.

Malnutrition is a widespread issue affecting over two~billion people worldwide, with one in nine suffering from chronic hunger, lacking adequate protein and calorie intake. The global population is steadily increasing and is projected to reach \num{9,7}~billion by 2050, thus increasing the demand for food by 60\%. This puts pressure on the already limited resources of the planet, threatening global food security. Additionally, current food systems contribute to environmental problems such as pollution and climate change. To address these challenges, it is necessary to transform food systems to make them more sustainable and capable of meeting global food demand~\parencite{hosseinkhani_Key_2022}.

Furthermore, excessive meat consumption is associated with a series of negative impacts on the environment, animal welfare, and human health. This is primarily due to greenhouse gas emissions, intensive land use, and disruption of biogeochemical cycles such as phosphorus and nitrogen, which contribute to global warming and biodiversity loss. Additionally, consumption of red and processed meat has been linked to various health risks, including an increased risk of developing diseases~\parencite{michel_multinational_2021}.

For these reasons, reducing current meat consumption could lead to significant benefits for the environment, animal welfare, and human health. Therefore, we can say that algae represent a promising solution, as they are rich in nutrients and bioactive compounds essential for a healthy diet and are increasingly being considered as potential foods or food ingredients.
